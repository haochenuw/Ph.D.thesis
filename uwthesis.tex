%  ========================================================================
%  Copyright (c) 1985-2014 The University of Washington
%
%  Licensed under the Apache License, Version 2.0 (the "License");
%  you may not use this file except in compliance with the License.
%  You may obtain a copy of the License at
%
%      http://www.apache.org/licenses/LICENSE-2.0
%
%  Unless required by applicable law or agreed to in writing, software
%  distributed under the License is distributed on an "AS IS" BASIS,
%  WITHOUT WARRANTIES OR CONDITIONS OF ANY KIND, either express or implied.
%  See the License for the specific language governing permissions and
%  limitations under the License.
%  ========================================================================
%

% Documentation for University of Washington thesis LaTeX document class
% by Jim Fox
% fox@washington.edu
%
%    Revised for version 2015/03/03 of uwthesis.cls
%
%    This document is contained in a single file ONLY because
%    I wanted to be able to distribute it easily.  A real thesis ought
%    to be contained on many files (e.g., one for each chapter, at least).
%
%    To help you identify the files and sections in this large file
%    I use the string '==========' to identify new files.
%
%    To help you ignore the unusual things I do with this sample document
%    I try to use the notation
%       
%    % --- sample stuff only -----
%    special stuff for my document, but you don't need it in your thesis
%    % --- end-of-sample-stuff ---


%    Printed in twoside style now that that's allowed
%
 
\documentclass [11pt, proquest] {uwthesis}[2015/03/03]

\usepackage{thesismacros}
\usepackage{algpseudocode}
\usepackage{algorithmicx}
\usepackage{algorithm}
\usepackage{url}
\renewcommand{\algorithmicrequire}{\textbf{Input:}}
\renewcommand{\algorithmicensure}{\textbf{Output:}}
\usepackage{amsmath,mathtools}

\usepackage[textwidth=50,textsize=tiny]{todonotes}
\setlength{\marginparwidth}{2cm}
\newcommand{\tinytodo}[2][]
{\todo[caption={#2}, #1]{\renewcommand{\baselinestretch}{0.5}\selectfont#2\par}}
\newcommand{\Katetodo}[1]{\tinytodo[color=green!20]{#1}}
\newcommand{\Haotodo}[1]{\tinytodo[color=red!20]{#1}}
\newcommand{\Kristintodo}[1]{\tinytodo[color=blue!20]{#1}}

\renewcommand{\algorithmicrequire}{{\bf Input:}}
\renewcommand{\algorithmicensure}{{\bf Output:}}


%
% The following line would print the thesis in a postscript font 

% \usepackage{natbib}
% \def\bibpreamble{\protect\addcontentsline{toc}{chapter}{Bibliography}}

\setcounter{tocdepth}{1}  % Print the chapter and sections to the toc
 

% ==========   Local defs and mods
%

% --- sample stuff only -----
% These format the sample code in this document


\usepackage{alltt}  % 
\newenvironment{demo}
  {\begin{alltt}\leftskip3em
     \def\\{\ttfamily\char`\\}%
     \def\{{\ttfamily\char`\{}%
     \def\}{\ttfamily\char`\}}}
  {\end{alltt}}
 
% metafont font.  If logo not available, use the second form
%
% \font\mffont=logosl10 scaled\magstep1
\let\mffont=\sf
% --- end-of-sample-stuff ---
 



\begin{document}
 
% ==========   Preliminary pages
%
% ( revised 2012 for electronic submission )
%

\prelimpages
 
%
% ----- copyright and title pages
%
\Title{Computational aspects of modular parametrizations \\ of elliptic curves}
\Author{Hao Chen}
\Year{1985-2014}
\Program{UW Information Technology}

\Chair{Name of Chairperson}{Title of Chair}{Department of Chair}
\Signature{First committee member}
\Signature{Next committee member}
\Signature{etc}

\copyrightpage

% \titlepage  

% --- sample stuff only -----
% unusual footnote not found in a real thesis
% You just use the \titlepage as commented out above

{\Degreetext{A dissertation%
  \footnote[2]{an egocentric imitation, actually}\\
  submitted in partial fulfillment of the\\ requirements for the degree of}
 \def\thefootnote{\fnsymbol{footnote}}
 \let\footnoterule\relax
 \titlepage
 }
\setcounter{footnote}{0}

% --- end-of-sample-stuff ---
 
%
% ----- signature and quoteslip are gone
%

%
% ----- abstract
%


\setcounter{page}{-1}
\abstract{
Abstract goes here.
}
 
%
% ----- contents & etc.
%
\tableofcontents
%\listoffigures
%\listoftables  % I have no tables
 
%
% ----- glossary 
%
\chapter*{Glossary}      % starred form omits the `chapter x'
\addcontentsline{toc}{chapter}{Glossary}
\thispagestyle{plain}
%
\begin{glossary}
\item[argument] replacement text which customizes a \LaTeX\ macro for each particular usage.
\end{glossary}
 
%
% ----- acknowledgments
%
%\acknowledgments{% \vskip2pc
  % {\narrower\noindent
%  The author wishes to express sincere appreciation to
%  University of Washington, where he has had the opportunity
%  to work with the \TeX\ formatting system,
 % and to the author of \TeX, Donald Knuth, {\it il miglior fabbro}.
  % \par}
%}

%
% ----- dedication
%
\dedication{\begin{center} to all of you\end{center}}

%
% end of the preliminary pages
 
 
 
%
% ==========      Text pages
%

\textpages
 
 
% ========== Chapter 1

\chapter{Fourier expansions of cuspidal modular forms forms at cusps}

Let $k$ be a positive even integer and let $f \in S_k(\Gamma_0(N))$ be a nonzero cusp form.
We are concerned with the problem of computing the Fourier expansion of $f$ at cusps of width one
other than the cusp $[\infty]$. Note that such cusps exist if and only if  $N$ is 
not square-free. We will give two algorithms, one numerical and the other exact, to compute such
expansions. The question is studied in the Ph.D. thesis of Christophe Delaunay. We draw insight from another preprint by 
F.Brunault. The question is also studied in [Edixhoven], where numerical algorithm is given. The algorithm 
in [Ed] for computing expansions requires working at a higher level: to compute expansions at cusps of denominator $Q$, one needs to compute period matrices for forms of level $NR^2$, where 
$R = \gcd(Q, \frac{N}{Q})$. As a contrast, our algorithm works at levels dividing $N$. 

\section{Preliminaries}

Let $N \geq 1$ be an integer and let $X_0(N)$ be the modular curve of level $N$. 
\begin{Definition}
Let $z \in \bQ \cup \{\infty\}$ be a cusp on $X_0(N)$. 
Write $z = [a/c]$ with $\gcd(a,c) =1$. The {\it denominator} of $z$ is $$d_z  = \gcd(c,N).$$  As a convention, we set $d_\infty = N$. Choose $\alpha \in SL_2(\bZ)$ such that $\alpha(\infty) = z$. 
The {\it width} of $z$ is 
\[
	h_z = \left| \frac{SL_2(\bZ)_\infty}{(\alpha^{-1} \{\pm I\} \Gamma_0(N) \alpha)_\infty}\right|
\]
where the subscript $\infty$ means taking  the isotropy subgroup of $\infty$ in the corresponding group.
\end{Definition}

The width of a cusp can be computed in terms of its denominator. In fact, we have
\begin{Lemma}
If $z$ is a cusp on $X_0(N)$, then
$$h_z = \frac{N}{\gcd(d_z^2, N)}.$$
\end{Lemma}

\begin{proof}
When $z = [\infty]$, we have $d_\infty = N$ and $h_\infty =  1$, so the formula holds trivially. Otherwise, write $z = [\frac{a}{c}]$ and find $\alpha = \abcd{a}{b}{c}{d} \in SL_2(\bZ)$. For $N' \in \bZ$ we compute 
\[
	\alpha   \abcd{1}{N'}{0}{1} \alpha^{-1} = \abcd{*}{*}{-c^2N'}{*}.
\]
Hence $\abcd{1}{N'}{0}{1}  \in  (\alpha^{-1} \{\pm I\} \Gamma_0(N) \alpha)_\infty \iff N \mid c^2 N' \iff \frac{N}{\gcd(d_z^2, N)} \mid N'$. This completes the proof.
\end{proof}

In particular, the width of a cusp $z$ is one if and only if $N \mid d_z^2$. 

Suppose $f$ be a cusp form on $\Gamma_0(N)$ and $\alpha \in SL_2(\bZ)$. Then $f[\alpha]$ is a cusp form on $\Gamma(N)$. So $f[\alpha]$ has a $q$-expansion, which is a power series in $q^{\frac{1}{N}}$. A natural thing to do is to define the expansion of $f$ at the cusp $z$ as the expansion of $f | [\alpha]$. However, one must note that this may not be well-defined: the expansion depends on the choice of $\alpha$. Nonetheless, when the denominator of the cusp is sufficiently divisible by the prime divisors of $N$, the expansion is well-defined as a power series in $q$. 

\begin{Lemma}
Let $z$ be a cusp on $X_0(N)$ with width one. Choose $\alpha \in SL_2(\bZ)$ such that 
$\alpha(\infty) =z$. Then $f | [\alpha]$ is a cusp form on $\Gamma_1(N)$. Moreover, the function $f|[\alpha] $ is independent of the choice of $\alpha$. 
\end{Lemma}

\begin{proof}
It is easy to verify that $\Gamma_1(N) \subseteq \alpha^{-1} \Gamma_0(N) \alpha$, hence the first claim holds. Now suppose $\beta \in SL_2(\bZ)$ is such that $\beta(\infty) = z$. Then $\alpha^{-1} \beta \in SL_2(\bZ)_\infty$. Since $z$ has width one, we have $\alpha^{-1} \beta \in \alpha^{-1}\Gamma_0(N) \alpha$. Hence $\beta \in \Gamma_0(N) \alpha$, and it follows that $f | [\beta] = f | [\alpha]$.
\end{proof}

In light of the lemma above, we define the {\it $q$-expansion of $f$ at a width one cusp $z$} to be the $q$-expansion of $f | [\alpha]$, and denote it by $f_z$. 


Assume further that $f$ is an eigenform under the Atkin-Lehner operators. We will show that in order to compute the expansion of $f|[\alpha]$ for any $\alpha \in SL_2(\bZ)$, it suffices to do so for $\alpha = \abcd{1}{0}{m}{1}$, where $0 \leq m < N$ and $N \mid \gcd(m,N)^2$. In particular, it suffices to compute the expansions of $f$ at a some cusps of width one.

\begin{Lemma}
For any $\alpha \in SL_2(\bZ)$, there exists a matrix $w_Q \in W_N$ and an upper triangular matrix $u \in GL_2(\bQ)$
such that $w\alpha = \alpha' u$, where $\alpha' = \abcd{a'}{b}{c'}{d'} \in SL_2(\bZ)$ satisfies $N \mid \gcd(N,c')^2$.
\end{Lemma}

Indeed, one may find $Q$ using Lemma. Now $f|[\alpha] = f|[w_Q][w_Q\alpha] = f|[w_Q][\alpha'][u] = \lambda_Q(f) f [\alpha'][u] =\lambda_Q(f) f[\alpha''][u]$, where $\alpha''$ is of form $\abcd{1}{0}{m}{1}$. Note that for an upper triangular matrix $u = \abcd{u_0}{u_1}{0}{u_2}$, we have $f[u](q) = f(q^{u_0/u_2} e^{2\pi i u_1/u_2})$.

\section{Reducing to the case of newforms}

The space $S_k(\Gamma_0(N))$ is spanned by elements of form $g(q^d)$, where $g$ is newform of level $M \mid N$ and $d$ is a divisor of $\frac{N}{M}$.  Note that $g(q^d) = d^{-k/2} g | \abcd{d}{0}{0}{1}$. For any $\alpha \in SL_2(\bZ)$, 
we can find $\alpha' \in SL_2(\bZ)$ and $u \in GL_2(\bQ)$ such that $\abcd{d}{0}{0}{1} \alpha  = \alpha' u$. Hence to compute all expansions $f | [\alpha]$, it suffices to give an algorithm for newforms.

In the rest of this chapter, we will restrict ourselves to solving the following problem: 

\begin{problem}
Let $f$ be a normalized newform in $S_2(\Gamma_0(N))$ and $z$ be a cusp on $X_0(N)$ of width one. Compute the $q$-expansion of $f_z$.
\end{problem}


\section{Twists of newforms}

For $f \in S_k(\Gamma_1(N), \epsilon)$ a newform with expansion $f = \sum_n a_n(f) q^n$ and $\chi$ a Dirichlet character, the {\it twist} $f_\chi$ is a modular form with expansion $f_\chi (q) = \sum a_n(f) \chi(n)  q^n$. 

\begin{Lemma}[Winnie Li, Proposition 3.1]
Let $F \in S_k(\Gamma_1(N), \epsilon)$, where $\epsilon$ is a character of conductor $N'$.  Let $\chi$ be a character modulo $M$. Put $\tilde{N} = lcm{N, N'M, M^2}$. Then $F_\chi \in S_k(\Gamma_1(\tilde{N}), \epsilon \chi^2)$.
\end{Lemma}

In particular, when $\epsilon$ is the trivial character and the conductor $M$ of $\chi$ satisfies $M^2 \mid N$, we have
$F_\chi \in S_k(\Gamma_1(N), \chi^2))$.

We write $f \otimes \chi$ for the unique newform such that $a_p(f \otimes \chi) = a_p(f_\chi)$ for all but finitely many primes $p$. From now, we refer to $f \otimes \chi$ as {\it the twist of $f$ by $\chi$}. 

We quote two results from [Winnie], which we will use extensively. First, we recall the definitions of  $U_d$ and $B_d$ operators. For a modular form $f = \sum a_n q^n$ and a positive integer $d$, we put 
\[
	f |U_d = \sum a_{nd} q^n, f |B_d = \sum a_n q^{nd}.
\]





\begin{Lemma}[Winnie Li, Theorem 3.1]
Let $q \mid N$ and $Q$ be the $q$-primary part of $N$. Write $N = QM$. Let $F$ be a newform in $S_k(\Gamma_1(N), \epsilon)$ with $\cond(\epsilon_Q) = q^{\alpha}, \alpha \geq 0$. Let $\chi$ be a character with conductor $q^{\beta}$, $\beta \geq 1$. Put $Q' = \max\{Q,q^{\alpha + \beta}, q^{2\beta}\}$. Then \\
(1)  For each prime $q' \mid M, F_\chi$ is not of level $Q'M/q$. \\
(2)  The exact level of $F_\chi$ is $Q'M$ provided (a) $\max\{q^{\alpha + \beta}, q^{2\beta}\} < Q$ if $Q' = Q$, or 
(b) $\cond(\epsilon_Q \chi) = \max \{q^{\alpha}, q^{\beta} \}$ if $Q' > Q$.
\end{Lemma}


\begin{Lemma}[Winnie Li, Theorem 3.2]
Let $q \mid N$ and $Q$ be the $q$-primary part of $N$. Write $N = QM$. Let $\chi$ be a character with conductor equal a power of $q$.  Let $F$ be a newform in $S_k(\Gamma_1(N), \epsilon)$. Then $f \otimes \chi$  is a newform in $S_k(\Gamma_1(Q'M,\epsilon \chi^2)$, where $Q'$ is a power of q, such that $$F_\chi  = f \otimes \chi - (f \otimes \chi) |U_q |B_q.$$
\end{Lemma}

Since our goal is to compute expansions of newforms on $\Gamma_0(N)$, we will make the following assumptions: 
from now, unless otherwise noted, we assume $f$ has trivial character, and that $\cond(\chi)^2 \mid N$.


Next, we consider the problem of identifying the newform $f \otimes \chi$. This includes finding its level, which we denote by $M_\chi$, and its $q$-expansion to arbitrarily many terms. We will assume that we have an oracle which, given weight $k$ and level $N$, computes the expansions of all newforms in $S_k(\Gamma_1(N))$ to arbitrarily many terms (for example, use in [Stein]).

Note that $f$ has trivial character. For such forms we have the following lemma.
\begin{Lemma}
Let $f \in S_k(\Gamma_0(N))$ and let $\chi$ be a Dirichlet character of conductor $Q$, such that $Q^2 \mid N$. Then the level $M_\chi$ of $f \otimes \chi$  satisfies $M_\chi \mid N$. 
\end{Lemma}

\begin{proof}
By [Winnie, theorem 3.1], we have $f_\chi \in S_k(N, \chi^2)$ (since $\alpha = 0$ and $q^{2\beta} \leq Q$). It now follows from theorem 3.2 that the level of $f \otimes \chi$ is a divisor of $N$.
\end{proof}

Now we proceed on how to recognise the level of $f \otimes \chi$ from the coefficients of $f$. One potential obstacle is that we do not know all Fourier coefficients of $f \otimes \chi$. We only know that $a_n(f \otimes \chi)  = a_n(f)\chi(n)$   when $\gcd(n, N) = 1$. This can be overcome using a variant of Sturm's argument. First we prove a lemma.
\begin{Lemma}
Let $f \in S_k(N, \epsilon)$ be a normalized newform. Then $f|U_q|B_q \in S_k(Nq^2, \epsilon)$.
\end{Lemma}

\begin{proof}
We use a standard fact that for any integer $d \geq 1$, the map $f \mapsto f|B_d$ takes $S_k(N, \epsilon)$ to 
$S_k(Nd, \epsilon)$. To prove the lemma, we consider two separate cases. First, assume $q \nmid N$, then we have 
$T_q = U_q + q^{k-1} \epsilon(q) B_q$. By our assumption, we have $f|T_q = a_q(f) f$. Therefore, we have 
$f|U_q|B_q = f|(T_q - q^{k-1} \epsilon(q) B_q)|B_q = a_q(f)f|B_q - q^{k-1} \epsilon(q) f|B_q^2$. Hence $f|U_q|B_q \in S_k(Nq^2, \epsilon)$. \\
Now assume $q \mid N$, so $U_q = T_q$. Hence $f|U_q|B_q = a_q(f) f|B_q \in S_k(Nq, \epsilon) \subseteq  S_k(Nq^2, \epsilon)$. 
\end{proof}

The next proposition generalised the usual Sturm bound argument for modular forms. 

\begin{Prop}
Let $g_1$, $g_2$ be two normalised newforms of levels $N_1 \mid N_2$ and the same nybentypus character $\epsilon$. Assume $\epsilon$ has prime power conductor $Q = q^\beta$ such that  $Q^2 \mid N$. Let 
$B$ be the Sturm bound for the congruence subgroup $\Gamma_1(Nq^2)$. Suppose 
\[
	a_n(g_1) = a_n(g_2), \, \mbox{for all }1 \leq n \leq B \mbox{ such that } \gcd(n,q) = 1.
\]
Then $g_1 = g_2$. 
\end{Prop}

\begin{proof}
Following [Winnie], we define the operator $K_q$ on the space of modular forms by
\[
	g|K_q = g - g|U_q|B_q. 
\]
Then the assumption is equivalent to the statement that $\delta  = (g_1 -g_2) |K_q$ has $a_n(\delta) = 0$ for all $1 \leq n \leq B$. Since $\delta \in S_k(Nq^2, \epsilon)$,  Sturm's theorem implies $\delta = 0$.

But then we know from [DS Theorem 5.7.1]  that $g_1 - g_2 \in S_k(N_2,\epsilon)^{old}$. Suppose $N_1 < N_2$, then $g_1$ is in the old subspace, hence so is $g_2$, a contradiction. Therefore we must have $N_1 = N_2$. It follows that $g_1 - g_2 \in S_k(N_2, \epsilon)^{new}$, since $g_1, g_2$ are newforms. This forces $g_1 - g_2  = 0$. 
\end{proof}

Now we are ready to describe the algorithm. 

\begin{algorithm}[H]
\caption{Identifying  $f \otimes \chi$}
\label{alg: twist}
\begin{algorithmic}[1]
    \Require $k$ -- a positive even integer; $f \in S_k(\Gamma_0(N))$ a normalized newform; $\chi$ a Dirichlet character of prime power conductor $Q = q^\beta$; $Q^2 \mid N$;  $B$ -- a positive integer
    \Ensure The level $M_\chi$ of $f \otimes \chi$ and the Fourier expansion of $f \otimes \chi$ up to $q^B$.
    \If{$Q = 1$}
    \State return $N$.
    \EndIf
    \State $Q' := \cond(\chi^2)$; $N_0 := \frac{N}{q^{v_q(N)}}$; $M_0 := Q'N_0$; $t := \frac{N}{M_0} \in \bZ$. 
    \For{each positive divisor $d$ of $t$}
    	\State Set $V_d := S_k(M_0d, \chi^2)$. 
	\State Compute a basis of newforms $\{g_1^{(d)}, \cdots g_{s_d}^{(d)}\}$ of $V_d$.
    	\State Set $B_d$ := the Sturm bound for $\Gamma_1(M_0dq^2)$. 
    	\For{$1 \leq j \leq s_d$} 
		\If{$a_n(g_i^{(d)})= a_n(f)\chi(n)$ for all $1 \leq n \leq B_d, \gcd(n,q) = 1$}
			\State return $M_0d$.
		\EndIf
	\EndFor
    \EndFor	
\end{algorithmic}
\end{algorithm}

%It is natural to define {\it $p$-minimality} of newforms. The definition mimics that of [Brunault]. 

%\begin{Definition}
%Let $f \in S_k(\Gamma_1(N))$ be a newform. Let $p$ be a prime such that $p^2 \mid N$. 
%We say $f$ is {\it $p$-minimal} if $f \otimes \chi$ is new of level $N$ for all Dirichlet character 
%$\chi: (\bZ/p^{\frac{v_p(N)}{2}}\bZ)^{\times} \to \bC^{\times}$.
%\end{Definition}

We give some sample computations applying the above algorithm. 

\begin{Example}
Let $f$ be the normalised newform attached to the elliptic curve 
\[
	E: 
\]
of label {\bf 50a}. Then $f \otimes \chi$ is new of level 50 for all Dirichlet characters $\chi$ with modulus 5. 
In other words, $f$ is 5-minimal. 
\end{Example}

As another example, we demonstrate a newform which is not $p$-minimal. 
\begin{Example}
Let $f$ be the normalised newform attached to the elliptic curve 
\[
E: y^2 + x y = x^{3} + x^{2} - 25 x - 111
\]
of label {\bf 98a}. Let $\chi$ be the Dirichlet character modulo 7 defined by $\chi(3 \pmod{7}) = -1$. 
We found that $f \otimes \chi$ is a newform of level 14, with $q$-expansion (todo: add expansion).
\end{Example}


\section{Pseudo-eigenvalues} 

Let $\epsilon$ be a Dirichlet character modulo $N$ and let $f$ be a newform in $S_k(N,\epsilon)$. For any divisor $Q$ of $N$ with $\gcd(Q, \frac{N}{Q}) =1$, there is an algebraic number  $w_Q(f)$ of absolute value one  and  a newform $g$ in $S_k(N, \overline{\epsilon_Q} \epsilon_{N/Q})$ such that 
\[
	W_Q(f) = w_Q(f) g, 
\]


\begin{Definition}
The number $w_Q(f)$ is called the pseudo-eigenvalue of $W_Q$ on $f$. We write $w(f) = w_N(f)$. 
\end{Definition}

For a power series  $f = \sum_{n \geq 0} a_nq^n$, its complex conjugate, denoted by $f^*$, is $$f^*(q) = \sum \overline{a_n}q^n.$$ From [Winnie] we have $W_N(f) = w(f) f^*$. In the rest of this section, we describe an algorithm to efficiently compute $w(f)$ numerically.  For a positive even integer $k$, let $\bM(k)$ denote the space of weight-$k$ modular symbols defined in \cite{stein2007modular}. The space $\bM(k)$ is a quotient of $\bZ[X,Y]_{k-2} \otimes \bP^1(\bQ)^2$, and $GL_2(\bQ)$ acts on $\bM(k)$ via the following rule
\[
	g (P(X,Y) \otimes \{\alpha, \beta\}) = P( g^{-1}(X,Y)^T) \{g(\alpha), g(\beta)\}.
\]
Most importantly, there is a pairing between $\bM(k)$ and the space of modular forms of weight $k$, defined as
\[
		\langle f, P(X,Y) \otimes \{\alpha, \beta\}) \rangle_k = \int_{\alpha}^{\beta} f(z) P(z,1) dz.
\]
We will suppress the subscript $k$ if its value is clear from context.


\begin{Lemma}
Let $M \in \bM(k)$ and $f \in S_k(\Gamma_1(N))$. Then 
\[
	N^{\frac{k}{2}-1} \langle f|W_N, M \rangle = \langle f, W_N M \rangle.
\]
\end{Lemma}

\begin{proof}
See proof of  \cite[Proposition 8.17]{stein2007modular}. Note that the extra factor $N^{\frac{k}{2}-1}$ is due to the different constants involved in the definition of the weight-$k$ action of $GL_2(\bQ)$ on modular forms. 
\end{proof}

The map  
\[
	*: P(x,y)\{\alpha, \beta\} \mapsto P(-x,y) \{-\bar{\alpha}, -\bar{\beta}\} 
\]
defines the {\it star involution} on the space $\bM(k)$. We have $\langle f^*, M \rangle = \overline{\langle f, M^*\rangle}.$


Now we are ready to prove the main theorem of this section.
\begin{theorem}
Let $f$ be a normalised newform on $\Gamma_1(N)$ with positive even weight $k$ and let $M \in \bM(k)$ be such that $W_N(M) = N^{k/2 -1} M^*$. Assume $\langle f, M \rangle \neq 0$.  Then 
\[
	w(f) = \frac{\langle f,M \rangle }{\overline{\langle f,M \rangle}}.
\]
\end{theorem} 

\begin{proof}
Since $W_N^2(M) = N^{k-2}M$ for all $M \in \bM(k)$, the assumption implies $W_N(M^*) = N^{k/2 -1} M$. 
Now 
\begin{align*}
& N^{k/2-1} \langle f|W_N, M^* \rangle  = \langle f,W_N(M^*) \rangle \, (\mbox{Lemma~\ref{}})\\ 
\implies & N^{k/2-1} w(f) \langle   f^*,M^* \rangle = N^{k/2-1} \langle f, M \rangle \\
\implies & w(f) = \frac{\langle f, M \rangle}{\langle f^*,M^* \rangle} \\ 
\implies &	w(f) = \frac{\langle f, M \rangle}{\overline{\langle f, M \rangle}}.
\end{align*}
\end{proof}

Suppose $\alpha, \beta \in \{z \in \bC | Im(z) > 0,  |z| = 1/\sqrt{N}\}$. Then it is easy to verify that $M = (xy)^{k/2-1} \otimes \{\alpha, \beta\}$ satisfies $W_N(M) = M^*$. Finally, we arrive at the algorithm to compute $w(f)$.




\begin{algorithm}[H]
\caption{Computing the pseudo-eigenvalue of newforms.}
\label{alg: pseudo-eigenvalue}
\begin{algorithmic}[1]
    \Require $k$ -- a positive even integer. $f \in S_k(\Gamma_1(N))$ a normalized newform.    
    \Ensure a numerical approximation of $w(f)$.
    \State $n_0 := 10$, $z_0 := \frac{i}{\sqrt{N}}$. $\delta = 10^{-3}$. 
    \State Randomly generate $n_0$ points $\{z_1, \cdots, z_{N_0}\} \subseteq \{z | 0 < Im(z) < \frac{1}{2\sqrt{N}}, |z| = \frac{1}{\sqrt{N}} \}$.
    \For{$1 \leq i \leq n_0$}
    	\State compute the period integral $c_i =  \int_{z_0}^{z_i} 2\pi i f(z) z^{\frac{k-2}{2}} dz$. 
	\State $w_i \gets c_i/\bar{c_i}$. 
    \EndFor
    \If{the standard deviation of $w_1, \cdots, w_{n_0}$ is less than $\delta$} 
     \State $w \gets \frac{1}{n_0}(\sum_i w_i)$. 
     \State \Return $w$.
    \Else
    	\State \Return {\bf FAIL}. 
    \EndIf
\end{algorithmic}
\end{algorithm}

%\begin{Remark}
%The period integral in step 4 of Algorithm~\ref{alg: pseudo-eigenvalue} is computed as follows: 
%This approach is taken from [Cre97].
%\end{Remark}


\section{Formula for the Fourier expansion of $f$ at width one cusps: Part 1}

\begin{Definition}
For a positive integer $c'$, let $S_c' = \abcd{1}{\frac{1}{c'}}{0}{1}$. If $\chi$ is a character modulo $c'$, we define the 
operator on modular forms 
\begin{equation*}
\label{formula: RS}
	f | R_\chi(c') = \sum_{u =0}^{c'-1} \bar{\chi}(u) f | S_{c'}^u.
\end{equation*}
\end{Definition}
Write $R_\chi$ in short for $R_\chi(\cond(\chi))$. Note that $f|R_\chi = g(\bar{\chi})f_\chi$.
Conversely, if $(a,M) = 1$, we have 
\begin{equation}
\label{formula: SR}
	\phi(c')[S_{c'}^u] = \sum_{\chi: cond(\chi) \mid c'} \chi(u) R_\chi(c').
\end{equation}

For our convenience, we introduce a new set of operators, which are basically the conjugates of $S_c'$ and $R_\chi(c')$ by $W_N$. Let $A_c' = \abcd{1}{0}{c'}{1}$.  Then we have 
\begin{Fact}
$-N \cdot A_{N/c'}^{-1} = W_N S_{c'} W_N$
\end{Fact}

From now on, we assume $c$ is a divisor of $N$ and $c' = \frac{N}{c}$. Then
\[
	[A_c]^{-1} = [W_N S_{c'} W_N].
\]

Since $[W_N]^2 = id$, we have
\[
	[A_c^{-u}] = [W_N S_{c'}^u W_N], \, \forall u \in \bZ. 
\]

Parallel to the notion of $R_\chi(c')$, we make the following definition.
\begin{Definition}
\[
	\Phi_\chi(c) = \sum_{u =0}^{c'-1} \bar{\chi}(u) [A_c^{-u}].
\]
\end{Definition}
Then $\Phi_\chi(c) = [W_N] R_\chi(c') [W_N]$. 
Similar to the equation~\ref{formula: SR}, we have
\begin{equation}
\label{formula}
	\varphi(c')[A_c^{-a}] = \sum_{\cond(\chi) \mid c'} \chi(a) \Phi_\chi(c) 
	 = \sum_{\cond(\chi) \mid c'} \chi(a) W_N R_\chi(c') W_N. 
\end{equation}

Finally, applying formula~\ref{formula} to $f$, we arrive at
\begin{eqnarray}
	f_{[\frac{a}{c}]} \left( q\right) &= \frac{1}{\varphi(c')}\sum_{\cond(\chi) \mid c'} \chi(a) f|[W_N R_\chi(c') W_N]. \\ 
	&= \frac{w_N(f) }{\varphi(c')}\sum_{\cond(\chi) \mid c'} \chi(-a) f| R_\chi(c') | W_N. 
\end{eqnarray}

Now it left to compute the expansions of  each $f| R_\chi(c') | W_N$ in the sum.

\section{Formula for the Fourier expansion of $f$ at width one cusps: Part 2}

In this section, we describe how to compute the expansion of $f| R_\chi(c') | W_N$. First note the following identity between operators on $S_k(\Gamma_1(N), \epsilon)$: $$T_p = U_p  + \epsilon(p) p^{\frac{k}{2}}B_p.$$

We recall some notations and a result from Delaunay's thesis (see []).
\begin{Definition}[[Delaunay, Definition III.2.4]]
Put $\cond'(\chi)$ multiplicatively. If $\chi_j$ is trivial character modulo $p_j^{a_j}$, set  $\cond'(\chi_j) = p_j$.
$\chi = \chi_{nt} \chi_{tr}$. Put 
\[
	g'(\chi) = (-1)^{|I|} \chi_{nt}(tr) g(\chi_{nt}). 
\]
\end{Definition}


\begin{Lemma}[Delaunay's thesis, Prop 2.6]
Let $c'$ be such that $c'^2 \mid N$. For a Dirichlet character $\chi$ mod $c'$, we have
$$f|R_{\chi}(c') = \begin{cases} g'(\bar{\chi}) f_{\chi_{nt}} & if \cond'(\chi) = c \\ 0 & else  \end{cases}$$
\end{Lemma}

Then we compute $f_{\chi_{nt}}$ by the following: suppose $g  = f \otimes \chi_{nt}$. Then 
\[
	f_{\chi_{nt}} = \prod_{i=1}^r g | K_{p_i}.
\]
Moreover, we have $$K_{p} = 1  - U_{p} B_p =  \begin{cases} 1- (T_p - \chi_{nt}^2(p) p^{\frac{k}{2}} B_p) |B_p & p \nmid M \\  1 - T_p |B_p & p \mid M \end{cases}.$$ Using the commutativity of $T$ and $B$, we can write  $f_{\chi_{nt}}$ in the form $\sum c_i g(q^d_i)$, where $c_i$ and $d_i$ are constants. To give a precise formula, we use the following notation. For a finite set $S$ of integers, let $p(S) = \prod_{s \in S} s$ denote the product of all elements in $S$.


\begin{theorem} \label{thm: formula1}
Let $S_\chi = \{ p_1, \cdots, p_r\}$ be the set of prime divisors of $\cond(\chi)$.
Let $\cB_{\chi,M} = \{(S_1, S_2) | S_1, S_2 \subseteq S_\chi, S_1 \cap S_2 = \emptyset, \gcd(M, p(S_2)) = 1. \}$. Write $g_\chi = f \otimes \chi$. Then
	$$f_{\chi_{nt}} =  \sum_{(S_1, S_2) \in B_{S,M}} a_{p(S_1)}(g_\chi)  p(S_2)^{k/2} \chi_{nt}^2(p(S_2)) g_\chi | B_{p(S_1) p(S_2)^2}.$$
\end{theorem}

Theorem~\ref{thm: formula1} will be our starting point of computing the expansion of $f$ at width one cusps. 
We will use it to compute $f_{\chi_{nt}} | W_N$.  First we prove two lemmas. 
\begin{Lemma} \label{lemma: bdwn}
Let $f$ be a newform of even weight $k$ on $\Gamma_1(M)$ and suppose $d, N$ are positive integers such that $Md \mid N$. Then
   $$f| B_d|W_N = \left(\frac{N}{Md^2} \right)^{k/2}  w_M(f)  \overline{f|B_{\frac{N}{Md}}}.$$
   where $w_M(f)$ is the pseudo-eigenvalue of $f$ defined in previous sections. 
\end{Lemma}
\begin{proof} Straightforward computation.
\begin{align*}
f |B_d | W_N & = d^{-k/2} f | \abcd{d}{0}{0}{1} \abcd{0}{-1}{N}{0} \\ 
& = d^{-k/2} f| \abcd{0}{-1}{M}{0} \abcd{N/md}{0}{0}{1} \abcd{d}{0}{0}{d} \\
& = \left( \frac{N}{Md^2} \right)^{k/2} f | W_M | B_{N/Md} \\
& = \left( \frac{N}{Md^2} \right)^{k/2} w_M(f) \bar{f} | B_{N/Md} \\ 
& = \left( \frac{N}{Md^2} \right)^{k/2} w_M(f) \overline{f | B_{N/Md}}.
\end{align*}
\end{proof}

Before stating the second lemma, we quote another result of Winnie Li on the coefficients of a newform at primes dividing the level.
\begin{Lemma}[Winnie Newform, Theorem 3 (iii)]  \label{lemma: winnie-vanishing}
Let $f = \sum_{n \geq 1} a_n(f) q^n$ be a normalized newform in $S_k(\Gamma_1(N), \epsilon)$, $p$ a prime dividing $N$.  Then \\
(1) If $\epsilon$ is a character modulo $N/p$ and  $p^2 \mid N$, then $a_p(f) = 0$. \\ 
(2) If $\epsilon$ is a character modulo $N/p$ and  $p^2 \nmid N$, then $a_p(f)^2 = \epsilon(p) p^{k-2}$. \\ 
(3) If $\epsilon$ is not a character modulo $N/p$, then $|a_p(f)| = p^{\frac{k-1}{2}}$.
\end{Lemma}


\begin{Lemma}
Using notations in Theorem~\ref{thm: formula1}, and assume $(S_1, S_2) \in \cB_{S,M}$ is such that $a_{p(S_1)}(g_\chi) \neq 0$. Then 
$M p(S_1) p(S_2)^2 \mid N$.
\end{Lemma}

\begin{proof}
Let $p$ be a prime divisor of $N' := M p(S_1) p(S_2)^2$. If $p \nmid M$, then $\ord_p(N') \leq \ord_p(\cond(\chi)^2) \leq \ord_p(N)$. So we assume $p \mid M$, hence $p \nmid p(S_2)$. If $p \nmid p(S_1)$, then it follows from $M \mid N$; 
if $p \mid p(S_1)$, we want to show that $\ord_p(M) < \ord_p(N)$. Suppose not, then $\ord_p(M) = \ord_p(N) \geq 2 \ord_p(\cond(\chi))$. Since $\cond(\chi^2) \leq \cond(\chi)$, we know $\chi^2$ is a character modulo $M/p$. Applying Lemma~\ref{lemma: winnie-vanishing} to the newform $g_\chi$ on level $M$, we see that $a_{p}(g_\chi) = 0$, hence $a_{p(S_1)}(g_\chi) = 0$ by multiplicativity.
\end{proof}

Applying Lemma~\ref{lemma: bdwn} to Theorem~\ref{thm: formula1}, we finally arrive at 

% Main Theorem of expansion 

\begin{theorem} \label{thm: ExpansionFormula}
Let $f$ be a normalized newform in $S_2(\Gamma_0(N))$ and $z = [a/c]$ be a cusp on $X_0(N)$ of width one. 
Then $f_z$ is 
\[
	f_z = \frac{w(f) }{\varphi(c')}\sum_{\chi: \cond'(\chi) = c'} \chi(-a) g'(\bar{\chi}) w(g_\chi) t_\chi.
\]
Here $t_\chi$ is as follows: let $M_\chi$ denote the level of $g_\chi := f \otimes \chi$. Then
\[
t_\chi = \sum_{(S_1, S_2) \in B_{S_{\chi_{nt}},M_\chi}} (-1)^{|S_1|} a_{p(S_1)}(g_\chi)  \left(\frac{N}{M_\chi p(S_1)^2p(S_2)^3} \right)^{k/2}\chi(p(S_2)) \overline{g_\chi | B_{\frac{N}{M_\chi p(S_1) p(S_2)^2}}}.
\]
\end{theorem}



This theorem gives us an algorithm to compute the expansion of $f_z$, which we describe below. But first, we take a closer look at what ingredients goes into the expansion. Given a newform $f \in S_k(\Gamma_0(N))$ and a width one cusp $z$ of denominator $c$. We need to consider the twist of $f$ by all Dirichlet characters of conductor dividing $c$. 
For each such character $\chi$, we then need to determine the level $M_\chi$ and $q$-expansion of the newform $f \otimes \chi$, the latter boils down to knowing $a_p(f \otimes \chi)$ for all $p \mid \cond(\chi)$. Then we need to compute the. Finally, we combine these information together and apply Throem~\ref{thm: ExpansionFormula} to compute $f_z$. 

\begin{algorithm}[H]
\caption{Computing Fourier coefficients of $f$ at width one cusps}
\begin{algorithmic}[1]
    \Require $f \in S_k(\Gamma_0(N))$ a newform; $a, c$ -- coprime integers such that $N \mid c^2$; $B$ -- a positive integer. 
    \Ensure The first $B$ Fourier coefficients of $f_{z}(q)$. 
    
    \State  $c' \gets N/c$. $X \gets$ The set of all Dirichlet characters $\chi$ such that $\cond'(\chi) = c'$. 
    \State compute $w(f)$  using Algorithm~\ref{alg: pseudo-eigenvalue}. 
    \For{$\chi$ in $X$}   
    	\State Compute $M_\chi$ and the expansion of $g_\chi := f \otimes \chi$ to $B$ terms, using  Algorithm~\ref{alg: twist}
    	\State Compute $g'(\bar{\chi})$. 
	\State Compute $w(g_\chi)$ using Algorithm~\ref{alg: pseudo-eigenvalue}.
    \EndFor
    \State Apply Theorem~\ref{thm: ExpansionFormula} to compute $f_z$ to $B$ terms. 
    \end{algorithmic}
\end{algorithm}

\section{A Converse Theorem}

Given the work in previous sections, it is a natural question then to ask whether the information on twists of $f$ is uniquely determined by the expansion of $f$ at width one cusps. The answer is yes, and the precise statement is in the following theorem. 

\begin{theorem}
Let $f$ be a normalized newform in $S_k(\Gamma_0(N))$. Assume the eigenvalue $w_N(f)$ is known. Suppose $c$ is a positive divisor of $N$ such that $N \mid c^2$.  Then the expansions of $f_z$, where $z$ runs through all cusps of denominator $c$, uniquely determines the following: for each Dirichlet character 
$\chi$ of such that $\cond'(\chi) = c'$, the level $M_\chi$, the pseudo-eigenvalue $w_{M_\chi}$ and the $q$-expansion of the newform $f \otimes \chi$. 
\end{theorem}

\begin{proof}
By plug in different $a$'s. We can solve for $t_\chi$.  Consider the first nonzero term of $t_\chi$. Suppose 
\[
	t_\chi = u_\chi q^{v_\chi} + O(q^{v_\chi + 1}), \, u_\chi \neq 0.
\]
Assuming that $\chi$ has prime power conductor $p^\beta > 1$, we claim that 
$$\left| \frac{v^{k/2}}{u} \right| = \begin{cases} p^{k/2} & if p \nmid M_\chi \\ p^{1/2} & if p \mid M_\chi \mbox{ and }  a_p(g) \neq 0 \\ 1 & else \end{cases}.$$

Proof of claim: the first and third case are easy to verify using Theorem~\ref{thm: ExpansionFormula}. Now assume $p \mid M$ and $a_p(g_\chi) \neq 0$. By Lemma~\ref{lemma: winnie-vanishing}, we have $|a_p(g_\chi)| = p^{k/2 - 1/2}$ or $p^{k/2 - 1}$. However, $|a_p(g_\chi)| = p^{k/2-1}$ only if $p \mid \mid M_\chi$ and $\chi^2$ is a character modulo $M_\chi/p$. This means $\chi^2$ is the trivial character. By another theorem of Winnie, we compute the $p$-level of $f = g_\chi \otimes \bar{\chi}$: note that $\max{p, p^{\alpha+\beta}, p^{2\beta}} > p$, so (ii) applies and the $p$-level of $f$ is equal to $\max(p^{\alpha}, p^{\beta})  = p^{\beta}$, i.e., $\ord_p(N) = \beta$. This is impossible since we have $p^{2 \beta} = \cond(\chi)^2 \mid N$. 

Therefore, we have $|a_p(g_\chi)| = p^{k/2-1/2}$ and the claim follows.

Since $k \geq 2$, we could determine which case we are in. Then we can read off $M_\chi$  and $w_M(g_\chi)$. For example, if we are in the second case, then the level can be computed via $M_\chi = \frac{N}{v_\chi p}$.  Now  the $N/M_\chi$'s coefficient of $t_\chi$ is 
\begin{eqnarray*}
	a_{\frac{N}{M}}(t_\chi) &= w(g_\chi) (\frac{N}{M})^{k/2} ( 1 -  |a_p(g_\chi)|^2 \chi^2(p) p^{-k/2} )  \\
	& = w(g_\chi) (\frac{N}{M})^{k/2} ( 1 - p^{k/2 -1} \chi^2(p)). 
\end{eqnarray*}

This allows us to solve $w(g_\chi)$. Finally, we compute $a_p(g_\chi)$ by $a_p(g) = \frac{-u_\chi}{w(g_\chi) \chi^2(p) (\frac{N}{Mp})^{k/2}}$. The value $a_p(g)$ determines the expansion of $g_\chi$. Recursively, we could solve for all $pn$-coefficients of $g_\chi$, from which we deduce it complete $q$-expansion. 


In the general case,  we consider the following subsets of $S_\chi$.  Let $S_1^* = \{ p \in S_\chi: p \mid M \}$, $S_2^* = 
S_\chi \setminus S_1^*$, and $\widetilde{S_1^*}= \{p \in S_1^*: a_p(g_\chi) \neq 0\}$.

It follows that the leading term of $t_\chi$ belongs to the summand corresponding to $(\widetilde{S_1^*}, S_2^*)$ in  Theorem~\ref{thm: ExpansionFormula}. Still writing the leading term as $u_\chi q^{v_\chi}$, we have 
\[
	u_\chi = w(g_\chi) \chi^2(p(S_2)) a_{p(\widetilde{S_1^*})}(g_\chi) p(\widetilde{S_1^*})^{-k} (p(S_2^*)^{-3k/2} \left(\frac{N}{M_\chi}\right)^{k/2}, v_\chi = \frac{N}{M_\chi p(\widetilde{S_1^*}) p(S_2^*)^2}. 
\]
Similar to the prime power conductor case above, we have $|a_{p(\widetilde{S_1^*})}(g_\chi)|  = p(\widetilde{S_1^*})^{k/2 -1/2}$.  So 
\begin{equation} \label{formula: converse}
	|v_\chi^{k} u_\chi^{-2}| = p(\widetilde{S_1^*}) p(S_2^*)^2.
\end{equation}
Hence we can factor $|v_\chi^{k} u_\chi^{-2}|$ and obtain $p(\widetilde{S_1^*})$
and $p(S_2^*)$. Then $M_\chi$ can be solved using $v_\chi$. Plug it back into $u_\chi$, we obtain $a_{p(\widetilde{S_1^*})} w(g_\chi)$. Finally, for each $p \in \widetilde{S_1^*}$,  the $v_\chi p$'s coefficient of 
$t_\chi$ allows us to compute $a_{p(\widetilde{S_1^*})/p}(g_\chi) w(g_\chi)$. These together determines $w(g_\chi)$ and 
$a_{p(\widetilde{S_1^*})}$. The other Fourier coefficients of $g_\chi$ can then be computed recursively. 
\end{proof}




\section{Fields of definitions}

\begin{Lemma}
Let $c$ be a cusp of denominator $d$ and let $d' = N/d$. Then 
\[
	\bQ(\{a_n(f, c)\}) \subseteq \bQ( \{a_n(f)\}, \zeta_{d'}). 
\]
\end{Lemma}

\begin{proof}
Let $K_0 = \bQ( \{a_n(f)\}$
Choose a form $0 \neq g \in S_k(\Gamma_1(N))$ with rational Fourier coefficients such that 
$h = \frac{f}{g}$ is non-constant. From [Cox, ] it is easy to see that $h \in $. Then we have 
\end{proof}



\section{Examples}

Let $E = {\bf 50a}$ and consider the 4 cusps of denominator 10 on $X_0(50)$. The corresponding first terms 
of $q$-expansions at these cusps are 

\iffalse
\begin{align*}
	a_1(f, \frac{1}{10}) &= \frac{1}{5} \zeta_{5}^{3} - \frac{3}{5} \zeta_{5}^{2} + \frac{3}{5} \zeta_{5} - \frac{1}{5} \\ 
	a_1(f, \frac{3}{10}) &= \frac{3}{5} \zeta_{5}^{3} + \frac{6}{5} \zeta_{5}^{2} + \frac{4}{5} \zeta_{5} + \frac{2}{5} \\
	a_1(f, \frac{7}{10}) &= \frac{2}{5} \zeta_{5}^{3} - \frac{1}{5} \zeta_{5}^{2} - \frac{4}{5} \zeta_{5} - \frac{2}{5}\\
	a_1(f, \frac{9}{10}) &=-\frac{6}{5} \zeta_{5}^{3} - \frac{2}{5} \zeta_{5}^{2} - \frac{3}{5} \zeta_{5} - \frac{4}{5}
\end{align*}
\fi


\section{Applications}

One applications of the computation done in this chapter is the norm method to the computation of $j$-polynomials 
introduced in Chapter~. Recall that the issue with the norm method for non-square free level is computing the 
expansions of form $f | \gamma$, where $\gamma$ runs over the set of right coset representatives of $\Gamma_0(N)$ 
in $SL_2(\bZ)$. To compute the norm of  $f$ when $N$ is non-square free, it suffices to compute the expansions of 
$f$ at all width-1 cusps. This is a consequence of the following lemma.

\begin{Lemma}
For any cusp $z$ of $X_0(N)$, there exists an Atkin-Lehner involution $w \in W(N)$ such that $z_1 = w(z)$ has 
width one.
\end{Lemma}

\begin{proof}
Let $z \neq [\infty]$ be a cusp. Recall that $z$ has width one if and only if its denominator $d(z)$ satisfies
$d(z)^2 \equiv 0 \pmod{N}$. Let $p$ be a prime divisor of $N$. Then it is easy to see that 
$v_p(d(w_p(z))) = v_p(N) -  v_p(d(z))$ and $v_l(d(w_p(z))) = v_l(d(z))$ for primes $l \neq p$. The lemma now follows by taking $w = \prod_{p \mid N: v_p(d(z)) \leq v_p(N)/2} w_p$. 
\end{proof}

\section{Norm guess and data}
 
 

 

 

\nocite{*}
\bibliographystyle{alpha}
\bibliography{uwthesis}


\end{document}
