%  ========================================================================
%  Copyright (c) 1985-2014 The University of Washington
%
%  Licensed under the Apache License, Version 2.0 (the "License");
%  you may not use this file except in compliance with the License.
%  You may obtain a copy of the License at
%
%      http://www.apache.org/licenses/LICENSE-2.0
%
%  Unless required by applicable law or agreed to in writing, software
%  distributed under the License is distributed on an "AS IS" BASIS,
%  WITHOUT WARRANTIES OR CONDITIONS OF ANY KIND, either express or implied.
%  See the License for the specific language governing permissions and
%  limitations under the License.
%  ========================================================================
%

% Documentation for University of Washington thesis LaTeX document class
% by Jim Fox
% fox@washington.edu
%
%    Revised for version 2015/03/03 of uwthesis.cls
%
%    This document is contained in a single file ONLY because
%    I wanted to be able to distribute it easily.  A real thesis ought
%    to be contained on many files (e.g., one for each chapter, at least).
%
%    To help you identify the files and sections in this large file
%    I use the string '==========' to identify new files.
%
%    To help you ignore the unusual things I do with this sample document
%    I try to use the notation
%       
%    % --- sample stuff only -----
%    special stuff for my document, but you don't need it in your thesis
%    % --- end-of-sample-stuff ---


%    Printed in twoside style now that that's allowed
%
 
\documentclass [11pt, proquest] {uwthesis}[2015/03/03]

\usepackage{thesismacros}
\usepackage{algorithmic}
\usepackage{algorithm}


\renewcommand{\algorithmicrequire}{{\bf Input:}}
\renewcommand{\algorithmicensure}{{\bf Output:}}


%
% The following line would print the thesis in a postscript font 

% \usepackage{natbib}
% \def\bibpreamble{\protect\addcontentsline{toc}{chapter}{Bibliography}}

\setcounter{tocdepth}{1}  % Print the chapter and sections to the toc
 

% ==========   Local defs and mods
%

% --- sample stuff only -----
% These format the sample code in this document


\usepackage{alltt}  % 
\newenvironment{demo}
  {\begin{alltt}\leftskip3em
     \def\\{\ttfamily\char`\\}%
     \def\{{\ttfamily\char`\{}%
     \def\}{\ttfamily\char`\}}}
  {\end{alltt}}
 
% metafont font.  If logo not available, use the second form
%
% \font\mffont=logosl10 scaled\magstep1
\let\mffont=\sf
% --- end-of-sample-stuff ---
 



\begin{document}
 
% ==========   Preliminary pages
%
% ( revised 2012 for electronic submission )
%

\prelimpages
 
%
% ----- copyright and title pages
%
\Title{The Suitability of the \LaTeX\ Text Formatter\\
  for Thesis Preparation by Technical and\\
  Non-technical Degree Candidates}
\Author{Hao Chen}
\Year{1985-2014}
\Program{UW Information Technology}

\Chair{Name of Chairperson}{Title of Chair}{Department of Chair}
\Signature{First committee member}
\Signature{Next committee member}
\Signature{etc}

\copyrightpage

% \titlepage  

% --- sample stuff only -----
% unusual footnote not found in a real thesis
% You just use the \titlepage as commented out above

{\Degreetext{A dissertation%
  \footnote[2]{an egocentric imitation, actually}\\
  submitted in partial fulfillment of the\\ requirements for the degree of}
 \def\thefootnote{\fnsymbol{footnote}}
 \let\footnoterule\relax
 \titlepage
 }
\setcounter{footnote}{0}

% --- end-of-sample-stuff ---
 
%
% ----- signature and quoteslip are gone
%

%
% ----- abstract
%


\setcounter{page}{-1}
\abstract{%
This sample dissertation is an aid to students who are attempting
to format their theses with \LaTeX, a sophisticated
text formatter widely used by mathematicians and scientists everywhere.
 
\begin{itemize}
\item It describes the use of a specialized
macro package developed specifically for thesis production
at the University.
The macros customize \LaTeX\ for the correct thesis style,
allowing the student to concentrate on the substance of
his or her text.%
\footnote{See Appendix A to obtain the source to this
 thesis and the class file.}
\item It demonstrates the solutions to a variety of
formatting challenges found in thesis production.
\item It serves as a template for a real dissertation.
\end{itemize}
}
 
%
% ----- contents & etc.
%
\tableofcontents
\listoffigures
%\listoftables  % I have no tables
 
%
% ----- glossary 
%
\chapter*{Glossary}      % starred form omits the `chapter x'
\addcontentsline{toc}{chapter}{Glossary}
\thispagestyle{plain}
%
\begin{glossary}
\item[argument] replacement text which customizes a \LaTeX\ macro for
each particular usage.
\item[back-up] a copy of a file to be used when catastrophe strikes
the original.  People who make no back-ups deserve
no sympathy.
\item[control sequence] the normal form of a command to \LaTeX.
\item[delimiter] something, often a character, that indicates
the beginning and ending of an argument.
More generally, a delimiter is a field separator.
\item[document class] a file of macros that tailors \LaTeX\ for
a particular document.  The macros described by this thesis
constitute a document class.
\item[document option] a macro or file of macros
that further modifies \LaTeX\ for
a particular document.  The option {\tt[chapternotes]}
constitutes a document option.
\item[figure] illustrated material, including graphs,
diagrams, drawings and photographs.
\item[font] a character set (the alphabet plus digits
and special symbols) of a particular size and style.  A couple of fonts
used in this thesis are twelve point roman and {\sl twelve point roman
slanted}.
\item[footnote] a note placed at the bottom of a page, end of a chapter,
or end of a thesis that comments on or cites a reference
for a designated part of the text.
\item[formatter] (as opposed to a word-processor) arranges printed
material according to instructions embedded in the text.
A word-processor, on the other hand, is normally controlled
by keyboard strokes that move text about on a display.
\item[\LaTeX] simply the ultimate in computerized typesetting.
\item[macro]  a complex control sequence composed of 
other control sequences.
\item[pica] an archaic unit of length.  One pica is twelve points and
six picas is about an inch.
\item[point] a unit of length.  72.27 points equals one inch.
\item[roman]  a conventional printing typestyle using serifs.
the decorations on the ends of letter strokes.
This thesis is set in roman type.
\item[rule] a straight printed line; e.g., \hrulefill.
\item[serif] the decoration at the ends of letter strokes.
\item[table] information placed in a columnar arrangement.
\item[thesis] either a master's thesis or a doctoral dissertation.
This document also refers to itself as a thesis, although it
really is not one.
 
\end{glossary}
 
%
% ----- acknowledgments
%
\acknowledgments{% \vskip2pc
  % {\narrower\noindent
  The author wishes to express sincere appreciation to
  University of Washington, where he has had the opportunity
  to work with the \TeX\ formatting system,
  and to the author of \TeX, Donald Knuth, {\it il miglior fabbro}.
  % \par}
}

%
% ----- dedication
%
\dedication{\begin{center}to my dear wife, Joanna\end{center}}

%
% end of the preliminary pages
 
 
 
%
% ==========      Text pages
%

\textpages
 
% ========== Chapter 1
 
\chapter {Introduction}
 
The utility of a clean, professionally prepared thesis is well
documented%
\footnote{See, for example,
  W.~Shakespeare\cite{Hamlet} for a recent discussion.}
and, even if you never intend to actually print your thesis,
you still ought to format it as if that were your intention.
 
\TeX\ facilitates that. It is a flexible,
complete and professional typesetting system.
It will produce {\bf pdf} output as required by the Graduate School.

\section{The Purpose of This Sample Thesis}
 
This sample is both a demonstration of the quality and
propriety of a \LaTeX formatted thesis and  
documentation for its preparation.
It has made extensive use of a custom class file
developed specifically for this purpose
at the University of Washington.  Chapter~II discusses
\TeX\ and \LaTeX.
Chapter III describes the additional macros and functions
provided by the custom thesis class file.  Finally, Chapter IV hopes to tie things up.
 
It is 
impossible to predict all the formatting problems one will encounter
and there will be problems that are best handled
by a specialist.  
The Graduate School may be able to help you find help.
Some departments may also be able to provide \LaTeX\ assistance.
 
 
\section{Conventions and Notations}
 
In this thesis the typist
refers to the user of \LaTeX---the one who
makes formatting decisions and chooses the appropriate
formatting commands.
He or she will most often be the degree candidate.
 
This document deals with \LaTeX\ typesetting commands and their
functions.  Wherever possible the conventions used to display
text entered by the typist and the resulting formatted output
are the same as those used by the \TeX books.
Therefore, {\tt typewriter type} is used to indicate text
as typed by the computer
or entered by the typist.
It is quite the opposite of {\it italics,} which indicates
a category rather than exact text.  For example,
{\tt alpha} and {\tt beta} might each be an example of a {\it label}.
 
 
\section{Nota bene}
 
This sample thesis was produced by the \LaTeX\ document class it describes
and its format is consonant with the Graduate School's electronic dissertation guidelines,
as of November, 2014, at least.
However, use of this package does not guarantee acceptability
of a particular thesis.
 
 
% ========== Chapter 2

\chapter{Fourier expansions of cuspidal modular forms forms at cusps}

Let $k$ be a positive even integer and let $f \in S_k(\Gamma_0(N))$ be a nonzero cusp form.
We are concerned with the problem of computing the Fourier expansion of $f$ at cusps of width 1
other than the cusp denoted by $[\infty]$. Note that such cusps exist precisely when $N$ is 
not square-free. We will give two algorithms, one numerical and the other exact, to compute such
expansions. The question is studied in Delaunay's thesis. We draw insight from another preprint by 
F.Brunault. The question is also studied in [Edixhoven], where numerical algorithm is given. The algorithm 
in [Ed] for computing expansions requires working at a higher level: to compute expansions at cusps of denominator $Q$, one needs to compute period matrices for forms of level $NR^2$, where 
$R = \gcd(Q, \frac{N}{Q})$. As a contrast, our algorithm works at levels dividing $N$. 

\section{preliminary definitions}

\begin{Definition}(Denominator of a cusp)
 Let $z = \left[ \frac{a}{c} \right]$, where $\gcd(a,c) = 1$, be a cusp on $X_0(N)$. Then the {\it denominator} of $z$ is 
\[
	d(z)  = \gcd(c,N)
\] 
As a convention, we set $d([\infty]) = N$. 
\end{Definition}

!to-do: the lemma that states the expansion is well-defined. 

Remark: since the space $S_k(\Gamma_0(N))$ is spanned by elements of form $g(q^d)$ where $g$ is newform on level $M$ and d is an integer such that $Md \mid N$. We have the following lemma relating the expansions of $g(q^d)$ and $g(q)$. 

As a consequence of Lemma~\ref{...}, we can restrict ourselves to newforms when solving problem 1. 

!to-do: add the lemma. 

\begin{Lemma}
A cusp $z$ on $X_0(N)$ has width one if and only if $z = [\infty]$ or $z = [\frac{a}{c}]$ for some integers $a,c$ such that
$\gcd(a,c) = 1$ and $c^2 \equiv 0 \pmod{N}$. In other words, the cusp $z$ has width one if and only if $N \mid d(z)^2$.
\end{Lemma}

\begin{proof}
Let $z = [\frac{a}{c}]$ be a cusp on $X_0(N)$ other than $[\infty]$. Choose integers $b,d$ such that $\delta =  \abcd{a}{b}{c}{d} \in SL_2(\bZ)$. Then $\delta([\infty]) = z$ and by definition, the width of $z$ is one if and only if $\delta \abcd{1}{1}{0}{1} \delta^{-1} \in \Gamma_0(N)$. Since $\delta \abcd{1}{1}{0}{1} \delta^{-1} = \abcd{1-ac}{a^2}{-c^2}{1+ac}$, the claim follows. 
\end{proof}

\section{Reducing to newforms at prime power conductors}

\section{Twists of newforms and the formula}

Reference: [Winnie]. 


\section{Pseudo-eigenvalues} 

Let $\epsilon$ be a Dirichlet character of modulus $N$ and let $f$ be a newform in $S_k(N,\epsilon)$. For any divisor $Q$ of $N$ with $\gcd(Q, \frac{N}{Q}) =1$, we have 
\[
	W_Q(f) = \lambda_Q(f) g, 
\]
where $\lambda_Q(f)$ is an algebraic number in $\bC$ of absolute value 1 and $g$ is a newform in $S_k(N, \overline{\epsilon_Q} \epsilon_{N/Q})$. 

\begin{Definition}
The number $\lambda_Q(f)$ is called the pseudo-eigenvalue of $W_Q$ on $f$. 
\end{Definition}

Recall that if $f = \sum a_nq^n$ is a modular form, then $f^*$ is defined as $f^*(q) = \sum \bar{a_n}q^n$. From now on, we take $Q = N$. Then from [Winnie] we have $W_N(f) = \lambda_N(f) f^*$. In the rest of this section, we describe an algorithm to efficiently compute the numerical value of $\lambda(f) := \lambda_N(f)$. 



\begin{algorithm}[H]
\caption{Computing the pseudo-eigenvalue of newforms.}
\label{alg: pseudo-eigenvalue}
\begin{algorithmic}[1]
    \REQUIRE $k \geq 0 $ an {\it even} integer. $f \in S_k(\Gamma_1(N))$ a normalized newform.    
    \ENSURE a numerical approximation of the complex number $\lambda_N(f)$ defined by $f|W_N = \lambda_N(f) f^*$.   
    \STATE $n_0 := 10$, $z_0 := \frac{i}{\sqrt{N}}$. $\delta = 10^{-3}$. 
    \STATE Randomly generate $n_0$ points $\{z_1, \cdots, z_{N_0}\} \subseteq \{z | 0 < Im(z) < 1/2\sqrt{N}, |z| = \sqrt{1}{N} \}$
    \FOR{$1 \leq i \leq n_0$}
    	\STATE compute the period integral $c_i =  \int_{z_0}^{z_i} 2\pi i f(z) z^{\frac{k-2}{2}} dz$. 
	\STATE $w_i := c_i/\bar{c_i}$. 
    \ENDFOR
    \STATE $w := (w_1, \cdots, w_{n_0})$. 
    \IF{the standard deviation of $w$ is less than $\delta$} 
     \STATE return $\frac{1}{n_0}(\sum_i w_i)$. 
    \ELSE
    	\STATE go back to step 1 and increase $n_0$. 
    \ENDIF
\end{algorithmic}
\end{algorithm}

\begin{remark}
When $f$ has trivial character, we have $g = f$ and $\lambda_N(f) = \pm 1$ is indeed an eigenvalue of $W_N$. This eigenvalue can be determined using modular symbols, which is faster than 
Algorithm~\ref{alg: pseudo-eigenvalue}. For more details, see documentation of sage. 
\end{remark}

\begin{remark}
The period integral in step 4 of Algorithm~\ref{lag: pseudo-eigenvalue} is computed as follows: 
This approach is taken from [Cre97].
\end{remark}

To prove the correctness of algorithm 1, we will define a space of symbols that mimics weight-$k$ modular symbols and then show that they share certain properties. The proofs follow [Stein-modform, Chapter ]. \\ 

\begin{Definition}


\end{Definition}

Let any element $g = \abcd{a}{b}{c}{d}  \in GL_2(\bQ)$ on $M(k)$ by 
\[
	g(P\{\alpha, \beta\}) = P(dX-bY, -cX+aY)\{g\alpha,g\beta\}.
\]
We define a pairing between $S_k(\Gamma_1(N))$ and $M(k)$ by the following:
\[
	(f, P\{\alpha, \beta\}) \stackrel{def}{=} \int_{\alpha}^{\beta} 2\pi i f(z) P(z,1) dz.
\]

\begin{Lemma}
Let $k \geq 0$ be even. Let $M \in M(k)$ and $f \in S_k(\Gamma_1(N))$. Then 
\[
	N^{\frac{k}{2}-1}(f|W_N, M) = (f, W_N M).
\]
\end{Lemma}

\begin{proof}
See [Stein,]. Note that the extra factor $N^{\frac{k}{2}-1}$ is due to the different definitions the weight-$k$ action of $GL_2(\bQ)$ on modular forms: in [Stein] it is defined by 
\[
blah
\]
\end{proof}

The map  
\[
	P(x,y)\{\alpha, \beta\} \mapsto P(-x,y) \{-\bar{\alpha}, -\bar{\beta}\} 
\]
defines an involution on the space $M(k)$ which we call the {\it start involution} and denote by $*$. Then 
\begin{Lemma}
$$(f^*, M) = \overline{(f, M^*)}.$$
\end{Lemma}
\begin{proof}

\end{proof}



\begin{Lemma}
Let $f$ be a normalised newform with even weight $k \geq 0$ on $\Gamma_1(N)$ and let $M \in M(k)$ be such that $W_N(M) = N^{k/2 -1} M^*$. Assume $(f, M) \neq 0$.  Then the pseudo-eigenvalue $\lambda_N(f)$ of $f$ is 
\[
	\lambda_N(f) = \frac{(f,M)}{\overline{(f,M)}}.
\]
\end{Lemma} 

\begin{proof}
Since $W_N^2(M) = N^{k-2}M$ for all $M \in M(k)$, the assumption implies $W_N(M^*) = N^{k/2 -1} M$. 
Now 
\begin{align*}
& N^{k/2-1}(f|W_N, M^*) = (f,W_N(M^*)) \, (\mbox{Lemma~\ref{}})\\ 
\implies & N^{k/2-1}\lambda_N(f)(f^*,M^*) = N^{k/2-1}(f, M) \\
\implies & \lambda_N(f) = \frac{(f,M)}{(f^*,M^*)} \\ 
\implies &	\lambda_N(f) = \frac{(f,M)}{\overline{(f,M)}}.
\end{align*}
\end{proof}

Finally, suppose $\alpha, \beta \in \{z \in \bC | Im(z) > 0,  |z| = 1/\sqrt{N}\}$. Then it is easy to verify that $M = (xy)^{k/2-1}\{\alpha, \beta\}$ satisfies the condition of the above lemma, i.e., $W_N(M) = M^*$. The correctness of Algorithm~\ref{} now follows. 



\section{Identifying level of twists}

We start by recalling some facts on twists of newforms from [Winnie]. 
\begin{theorem}[Winnie, theorem 3.2]
Let $f \in S_k(N, \epsilon)$ be a newform. Let $\chi$ be a Dirichlet character of modulus $N$ s.t., the conductor of $\chi$ is equal to $Q = q^\beta$, where q is a 
prime divisor of $N$. Then there is a newform $h \in S_k(Q'M, \epsilon \chi^2)$, such that
$$f_\chi = h -  h|U_q|B_q. $$
\end{theorem}

By the above theorem, we can compute the $q$-expansion of $f_\chi | W_N$ once we identify the level and $q$-expansion of $h$. 

Notation: For convenience we write $f \otimes \chi$ for $h$.

We are mainly interested in the case where $f$ has trivial nebentypus character. For such forms 
we have the following lemma.
\begin{Lemma}[$h$ has Level dividing $N$]
Let $f \in S_k(\Gamma_0(N))$. Suppose $Q = cond(\chi)$ has $Q^2 \mid N$. Then $f \otimes \chi$ is new of level $N' \mid N$. 
\end{Lemma}

\begin{proof}
By [Winnie, theorem 3.1], we have $f_\chi \in S_k(N, \chi^2)$ (since $\alpha = 0$ and $q^{2\beta} \leq Q$). It now follows from theorem 3.2 that the level of $f \otimes \chi$ is a divisor of $N$.
\end{proof}

Now we proceed on how to algorithmically recognise the level of $f \otimes \chi$ from the coefficients of $f$. One potential obstacle is that we do not know all Fourier coefficients of $f \otimes \chi$. We only know that 
\[
	a_n(f \otimes \chi)  = a_n(f)\chi(n), \,  \mbox{ if }  \gcd(n, N) = 1.
\]
We will overcome this using a variant of Sturm's bound argument. First we recall the usual Sturm bound. 
We quote theorem from [Stein]. Or rather, let's prove a lemma first. 

\begin{Lemma}
Let $f \in S_k(N, \epsilon)$ be a normalized newform. Then $f|U_q|B_q \in S_k(Nq^2, \epsilon)$.
\end{Lemma}

\begin{proof}
We use a standard fact that for any integer $d \geq 1$, the map $f \mapsto f|B_d$ takes $S_k(N, \epsilon)$ to 
$S_k(Nd, \epsilon)$. To prove the lemma, we consider two separate cases. First, assume $q \nmid N$, then we have 
$T_q = U_q + q^{k-1} \epsilon(q) B_q$. By our assumption, we have $f|T_q = a_q(f) f$. Therefore, we have 
$f|U_q|B_q = f|(T_q - q^{k-1} \epsilon(q) B_q)|B_q = a_q(f)f|B_q - q^{k-1} \epsilon(q) f|B_q^2$. Hence $f|U_q|B_q \in S_k(Nq^2, \epsilon)$. \\
Now assume $q \mid N$, so $U_q = T_q$. Hence $f|U_q|B_q = a_q(f) f|B_q \in S_k(Nq, \epsilon) \subseteq  S_k(Nq^2, \epsilon)$. 
\end{proof}

The next proposition generalised the usual Sturm bound argument for modular forms. 

\begin{Prop}
Let $g_1$, $g_2$ be two normalised newforms of levels $N_1 \mid N_2$. Suppose $g_i \in S_k(N_2, \epsilon)$, where $\epsilon$ has conductor $Q = q^\beta$, $q$ is a prime, and  $Q^2 \mid N$. Let 
$B$ be the Sturm bound for the congruence subgroup $\Gamma_1(Nq^2)$. Suppose 
\[
	a_n(g_1) = a_n(g_2), \, 1 \leq n \leq B \mbox{ and } \gcd(n,q) = 1.
\]
Then $g_1 = g_2$. 
\end{Prop}

\begin{proof}
Following [Winnie], we define the operator $K_q$ on the space of modular forms by
\[
	g|K_q = g - g|U_q|B_q. 
\]
Condition is equivalent to the statement that $\delta  = (g_1 -g_2) |K_q$ has $a_n(\delta) = 0$ for all $1 \leq n \leq B$. 
From Lemma~\ref{...} we know $\delta \in S_k(Nq^2, \epsilon)$. Hence Sturm's theorem \ref{...},  implies $\delta = 0$, i.e., $g_1 - g_2 |K_q = 0$.  

Theorem 5.7.1 in [DS] then implies that $g_1 - g_2 \in S_k(N_2,\epsilon)^{old}$. If $N_1 < N_2$, then $g_1$ in the old subspace. So $g_2 \in S_k(N_2,\epsilon)^{old}$, a contradiction. Hence $N_1 = N_2$. Then $g_1 - g_2 \in S_k(N_2, \epsilon)^{new}$, and since the new subspace and the old subspace have trivial intersection,  we have $g_1 - g_2  = 0$. 
\end{proof}

Now we are ready to describe the algorithm that given $f$ and $\chi$, determines the level of $f \otimes \chi$.

\begin{algorithm}[H]
\caption{Identifying level of $f \otimes \chi$}
\begin{algorithmic}[1]
    \REQUIRE $f \in S_k(\Gamma_0(N))$ a newform; $\chi$ a Dirichlet character of prime power conductor $Q = q^\beta$; $Q^2 \mid N$. 
    \ENSURE The positive integer $M$ such that the $f \otimes \chi$ is new of level $M$. 
    \IF{$Q = 1$}
    \STATE return $N$.
    \ENDIF
    \STATE $Q' := \cond(\chi^2)$; $N_0 := \frac{N}{q^{v_q(N)}}$; $M_0 := Q'N_0$; $t := \frac{N}{M_0} \in \bZ$. 
    \FOR{each positive divisor $d$ of $t$}
    	\STATE Set $V_d := S_k(M_0d, \chi^2)$. 
	\STATE Compute a basis of newforms $\{g_1^{(d)}, \cdots g_{s_d}^{(d)}\}$ of $V_d$.
    	\STATE Set $B_d$ := the Sturm bound for $\Gamma_1(M_0dq^2)$. 
    	\FOR{$1 \leq j \leq s_d$} 
		\IF{$a_n(g_i^{(d)})= a_n(f)\chi(n)$ for all $1 \leq n \leq B_d, \gcd(n,q) = 1$}
			\STATE return $M_0d$.
		\ENDIF
	\ENDFOR
    \ENDFOR	
\end{algorithmic}
\end{algorithm}

The correctness of the algorithm is ensured by Lemma~ and Lemma~. 

It is natural to define {\it $p$-minimality} of newforms. The definition mimics that of [Brunault]. 

\begin{Definition}
Let $f \in S_k(\Gamma_1(N))$ be a newform. Let $p$ be a prime such that $p^2 \mid N$. 
We say $f$ is {\it $p$-minimal} if $f \otimes \chi$ is new of level $N$ for all Dirichlet character 
$\chi: (\bZ/p^{\frac{v_p(N)}{2}}\bZ)^{\times} \to \bC^{\times}$.
\end{Definition}

We give some example computations applying the above algorithm. 

\begin{Example}
Let $f$ be the normalised newform attached to the elliptic curve 
\[
	E: 
\]
of label {\bf 50a}. Then $f \otimes \chi$ is new of level 50 for all Dirichlet characters $\chi$ with modulus 5. 
In other words, $f$ is 5-minimal. 
\end{Example}

As another example, we demonstrate a newform which is not $p$-minimal. 
\begin{Example}
Let $f$ be the normalised newform attached to the elliptic curve 
\[
E: y^2 + x y = x^{3} + x^{2} - 25 x - 111
\]
of label {\bf 98a}. Let $\chi$ be the Dirichlet character modulo 7 defined by $\chi(3 \pmod{7}) = -1$. 
Then $f \otimes \chi$ is a newform of level 14. 

\end{Example}

\section{Fields of definitions}

\begin{Lemma}
Let $c$ be a cusp of denominator $d$ and let $d' = N/d$. Then 
\[
	\bQ(\{a_n(f, c)\}) \subseteq \bQ( \{a_n(f)\}, \zeta_{d'}). 
\]
\end{Lemma}

\begin{proof}
Let $K_0 = \bQ( \{a_n(f)\}$
Choose a form $0 \neq g \in S_k(\Gamma_1(N))$ with rational Fourier coefficients such that 
$h = \frac{f}{g}$ is non-constant. From [Cox, ] it is easy to see that $h \in $. Then we have 
\end{proof}



\section{Matrix solution? this is way harder? }



\section{Examples}

Let $E = {\bf 50a}$ and consider the 4 cusps of denominator 10 on $X_0(50)$. The corresponding first terms 
of $q$-expansions at these cusps are 

\iffalse
\begin{align*}
	a_1(f, \frac{1}{10}) &= \frac{1}{5} \zeta_{5}^{3} - \frac{3}{5} \zeta_{5}^{2} + \frac{3}{5} \zeta_{5} - \frac{1}{5} \\ 
	a_1(f, \frac{3}{10}) &= \frac{3}{5} \zeta_{5}^{3} + \frac{6}{5} \zeta_{5}^{2} + \frac{4}{5} \zeta_{5} + \frac{2}{5} \\
	a_1(f, \frac{7}{10}) &= \frac{2}{5} \zeta_{5}^{3} - \frac{1}{5} \zeta_{5}^{2} - \frac{4}{5} \zeta_{5} - \frac{2}{5}\\
	a_1(f, \frac{9}{10}) &=-\frac{6}{5} \zeta_{5}^{3} - \frac{2}{5} \zeta_{5}^{2} - \frac{3}{5} \zeta_{5} - \frac{4}{5}
\end{align*}
\fi


\section{Applications}

One applications of the computation done in this chapter is the norm method to the computation of $j$-polynomials 
introduced in Chapter~. Recall that the issue with the norm method for non-square free level is computing the 
expansions of form $f | \gamma$, where $\gamma$ runs over the set of right coset representatives of $\Gamma_0(N)$ 
in $SL_2(\bZ)$. To compute the norm of  $f$ when $N$ is non-square free, it suffices to compute the expansions of 
$f$ at all width-1 cusps. This is a consequence of the following lemma.

\begin{Lemma}
For any cusp $z$ of $X_0(N)$, there exists an Atkin-Lehner involution $w \in W(N)$ such that $z_1 = w(z)$ has 
width one.
\end{Lemma}

\begin{proof}
Let $z \neq [\infty]$ be a cusp. Recall that $z$ has width one if and only if its denominator $d(z)$ satisfies
$d(z)^2 \equiv 0 \pmod{N}$. Let $p$ be a prime divisor of $N$. Then it is easy to see that 
$v_p(d(w_p(z))) = v_p(N) -  v_p(d(z))$ and $v_l(d(w_p(z))) = v_l(d(z))$ for primes $l \neq p$. The lemma now follows by taking $w = \prod_{p \mid N: v_p(d(z)) \leq v_p(N)/2} w_p$. 
\end{proof}

\section{Norm guess and data}
 
 
 
 
% ========== Chapter 3
 
\chapter{The Thesis Unformatted}
 
This chapter describes the uwthesis class (\texttt{uwthesis.cls},
version dated 2014/11/13)
in detail 
and shows how it was used to format the thesis.
A working knowledge of Lamport's \LaTeX\ manual\cite{Lbook} is assumed.
 
\section{The Control File}
 
The source to this sample thesis is a single file
only because ease of distribution was a concern.
You should not do this.  Your task will be much easier if you
break your thesis into several files:  a file for the preliminary pages,
a file for each chapter,  one for the glossary, and one for each
appendix.  Then use a control file to tie them all together.
This way you can edit and format parts of your thesis much more
efficiently.
 
Figure~\ref{control-file} shows a control file that
might have produced this thesis.
It sets the document style, with options and parameters,
and formats the various parts of the thesis---%
but contains no text of its own.
 
 
%  control file caption and figure
%
%
\begin{figure}[p]
 \begin{fullpage}
  \uwsinglespace
  \begin{verbatim}
    % LaTeX thesis control file
 
    \documentclass [11pt, proquest]{uwthesis}[2014/11/13]
 
    \begin{document}
 
    % preliminary pages
    %
    \prelimpages
    \include{prelim}
 
    % text pages
    %
    \textpages
    \include{chap1}
    \include{chap2}
    \include{chap3}
    \include{chap4}
 
    % bibliography
    %
    \bibliographystyle{plain}
    \bibliography{thesis}
 
    % appendices
    %
    \appendix
    \include{appxa}
    \include{appxb}
 
    \include{vita} 
    \end{document}
  \end{verbatim}
  \caption[A thesis control file]%
   {\narrower A thesis control file ({\tt thesis.tex}).
   This file is the input to \LaTeX\ that will produce a
   thesis.  It contains no text, only commands which
   direct the formatting of the thesis.
   }
  \label{control-file}
 \end{fullpage}
\end{figure}
 
The first section, from the \verb"\documentclass" to
the \verb"\begin{document}", defines the document class and options.
This sample thesis specifies the \texttt{proquest} style, which is now
required by the Graduate School and is the default.  
Two other, now dated, other styles are available:  \verb"twoside", which is similar but 
produces a wider binding margin and is more suitable for paper printing; and
\verb"oneside", which is really old fashoned.
This sample also specified a font size
of 11 points. 
Possible font size options are: \verb"10pt", \verb"11pt", and \verb"12pt".
Default is 12 points, which is the preference
of the Graduate School. If you choose a smaller size be sure to
check with the Graduate School for acceptability.  The smaller fonts
can produce very small sub and superscripts.

Include most additional formatting packages with \verb"\usepackage",
as describe by Lamport\cite{Lbook}.  The one exception to this
rule is the \verb"natbib" package.  Include it with the \verb"natbib"
document option.
 
Use the \verb"\includeonly" command to format only a part of your
thesis.  See Lamport\cite[sec. 4.4]{Lbook} for usage and limitations.

 
\section{The Text Pages}
 
A chapter is a major division of the thesis.  Each chapter begins
on a new page and has a Table of Contents entry.
 
\subsection{Chapters, Sections, Subsections, and Appendices}
 
 
Within the chapter title use a \verb"\\" control sequence to separate lines
in the printed title (recall Figure \ref{start-2}.).
The \verb"\\" does not affect the Table of Contents entry.
 
Format appendices just like chapters.
The control sequence \verb"\appendix" instructs \LaTeX\ to
begin using the term `Appendix' rather than `Chapter'.
 
 
Specify sections and subsections of a chapter 
with  \verb"\section" and \verb"\subsection", respectively.
In this thesis chapter and section
titles are written to the table of contents.
Consult Lamport\cite[pg. 176]{Lbook} to see which
subdivisions of the thesis can be written to the table of contents.
The \verb"\\" control sequence is not permitted in section and
subsection titles.
 
 
\subsection{Footnotes}
 
\label{footnotes}
 Footnotes format as described in the \LaTeX\ book.  You can also
 ask for end-of-chapter or end-of-thesis notes.
 The thesis class will automatically set these up if
 you ask for the document class option \texttt{chapternotes}
 or \texttt{endnotes}.  
 
If selected, chapternotes will print automatically.  If you choose
endnotes however you must explicitly indicate when to print the notes 
with the command \verb"\printendnotes".  See the style guide for
suitable endnote placement.  

\subsection{Figures and Tables}
Standard \LaTeX\ figures and tables, see Lamport\cite[sec.~C.9]{Lbook},
normally provide the most convenient means to position the figure.
Full page floats and facing captions are exceptions to this rule.

If you want a figure or table to occupy a full page enclose the
contents in a \texttt{fullpage} environment.  
See figure~\ref{facing-caption}.

\subsubsection{Facing pages}
Facing page captions are an artifact of traditional, dead-tree printing,
where a left-side (even) page faces a right-side (odd) page.

In the \texttt{twoside} style, a facing caption
is full page caption for a full page figure or table
and should face the illustration to which it refers.
You must explicitly format both pages. 
The caption part appears on an even page
(left side) and the figure or table
comes on the following odd page (right side).
Enclose the float contents for the caption 
in a \texttt{leftfullpage} environment,
and enclose the float contents for the figure or table 
in a \texttt{fullpage} environment.
The first page (left side) contains the caption. The second page
(right side) could be left blank.  A picture or graph might be pasted onto
this space. See figure~\ref{facing-caption}.


\begin{figure}[t]
\uwsinglespace
\begin{verbatim}
     \begin{figure}[p]% the left side caption
       \begin{leftfullpage}
         \caption{ . . . }
       \end{leftfullpage}
     \end{figure}
     \begin{figure}[p]% the right side space
       \begin{fullpage}
          . . .
          ( note.. no caption here )
       \end{fullpage}
     \end{figure}
\end{verbatim}
\caption[Generating a facing caption page]{This text would create a
  double page figure in the two-side styles. }
\label{facing-caption}
\end{figure}
 
You can use these commands with the \texttt{proquest} style, but they have little
effect on online viewing.
 
 
\subsection{Horizontal Figures and Tables}
Figures and tables may be formatted horizontally
(a.k.a.\ landscape) as long as their captions appear
horizontal also.  \LaTeX\ will format landscape material for you.

Include the \texttt{rotating} package 
\begin{demo}
\\usepackage[figuresright]\{rotating\}
\end{demo}
and read the documentation that comes with the package. 

Figure~\ref{sideways} is an example of how a landscape
table might be formatted. 

\begin{figure}[t]
\uwsinglespace
\begin{verbatim}
     \begin{sidewaystable}
         ...
         \caption{ . . . }
     \end{sidewaystable}
\end{verbatim}
\caption[Generating a landscape table]{This text would create a
  landscape table with caption.}
\label{sideways}
\end{figure}
 


\subsection{Figure and Table Captions}
Most captions are formatted with the \verb"\caption" macro as described 
by Lamport\cite[sec. C.9]{Lbook}. 
The uwthesis class extends this macro to allow
continued figures and tables, and to provide multiple figures and
tables with the same number, e.g., 3.1a, 3.1b, etc.
 
To format the caption for the first part of
a figure or table that cannot fit
onto a single page use the standard form:
\begin{demo}
\\caption[\textit{toc}]\{\textit{text}\}
\end{demo}
To format the caption for the subsequent parts of 
the figure or table 
use this caption:
\begin{demo}
\\caption(-)\{(continued)\}
\end{demo}
It will keep the same number and the text of the caption will be 
{\em(continued)}.

To format the caption for the first part of
a multi-part figure or table
use the format:
\begin{demo}
\\caption(a)[\textit{toc}]\{\textit{text}\}
\end{demo}
The figure or table will be lettered (with `a') as well as numbered.
To format the caption for the subsequent parts of 
the multi-part figure or table
use the format:
\begin{demo}
\\caption(\textit{x})\{\textit{text}\}
\end{demo}
where {\em x} is {\tt b}, {\tt c}, \ldots.
The parts will be lettered (with `b', `c', \ldots).

\subsection{Line spacing}

Normally line spacing will come out like it should. However, the 
ProQuest style allows single spacing in certain situations:
figure content, some lists, and etc.
Use \verb"\uwsinglespace" to switch to single spacing within
a \verb"\begin{}" and \verb"\end{}" block.
The code examples in this document does this. 

\section{The Preliminary Pages}
 
These are easy to format only because they are relatively invariant
among theses.  Therefore the difficulties have already been encountered
and overcome by \LaTeX\ and the thesis document classes.

Start with the definitions that describe your thesis.
This sample thesis was printed with the parameters:

\begin{demo}
\\Title\{The Suitability of the \\LaTeX\\ Text Formatter\\\\
   for Thesis Preparation by Technical and\\\\
   Non-technical Degree Candidates\}
\\Author\{Jim Fox\}
\\Program\{UW Information Technology\}
\\Year\{2012\}

\\Chair\{Name of Chairperson\}\{title\}\{Chair's department\}
\\Signature\{First committee member\}
\\Signature\{Next committee member\}
\\Signature\{etc\}

\end{demo}
 
Use two or more \verb"\Chair" lines if you have co-chairs.
 
\subsection{Copyright page}
Print the copyright page with \verb"\copyrightpage".

\subsection{Title page}
Print the title page with \verb"\titlepage".
The title page of this thesis was printed with%
\footnote{Actually, it wasn't.  I added a footnote---something you would not do.}
 
\begin{demo}
\\titlepage
\end{demo}
 
You may change default text on the title page with these
macros.  You will have to redefine \verb"\Degreetext", for instance,
if you're writing a Master's thesis instead of a dissertation.\footnote{If you use these they can
be included with the other information before \\copyrightpage".}

\begin{list}{}{\itemindent\parindent\itemsep0pt
   \def\makelabel#1{\texttt{\char`\\#1}\hfill}\uwsinglespace}
\item[Degree\char`\{{\it degree name}\char`\}]
   defaults to ``Doctor of Philosophy''
\item[School\char`\{{\it school name}\char`\}] defaults to
``University of Washington''
\item[Degreetext\char`\{{\it degree text}\char`\}] defaults to
``A dissertation submitted \ldots''
\item[textofCommittee\char`\{{\it committee label}\char`\}] defaults to
``Reading Committee:''
\item[textofChair\char`\{{\it chair label}\char`\}] defaults to
``Chair of the Supervisory Committee:''
\end{list}

These definitions must appear \underline{before} the \verb"\titlepage" command.

 
\subsection{Abstract}
Print the
abstract with \verb"\abstract".
It has one argument, which is the text of the abstract.
All the names have already been defined.
The abstract of this thesis was printed with
 
\begin{demo}
\\abstract\{This sample . . . `real' dissertation.\}
\end{demo}
 
 
\subsection{Tables of contents}
Use the standard \LaTeX\ commands to format these items.
 
 
\subsection{Acknowledgments}
Use the \verb"\acknowledgments" macro to format the acknowledgments page.
It has one argument, which is the text of the acknowledgment.
The acknowledgments of this thesis was printed with
 
\begin{demo}
\\acknowledgments\{The author wishes . . . \{\\it il miglior fabbro\}.\\par\}\}
\end{demo}
 
 
\subsection{Dedication}
Use the \verb"\dedication" macro to format the dedication page.
It has one argument, which is the text of the dedication.
 
\subsection{Vita}
Use the \verb"\vita" macro to format the curriculum vitae.
It has one argument, which chronicles your life's accomplishments.

Note that the Vita is not really a preliminary page.
It appears at the end of your thesis, just after the appendices.
 
 
%%  
%% \section{Customization of the Macros}
%%  
%% Simple customization, including 
%% alteration of default parameters,  changes to dimensions,
%% paragraph indentation, and margins, are not too difficult.
%% You have the choice of modifying the class file ({\tt uwthesis.cls})
%% or loading
%% one or more personal style files to customize your thesis.
%% The latter is usually most convenient, since you do not need
%% to edit the large and complicated class file.
%% 
 


% ========== Chapter 4
 
\chapter{Running \LaTeX\\
  ({\it and printing if you must})}
 
 
From a given source \TeX\ will produce exactly the same document
on all computers and, if needed, on all printers.  {\it Exactly the same}
means that the various spacings, line and page breaks, and
even hyphenations will occur at the same places.

How you edit your text files and run \LaTeX\ varies
from system to system and depends on your personal preference.

\section{Running}

The author is woefully out of his depth where 
\TeX\ on Windows is concerned.  Google would be his resource.
On a linux system he types

\begin{demo}
\$\ pdflatex uwthesis
\end{demo}

and it generally works.

 
\section{Printing}
 
All implementations of \TeX\ provide the option of {\bf pdf} output,
which is all the Graduate School requires.  Even if you intend to
print a copy of your thesis create a 
{\tt pdf}.  It will print most anywhere.

\printendnotes

%
% ==========   Bibliography
%
\nocite{*}   % include everything in the uwthesis.bib file
\bibliographystyle{plain}
\bibliography{uwthesis}
%
% ==========   Appendices
%
\appendix
\raggedbottom\sloppy
 
% ========== Appendix A
 
\chapter{Where to find the files}
 
The uwthesis class file, {\tt uwthesis.cls}, contains the parameter settings,
macro definitions, and other \TeX nical commands which
allow \LaTeX\ to format a thesis.  
The source to
the document you are reading, {\tt uwthesis.tex},
contains many formatting examples
which you may find useful.
The bibliography database, {\tt uwthesis.bib}, contains instructions
to BibTeX to create and format the bibliography.
You can find the latest of these files on:

\begin{itemize}
\item My page.
\begin{description}
\item[] \verb%http://staff.washington.edu/fox/tex/uwthesis.html%
\end{description}

\item CTAN
\begin{description}
\item[]  \verb%http://tug.ctan.org/tex-archive/macros/latex/contrib/uwthesis/%
\item[]  (not always as up-to-date as my site)
\end{description}

\end{itemize}

\vita{Jim Fox is a Software Engineer with UW Information Technology at the University of Washington.
His duties do not include maintaining this package.  That is rather
an avocation which he enjoys as time and circumstance allow.

He welcomes your comments to {\tt fox@uw.edu}.
}


\end{document}
