%  ========================================================================
%  Copyright (c) 1985-2014 The University of Washington
%
%  Licensed under the Apache License, Version 2.0 (the "License");
%  you may not use this file except in compliance with the License.
%  You may obtain a copy of the License at
%
%      http://www.apache.org/licenses/LICENSE-2.0
%
%  Unless required by applicable law or agreed to in writing, software
%  distributed under the License is distributed on an "AS IS" BASIS,
%  WITHOUT WARRANTIES OR CONDITIONS OF ANY KIND, either express or implied.
%  See the License for the specific language governing permissions and
%  limitations under the License.
%  ========================================================================
%

% Documentation for University of Washington thesis LaTeX document class
% by Jim Fox
% fox@washington.edu
%
%    Revised for version 2015/03/03 of uwthesis.cls
%
%    This document is contained in a single file ONLY because
%    I wanted to be able to distribute it easily.  A real thesis ought
%    to be contained on many files (e.g., one for each chapter, at least).
%
%    To help you identify the files and sections in this large file
%    I use the string '==========' to identify new files.
%
%    To help you ignore the unusual things I do with this sample document
%    I try to use the notation
%       
%    % --- sample stuff only -----
%    special stuff for my document, but you don't need it in your thesis
%    % --- end-of-sample-stuff ---


%    Printed in twoside style now that that's allowed
%
 
\documentclass [11pt, proquest] {uwthesis}[2015/03/03]

\usepackage{thesismacros}
\usepackage{algpseudocode}
\usepackage{algorithmicx}
\usepackage{algorithm}
\usepackage{url}
\renewcommand{\algorithmicrequire}{\textbf{Input:}}
\renewcommand{\algorithmicensure}{\textbf{Output:}}
\usepackage{amsmath,mathtools}
\usepackage{commath} % for the \dif operator.
\newcommand{\cE}{\mathcal{E}}
\newcommand{\N}{\mathfrak{N}}
\renewcommand{\H}{\mathcal{H}}
\newcommand{\fl}{\mathfrak{l}}

\usepackage[textwidth=50,textsize=tiny]{todonotes}
\setlength{\marginparwidth}{2cm}
\newcommand{\tinytodo}[2][]
{\todo[caption={#2}, #1]{\renewcommand{\baselinestretch}{0.5}\selectfont#2\par}}
\newcommand{\Katetodo}[1]{\tinytodo[color=green!20]{#1}}
\newcommand{\Haotodo}[1]{\tinytodo[color=red!20]{#1}}
\newcommand{\Kristintodo}[1]{\tinytodo[color=blue!20]{#1}}



%
% The following line would print the thesis in a postscript font 

% \usepackage{natbib}
% \def\bibpreamble{\protect\addcontentsline{toc}{chapter}{Bibliography}}

\setcounter{tocdepth}{1}  % Print the chapter and sections to the toc
 

% ==========   Local defs and mods
%

% --- sample stuff only -----
% These format the sample code in this document


\usepackage{alltt}  % 
\newenvironment{demo}
  {\begin{alltt}\leftskip3em
     \def\\{\ttfamily\char`\\}%
     \def\{{\ttfamily\char`\{}%
     \def\}{\ttfamily\char`\}}}
  {\end{alltt}}
 
% metafont font.  If logo not available, use the second form
%
% \font\mffont=logosl10 scaled\magstep1
\let\mffont=\sf
% --- end-of-sample-stuff ---
 



\begin{document}
 
% ==========   Preliminary pages
%
% ( revised 2012 for electronic submission )
%

\prelimpages
 
%
% ----- copyright and title pages
%
\Title{Computational aspects of modular parametrizations \\ of elliptic curves}
\Author{Hao Chen}
\Year{2011-2016}
\Program{UW Mathematics}

\Chair{William Arthur Stein}{Professor}{Department of Mathematics}
\Signature{First committee member}
\Signature{Next committee member}
\Signature{etc}

\copyrightpage

% \titlepage  

% --- sample stuff only -----
% unusual footnote not found in a real thesis
% You just use the \titlepage as commented out above

{\Degreetext{A dissertation 
  submitted in partial fulfillment of the\\ requirements for the degree of}
 \def\thefootnote{\fnsymbol{footnote}}
 \let\footnoterule\relax
 \titlepage
 }
\setcounter{footnote}{0}

% --- end-of-sample-stuff ---
 
%
% ----- signature and quoteslip are gone
%

%
% ----- abstract
%


\setcounter{page}{-1}
\abstract{
We investigate computational problems related to modular parametrizations of 
elliptic curves defined over $\mathbb{Q}$. Chapter 1 covers the background materials. In Chapter 2,
we develop algorithms to compute the Mazur Swinnerton-Dyer critical subgroup of elliptic curves, and verify that for all elliptic curves of rank two and conductor $<$ 1000, the critical subgroup is torsion. 
In Chapter 3, we develop algorithms to compute Fourier expansions of $\Gamma_0(N)$ newforms at cusps other than the cusp at infinity. In Chapter 4 we study properties of Chow-Heegner points. We proved the index of Chow-Heegner points are always divisible by two when the conductor $N$ has many prime divisors, and we develop algebraic algorithms to compute them. 
}
 
%
% ----- contents & etc.
%
\tableofcontents
%\listoffigures
%\listoftables  % I have no tables
 
%
% ----- glossary 
%
\chapter*{Glossary}      % starred form omits the `chapter x'
\addcontentsline{toc}{chapter}{Glossary}
\thispagestyle{plain}
%
\begin{glossary}
\item[An item] a description of item.
\end{glossary}
 
%
% ----- acknowledgments
%
%\acknowledgments{% \vskip2pc
  % {\narrower\noindent
%  The author wishes to express sincere appreciation to
%  University of Washington, where he has had the opportunity
%  to work with the \TeX\ formatting system,
 % and to the author of \TeX, Donald Knuth, {\it il miglior fabbro}.
  % \par}
%}

%
% ----- dedication
%
\dedication{\begin{center} to my better self\end{center}}

%
% end of the preliminary pages
 
 
 
%
% ==========      Text pages
%

\textpages

\chapter{Introduction}

\section{Elliptic Curves}
A {\em nonsingular projective curve} $X/\bQ$ is an algebraic variety $X \subset \bP^n$ such that $\dim(X) = 1$, $X$ is smooth,  and it has defining equations with coefficients in $\bQ$. We'll abbreviate it with `curve'.

For every curve $X$ there is a nonnegative integer $g(X)$ called its {\em genus}, and a field $k(X)$ of transcendence degree 1 over $\bQ$, called its {\em function field}. We have the following
\begin{Fact}
\label{fact: degree}
If $x$ is a non constant rational function,  then $[k(X): k(x)] = \deg x$.
\end{Fact}

An {\em elliptic curve }over $\bQ$ is the projective closure in $\bP^2$ of an affine curve given by the Weierstrass equation $y^2 = x^3 + Ax+B$ with $A,B \in \bQ$ and $4A^3+27B^2 \neq 0$. Elliptic curves are examples of curves of genus 1.

\section{Modular curves}


Modular curve: The {\em modular group} $SL_2(\bZ)$ is the group of integer matrices with determinant 1. For each integer $N > 1$, consider the subgroup 
\[
	\Gamma_0(N) = \left \{ \abcd{a}{b}{c}{d} \in SL_2(\bZ):  N \mid c \right\}
\]
The {\em modular curve} $X_0(N)$ is defined by $\Gamma_0(N) \backslash (H \cup \bP^1(\bQ))$. The 
equivalence classes of $\bP^1(\bQ)$ under $\Gamma_0(N)$ are called {\em cusps}. The set of cusps is finite.

Fact: $X_0(N)$ is a nonsingular projective algebraic curve.

\section{Modular forms}

Notation: Let $\H$ denote the complex upper half plane.
For each integer $k$, the weight-k action of $SL_2(\bZ)$ on functions $f: H \to \bC$ as follows: 
suppose $g  = \abcd{a}{b}{c}{d} \in SL_2(\bZ)$, define
\[
	f|[g]_k(z) = (cz+d)^{-k}f(gz).
\]
In this report we'll only consider the cases $k = 0$ or $k = 2$. We will omit the $k$ in the subscript, since we always 
mean $k = 2$ when $f$ is a modular form of weight 2, and $k = 0$ when $f$ is a modular function.
\begin{Definition}
A {\em modular form} of weight 2 and level $N$ is a holomorphic function $f: \H \to \bC$ satisfying  \\
(1) $f|[g]_2 = f$ for all $g \in \Gamma_0(N)$. \\
(2) $f$ can be holomorphically extended to $\H \cup \bP^1(\bQ)$. 
\end{Definition}
A modular form $f$ is a {\em cusp form} if  $f$ is zero at all cusps. For the precise definition of modular form, see  \cite[I.2]{diamond2005first}.

The vector space $S_2(\Gamma_0(N))$ of cusp forms of weight 2 and level $N$ is finite dimensional, and we have an isomorphism 
\[
	S_2(\Gamma_0(N)) \cong \Omega^1(X_0(N)(\bC)), 
\]
which sends $f$ to $f(z)dz$. As a consequence, $\dim_\bC  S_2(\Gamma_0(N)) = g(X_0(N))$.

We are going to be interested in a subset of cusp forms on $\Gamma_0(N)$ called  {\em newforms}(see \cite[V.8]{diamond2005first} for the precise definition of newforms).
In particular, all cusp forms attached to elliptic curves defined over $\bQ$ are newforms. We summarise some properties of newforms below:

(a) Newforms have q-expansions $f(q) = \sum_{n \geq 1} a_nq^n$ where $q  = e^{2\pi i z}$, $a_1= 1$, and $a_n \in \bZ$
for all $n > 1$. There exists algorithms to compute the coefficients $a_n$. In fact, all modular functions and modular forms 
used in this report, unless otherwise noted, have rational q-expansions, i.e., $\sum_{n \geq -m}b_nq^n$ with $b_i \in \bQ$

(b) For each $N$ we have an elementary abelian 2-group $W \subseteq Aut_\bQ(X_0(N))$, which we 
call the {\em Atkin-Lehner Group}, with generators $\{w_p\}_{p \mid N}$. The non-trivial elements of $W$ are called {\em Atkin-Lehner} involutions. They act on $S_2(\Gamma_0(N))$ by invertible linear transformations. Any newform $f$ is an eigenvector of every $w \in W$ with eigenvalue $\pm1$, i.e., $f|w = \pm f$. In particular, we have $f|{w_N} = (-1)^{r_{an}(E)}f$ where $w_N = \prod_{p \mid N} w_p$ is also called the Fricke involution.

We state the {\em modularity theorem}:

\begin{theorem} (Wiles, Taylor et al.)
For every elliptic curve $E/\bQ$ with conductor $N$ there exists a surjective morphism 
\[
	\varphi: X_0(N) \to E
\]
defined over $\bQ$. Moreover,  $\varphi^*(\omega_E) = c \cdot 2\pi i f_E(z) dz$  where $f_E$ is a newform of weight 2 and level $N$, called \textbf{the modular form attached to $E$}, and $c \in \bC^{\times}$.
\end{theorem}

Let $\omega_f$ denote the differential $f(z)dz \in S_2(\Gamma_0(N))$. Every elliptic curve $E$ has a {\em period lattice} $\Lambda$, which is a lattice in $\bC$. There is an isomorphism $\iota: E \cong \bC/\Lambda$. The composition $\iota \circ \varphi: X_0(N) \to \bC/\Lambda$ can be written as 
\[
	[z] \mapsto \int_{z}^{\infty} \omega_f \pmod{\Lambda}. 
\]


\red{fixme: define newforms. Maybe use Cornell-Silverman.}

\chapter{Computing the Mazur Swinnerton-Dyer critical subgroup of elliptic curves}


\subsection{Preliminaries}
Let $E$ be an elliptic curve over $\bQ$ and let $L(E,s)$ be the  $L$-function of $E$. The rank part of the Birch and Swinnerton-Dyer (BSD) conjecture states that
\[
	\rank(E(\bQ)) = \ord_{s=1}L(E,s).
\]
The right hand side is called the {\em analytic rank of $E$}, and is denoted by $r_{\an}(E)$. The left hand side is called the {\em algebraic rank of $E$}. The rank part of the BSD conjecture is still open when $r_{\an}(E) > 1$, and its proof for the case $r_{\an}(E) = 1$ uses the {\em Gross-Zagier formula}, which relates the value of certain $L$-functions to heights of Heegner points.

Let $N$ denote the conductor of $E$. The modular curve $X_0(N)$
is a nonsingular  projective curve defined over $\bQ$. Since $E$ is modular(Breuil, Conrad, Diamond, and Taylor \cite{breuil2001modularity}), there is a surjective morphism $\varphi:X_0(N) \to E$ defined over $\bQ$. Let $\omega_E$ be the invariant differential on $E$ and let $\omega =  \varphi^*(\omega_E)$. Then $\omega$ is a holomorphic differential on $X_0(N)$ and we have $\omega  = c f(z) dz$, where $f$ is the normalized newform attached to $E$ and
$c$ is a nonzero constant. In the rest of the paper, we fix the following notations: the elliptic curve $E$, the conductor $N$, the morphism $\varphi$, and the differential $\omega$. Let $R_\varphi$ be the ramification divisor of $\varphi$.



\begin{Definition}[Mazur and Swinnerton-Dyer \cite{mazur-sd}]
The {\em critical subgroup} of $E$ is
\[
	E_{\crit}(\bQ)  = \langle \tr(\varphi([z])) : [z] \in \supp R_\varphi \rangle,
\]
where $\tr(P) = \sum_{\sigma: \bQ(P) \to \bar{\bQ}} P^{\sigma}$.
\end{Definition}

Since the divisor $R_\varphi$ is defined over $\bQ$, every point $[z]$ in its
support is in $X_0(N)(\overline{\bQ})$, hence $\varphi([z]) \in E(\overline{\bQ})$, justifying the trace operation. The group $E_{\crit}(\bQ)$ is a subgroup of $E(\bQ)$. Observe that $R_\varphi = \Div(\omega)$, thus $\deg R_{\varphi} = 2g(X_0(N))-2$. In the rest of the paper, we use the notation $\Div(\omega)$ in place of the ramification divisor $R_\varphi$. In addition, we will assume $E$ is an optimal elliptic curve, so $\varphi$ is unique up to sign. This justifies the absence of $\varphi$ in the notation $E_{\crit}(\bQ)$.

Recall the construction of {\em Heegner points}: for an imaginary quadratic order $\cO = \cO_d$ of discriminant $d < 0$,
let $H_d(x)$ denote its {\em Hilbert class polynomial}.

\begin{Definition}
\label{def:heegner}
A point $[z] \in X_0(N)$ is a {\em ``generalized Heegner point''} if there exists a negative discriminant $d$ s.t.
$H_d(j(z)) = H_d(j(Nz)) = 0$.
If in addition we have $(d,2N) = 1$, then $[z]$ is a {\em Heegner point}.
\end{Definition}

For any discriminant $d$, let $E_d$ denote the quadratic twist of $E$ by $d$. Then the Gross-Zagier formula in \cite{gross1986heegner} together with a non-vanishing theorem for $L(E_d,1)$ (see, for example, Bump, Friedberg, and Hoffstein \cite{bump1990nonvanishing}) implies the following

\begin{theorem}
\label{thm0}
(1) If $r_{\an}(E) = 1$, then there exists a Heegner point $[z]$ on $X_0(N)$ such that  $\tr(\varphi([z]))$ has infinite order in $E(\bQ)$. \\
(2) If $r_{\an}(E) \geq 2$, then $\tr(\varphi([z])) \in E(\bQ)_{\tors}$ for every  ``generalized Heegner point'' $[z]$ on $X_0(N)$.
\end{theorem}

The first case in the above theorem is essential to the proof of rank BSD conjecture for $r_{\an}(E) = 1$. We observe that the defining generators of the critical subgroup also take the form $\tr(\varphi([z]))$. Then a natural question is:

\begin{question}
Does there exist an elliptic curve $E$ defined over $\bQ$ such that \\
$r_{\an}(E) \geq 2$ and $\rank(E_{\crit}(\bQ)) >0$?
\end{question}
We will show that the answer is negative for all elliptic curves with conductor $N <1000$, using {\em critical polynomials} attached to elliptic curves.



\subsection{Main results}
\label{sec:form of result}
Let $E, N, \varphi$, and $\omega$ be as defined previously, and write $\Div(\omega) = \sum_{[z] \in X_0(N)} n_z[z]$. Let $j$ denote the $j$-invariant function.
\begin{Definition}
The {\em critical j-polynomial} of $E$ is
\begin{equation*}
	F_{E,j}(x) = \prod_{z \in \supp \Div(\omega), j(z) \neq \infty}(x-j(z))^{n_z}.
\end{equation*}
\end{Definition}
Because $\Div(\omega)$ is defined over $\bQ$ and has degree $2g(X_0(N))-2$, we have $F_{E,j}(x) \in \bQ[x]$ and $\deg F_{E,j} \leq 2g(X_0(N))-2$, where equality holds if $\Div(\omega)$ does not contain cusps. For any non-constant modular function $h \in \bQ(X_0(N))$, the {\em critical $h$-polynomial} of $E$ is defined similarly, by replacing $j$ with $h$.

In this paper we give two algorithms {\em Poly Relation} and {\em Poly Relation-YP} to
compute critical polynomials. The algorithm {\em Poly Relation} computes the critical $j$-polynomial $F_{E,j}$, and the algorithm {\em Poly Relation} computes the critical $h$-polynomial $F_{E,h}$ for some modular function $h$ chosen within the algorithm. We then relate the critical polynomials to the critical subgroup via the following theorem. Recall that 
$H_d(x)$ denotes the Hilbert class polynomial associated to a negative discriminant $d$. We prove the following theorem.
\begin{theorem}
\label{thm1}
Suppose $r_{\an}(E) \geq 2$, and assume at least one of the following holds: \\
(1) $F_{E,h}$ is irreducible for some non-constant function $h \in \bQ(X_0(N))$. \\
(2) There exist negative discriminants $D_k$ and positive integers $s_k$ for $1 \leq k \leq m$ with
$\bQ(\sqrt{D_{k}}) \neq \bQ(\sqrt{D_{k'}})$ for all $k \neq k'$, and an irreducible polynomial $F_0 \in \bQ[x]$, such that 
\[
F_{E,j} = \prod_{k =1}^m H_{D_{k}}^{s_k}\cdot F_0.
\]

Then $\rank(E_{\crit}(\bQ))  = 0$.
\end{theorem}
Combining Theorem~\ref{thm1} with our computation of critical polynomials, we verified the main result of this chapter stated in the following corollary.
\begin{Corollary}
\label{cor2}
For all elliptic curves $E$ of analytic rank 2 and conductor $N$ smaller than 1000, the rank of $E_{\crit}(\bQ)$ is zero.
\end{Corollary}


\section{The norm method}


Let $E$ be an elliptic curve over $\bQ$ with square free conductor $N$, and let $f$ be the associated newform. Taking $A_1, \cdots, A_n$ to be a set of representatives for $\Gamma_0(N) \backslash SL_2(\bZ)$, we define

\begin{Definition}
The {\em norm} of $f$ is the product 
\[
\Norm_N(f) = \prod_{i=1}^n f|[A_i]_2
\]
\end{Definition}
By construction, we know that $\Norm_N(f)$ is a modular form of weight $2n$ on $SL_2(\bZ)$.
\begin{Remark}
In practice we normalise  $\Norm_N(f)$ so that its $q$-expansion has leading coefficient 1. 
\end{Remark}

We recall from \cite[III.7]{diamond2005first} the formulae for the number of {\em elliptic points} on $\Gamma_0(N)$ of order 2 and 3:

\[
\epsilon_2(N) = \prod_{p \mid N} \left( 1+\left(\frac{1}{p}\right) \right), \; \epsilon_3(N) = \prod_{p \mid N} \left( 1+\left(\frac{-3}{p}\right) \right). 
\]
Let $E_4, E_6$ be the {\em Eisenstein series} of level 1 weight 4 and 6, respectively. Let 
$\Delta$ be the {\em discriminant modular form}, which is a cusp form of level 1 and weight 12.

\begin{theorem}
If
\[
    F_f(q) = \frac{\Norm_N(f)(q)}{\Delta^A E_4^B E_6^C}
\]
where 
\begin{align*}
B = \epsilon_3(N)k,  \; \; C = \frac{\epsilon_2(N)k}{2}, \; \mbox{ and } A = \frac{k[SL_2(\bZ):\Gamma_0(N)]-4B-6C}{12}.
\end{align*}
Then $F_{E,j}(j(q)) = F_f(q)$.
\end{theorem}


If we can compute $\Norm_N(f)$, we will have an algorithm to compute $F_{E,j}(x)$ by `cancelling the poles'
as done in \cite{ahlgren2003weierstrass}.  We are going to compute the q-expansion of $\Norm_N(f)$ when $N$ is square free. First, we deal with the case when $N = p$ is prime. Following \cite{ahlgren2003weierstrass}, 
 we define
\begin{Definition}
For any holomorphic function $f: H \to \bC$
\[
     \mathfrak{N}_p(f) = f \cdot \prod_{i=0}^{p-1} f \left(\frac{z+i}{p} \right).
\]
\end{Definition}


\begin{Lemma}\cite{ahlgren2003weierstrass}
\label{lemma: ono}
Let $f$ be a newform of prime level $p$, then $\Norm_p(f) = \N_p(f)$.
\end{Lemma}

This lemma allows us to compute the $q$-expansion of $\N_p(f)$ from the $q$-expansion of $f$.


A slight generalization of Lemma \ref{lemma: ono} yields
\begin{Lemma}(C.)
\label{lemma: norm-sqfree}
If $N$ is square free with prime factorization $N = p_1\cdots p_n$ then
\[
    \Norm_N(f) = \N_{p_1} \circ \N_{p_2} \circ \cdots \N_{p_n}(f).
\]
\end{Lemma}

The key idea of the proof is that when $N$ is square free, the coset representatives of $\Gamma_0(N) \backslash SL_2(\bZ)$ have a simple description. To be precise, we state a lemma.
\begin{Lemma}
\label{lemma: cosrep}

Let $p,M$ be positive integers with $p$ prime,$(p,M)  =1$. Consider the matrices
\[
	\alpha_i  = \abcd{1}{0}{iM}{1} , \; 0 \leq i \leq p-1, \alpha_p = \abcd{1}{(mp-1)/M}{M}{mp},
\]
where $m$ is any integer such that $mp \equiv 1 \pmod{M}$.
Then 
\[
	\Gamma_0(pM) \backslash \Gamma_0(M) =  \bigcup_{i =0}^{p-1} \Gamma_0(pM) \alpha_i.
\]
\end{Lemma}

The proof of Lemma \ref{lemma: cosrep} is omitted. Note that it is a special case of \cite[III, Ex 3.7.7]{diamond2005first}).


\textit{Proof of Lemma} \ref{lemma: norm-sqfree}: We use induction on the number of prime divisors $\sigma(N)$ of 
$N$. The case $\sigma(N) = 1$ is covered by Lemma \ref{lemma: ono}. For the inductive step, choose any prime $p \mid N$ and 
write $N = pM$. Let $\{\beta_j\}$ be a set of coset representatives for $\Gamma_0(M) \backslash SL_2(\bZ)$. By Lemma
\ref{lemma: cosrep}, if $\alpha_i  = \abcd{1}{0}{iM}{1}$, then
\[
	\Norm_N(f) = \sum_{i,j} f|{[\beta_j][\alpha_i]} = \sum_i \Norm_M(f)|[\alpha_i].
\]
We are going to show $SL_2(\bZ) = \cup \Gamma_0(p) \alpha_i$. This can be seen by 
direct calculation:  first, $\abcd{1}{0}{iM}{1}\abcd{1}{0}{jM}{1}^{-1} = \abcd{*}{*}{(i-j)M}{1}$;
second, $\abcd{1}{(mp-1)/M}{M}{mp}\abcd{1}{0}{iM}{1}^{-1} = \abcd{*}{*}{M(1-mpi)}{*}$. Therefore we have 
\[
	\Norm_N(f) = \Norm_p(\Norm_M(f)).
\]
Using the inductive hypothesis $\Norm_M(f) = \prod_{p \mid M} \N_p(f)$, we conclude that 
$\Norm_N(f) = \prod_{p \mid N} \N_p(f)$. The proof is complete.


Now we can describe our algorithm to compute $F_{E,j}$ when the conductor of $E$ is square free. 

\begin{algorithm}
\caption{Norm method to compute $F_{E,j}$ when $N_E$ is square free.}          % give the algorithm a caption
\label{NORM}                           % and a label for \ref{} commands later in the document
\begin{algorithmic} [1]               % enter the algorithmic environment
    \Require  $E$ = Elliptic curve over $\bQ$ with conductor $N$.   
    \Ensure The critical $j$-polynomial $F_{E,j}(x)$.
    
    \State{Use Lemma?, compute the $q$-expansion of $\Norm_N(f)$.}
    \State{Use Theorem? to compute $F_f(q)$. Normalize $F_f(q) $ so that it has leading coefficient 1}.
    \State{Set $n \Leftarrow -\ord_{q}F_f(q)$. Compute q-expansion of $j$ to precision $2n$}.
    \While{$n >= 0$}
	\State $a_n \Leftarrow q^{-n} \mbox{ coefficient } of F_f$.
	\State{$F_f \Leftarrow F_f - a_nj^n$}.
	\State $n \Leftarrow n - 1$.
    \EndWhile
    \State Output $F_{E,j}(x) = \sum_{i=0}^n a_ix^i$.
\end{algorithmic}
\end{algorithm}



\section{The algorithm {\em Poly relation}}
\label{sec: IPR}

Let $C/\bQ$ be a nonsingular projective curve. For a rational function $r \in \bQ(C)$, let $\Div_0(r)$ denote its divisor of 
zeros, and define $\deg r=  \deg(\Div_0(r))$.
\begin{Definition}
Let $C/\bQ$ be a nonsingular projective curve, and let $r, u$ be two non-constant rational functions on $C$. 
A {\em minimal polynomial relation between $r$ and $u$} is an irreducible polynomial $P(x,y) \in \bQ[x,y]$ such that $P(r,u) = 0$ and $deg_x(P) \leq \deg u, \deg_y(P) \leq \deg r$.
\end{Definition}
Minimal polynomial relation always exists and is unique up to scalar multiplication.
Write $\Div(r) =  \sum_{[z] \in X_0(N)} n_z[z]$ and $P(x,y) = f_n(y)x^n + \cdots + f_1(y)x + f_0(y)$. We will prove that  

\begin{Prop}\label{prop: multiplicity}
If $\bQ(C) = \bQ(r,u)$ and $\gcd(f_0(y), f_n(y)) = 1$, then there is a constant $c \neq 0$ s.t.
\[
		f_0(y) = c \prod_{z \in\Div_{0}(r) \setminus \Div_{\infty}(u)} (y - u(z))^{n_z}.
\]
\end{Prop}

\begin{proof}
Dividing $P(x,y)$ by $f_n(y)$, we get $x^n + \cdots + \frac{f_0(x)}{f_n(y)}$, which is a minimal polynomial of $r$ over $\bQ(u)$. So $\Norm_{\bQ(r,u)/\bQ(u)}(r) = \frac{f_0(u)}{f_n(u)}$. The rest of the proof uses a fact on extensions of valuations(see, for example,  \cite[Theorem 17.2.2]{stein2012algebraic}), which we now quote.

\begin{Lemma}\label{lem: valuations}
Suppose $v$ is a nontrivial valuation on a field $K$ and let $L$ be a finite extension of $K$. Then for any $a \in L$,
\[
	\sum_{1 \leq j \leq J} w_j(a) = v(\Norm_{L/K}(a)),
\]
where the $w_j$ are normalized valuations equivalent to extensions of $v$ to $L$.
\end{Lemma}

We continue with the proof. For any $z_0 \in C$ such that $u(z_0) \neq \infty$, consider the valuation $v = \ord_{(u - u(z_0))}$ on $\bQ(u)$. The set of extensions of $v$ to $\bQ(C) = \bQ(r,u)$ is in bijection with $\{ z \in C : u(z) = u(z_0) \}$. Take $a = r$ and
apply Lemma~\ref{lem: valuations}, we obtain
\[
\sum_{z: u(z) = u(z_0)} \ord_z(r)  = \ord_{u-u(z_0)} \frac{f_0(u)}{f_n(u)}.
\]
Combining the identities for all $z_0 \in C \setminus \Div_\infty(u)$, we have for some constant $c$,
\[
	\prod_{z \in \Div(r): u(z) \neq \infty}{(y-u(z))^{n_z}}	=	c  \cdot \frac{f_0(y)}{f_n(y)}.
\]
If $r(z) = 0$, then the condition $\gcd(f_0(y), f_n(y)) = 1$ implies that $f_0(u(z)) = 0$ and $f_n(u(z)) \neq 0$. Therefore,
since $\gcd(f_0,f_n) = 1$, we must have 
\[
		f_0(y) = c \prod_{z \in \Div_{0}(r) \setminus \Div_{\infty}(u)} (y - u(z))^{n_z}.
\]
This completes the proof.
\end{proof}

For completeness we also deal with the case where $u(z) = \infty$, which was left out in the above proof. The corresponding valuation on $\bQ(u)$ is $\ord_{\infty}$ defined by 
$\ord_{\infty}(g/h) = \deg g - \deg h$ for $0 \neq g,h  \in \bQ[u]$. We derive that 
\begin{equation*}
	\sum_{z: u(z) = \infty} \ord_z(r)  = \deg f_n - \deg f_0.
\end{equation*}

Next we apply Proposition \ref{prop: multiplicity} to the computation of $F_{E,j}$.
In the rest of the paper, $\dif j = j'(z)dz$ is viewed as a differential on $X_0(N)$.
Fix the following two modular functions on $X_0(N)$:
\begin{equation}
\label{eq: ru}
r = j(j-1728) \frac{\omega}{\dif j}, \;  u = \frac{1}{j}.
\end{equation}


First we compute the divisor of $r$. Let $\cE_2(N)$ and $\cE_3(N)$ denote
the set of elliptic points of order 2 and 3 on $X_0(N)$, respectively. Then
\begin{equation}
\label{eq: divdj}
	\Div(\dif j) =  -j^*(\infty) - \sum_{c = cusp} c + \frac{1}{2} \left( j^*(1728) - \sum_{z \in \cE_2(N)} z \right) + \frac{2}{3} \left( j^*(0) - \sum_{z \in \cE_3(N)} z \right).
\end{equation}
Writing $j^*(\infty) = \sum_{c = cusp} e_c[c]$, we obtain
\begin{equation}
\label{eq: divr}
	\Div(r) = \Div(\omega) + \frac{1}{2} \left( j^*(1728) + \sum_{z \in \cE_2(N)} z \right) +  \frac{1}{3} \left( j^*(0) + 2\sum_{z \in \cE_3(N)} z \right)- \sum_{c = cusp} (e_c-1)[c].
\end{equation}
Note that (\ref{eq: divr}) may not be the simplified form of $\Div(r)$, due to possible cancellations when $\supp \Div(\omega)$ contains cusps. But since the definition of $F_{E,j}$ only involves critical points that are not cusps, the form of $\Div(r)$ in (\ref{eq: divr}) works fine for our purpose.

Next we show $\bQ(r,u) = \bQ(X_0(N))$ for the functions $r, u$ in (\ref{eq: ru}). First we prove a lemma.

\begin{Lemma}
\label{lem: different}
Let $N > 1$ be an integer and $f \in S_2(\Gamma_0(N))$ be a normalized newform. Suppose $\alpha \in SL_2(\bZ)$   and $f|[\alpha] = f$, then $\alpha \in \Gamma_0(N)$.
\end{Lemma}

\begin{proof}
Write $\alpha = \abcd{a}{b}{M}{d}$. First we show that it suffices to consider the case
where $d = 1$. Since $(M,d) = 1$, there exists $y,w \in \bZ$ such that $My+dw = 1$. By replacing $(y,w)$ with $(y+kd, w-kM)$ if necessary, we may
assume $(y,N) = 1$. Now we can find $x,z \in \bZ$ such that $\gamma = \abcd{x}{y}{Nz}{w} \in \Gamma_0(N)$, and  $\alpha \gamma = \abcd{*}{*}{M}{1} \in \SL_2(\bZ)$ with $f|[\alpha\gamma] = f[\gamma] =f $. We then further reduce 
to the case where $\alpha = \abcd{1}{0}{M}{1}$, by noticing that $\abcd{1}{-b}{0}{1}  \in \Gamma_0(N)$ and
\[
	\abcd{1}{-b}{0}{1} \cdot \abcd{a}{b}{M}{1} = \abcd{1}{0}{M}{1}.
\]

Let $w_N = \abcd{0}{-1}{N}{0}$ be the Fricke involution on $X_0(N)$. Then $f|[w_N] = \pm f$, hence $f|[w_N\alpha w_N] = f$. We compute that $w_N\alpha w_N = \abcd{-N}{M}{0}{-N}$, thus
$f(q) = f|[\abcd{-N}{M}{0}{-N}](q) = f(q \zeta_N^{-M})$, where $\zeta_N = e^{2 \pi i /N}$.
The leading term of $f(q)$ is $q$, while the leading term of  $f(q \zeta_N^{-M})$ is $\zeta_N^{-M} q$. So we must have $\zeta_N^{-M} =1$, i.e., $N \mid M$. Hence $\alpha \in \Gamma_0(N)$ and the proof is complete.
\end{proof}

\begin{Prop}
\label{prop: gen}
Let $r, u$ be the two functions on $X_0(N)$ defined in (\ref{eq: ru}), then $\bQ(r,u) = \bQ(X_0(N))$.
\end{Prop}

\begin{Lemma}
Let $g$ be the genus of $X_0(N)$. If $T \geq 2g-2$ is a positive integer, then $rj^T$ and $u$ satisfy the second condition of Proposition \ref{prop: multiplicity}.

%i.e., if $P(x,y) = f_n(y)x^n + \cdots + f_1(y)x + f_0(y)$ is a minimal polynomial relation of $rj^T$ and $u$, then $\gcd(f_n, f_0) = 1$.
\end{Lemma}

\begin{proof}
Let $r_1 = rj^T$. When $T \geq 2g-2$, the support of  $\Div_\infty(r_1)$ is the set of all cusps. Suppose $\gcd(f_n, f_0) >1$. Let  $p(y)$ be an irreducible factor of $\gcd(f_0,f_n)$. Consider the valuation $\ord_p$ on the field $K(y)$. Since $P(x,y)$ is irreducible, there exists
an integer $i$ with $0<i<n$ such that $p(y) \nmid f_i$.  Thus the Newton polygon of $P$ with respect to the valuation 
$\ord_p$ has at least one edge with negative slope and one edge with positive slope. Therefore, for any Galois extension of $L$ of $K(u)$ containing $K(r,u)$ and a valuation $\ord_{\fp}$ on $L$ extending $\ord_p$, where $\fp$ is an irreducible polynomial in $L[y]$ dividing $p(y)$, there exists two conjugates $r', r''$ of $r$ such that $\ord_{\fp}(r') < 0$ and $\ord_{\fp}(r'') >0$. This implies that $\Div_0(r') \cap \Div_\infty(r'') \neq \emptyset$. Fix $L = K(X(N))$, then all conjugates of $r_1$  in $K(X(N))/K(u)$ are of the form $r_1(\alpha z)$ for some $\alpha \in \SL_2(\bZ)$, Hence the set of poles of any conjugate of $r_1$ is the set of all cusps on $X(N)$, a contradiction.
\end{proof}


Note that for any $T \in \bZ$, we have $\bQ(rj^T,u) = \bQ(r,u) = \bQ(X_0(N))$. Hence when $T \geq 2g-2$, the pair $(rj^T,u)$ satisfies both assumptions of Proposition \ref{prop: multiplicity}. We thus obtain
\begin{theorem}
Let $T \geq 2g-2$ be a positive integer and let 
\[
P(x,y) = f_n(y)x^n + \cdots + f_1(y)x + f_0(y)
\]
be a minimal polynomial relation of $rj^T$ and $u$. Then there exist integers $A$, $B$ and a nonzero constant $c$ such that
\[
		F_{E,j}(y) = c f_0(1/y) \cdot y^{A} (y - 1728)^B.
\]
The integers $A$ and $B$ are defined as follows. Let $\epsilon_i(N) = |\cE_i(N)|$ for $i$ = 2 or 3 %denote the number of elliptic points on $X_0(N)$ of period $i$.
and let $d_N = [SL_2(\bZ): \Gamma_0(N)]$, then $A = \deg f_n - T \cdot d_N - \frac{1}{3}(d_N + 2\epsilon_3(N))$ and $B = -\frac{1}{2}(d_N+\epsilon_2(N))$.
\end{theorem}
%    Text of article.

\begin{proof}
Write $\Div(\omega) =\sum_{[z] \in X_0(N)} n_z[z]$. Applying Proposition \ref{prop: multiplicity} to $rj^T$ and $u$, we get
\begin{equation*}
	\prod_{z: u(z) \neq 0,\infty} (y-u(z))^{n_z} \cdot (y-1/1728)^{\frac{1}{2}(d_N+\epsilon_2(N))}  = c f_0(y)
\tag{a}
\end{equation*}
and
\begin{equation*}
	\sum_{z: u(z) = \infty} \ord_z(\omega) + T \cdot d_N + \frac{1}{3}(d_N + 2\epsilon_3(N))  = \deg f_n - \deg f_0.
\tag{b}
\end{equation*}
To change from $u = \frac{1}{j}$ to $j$, we replace $y$ by $1/y$ in (a) and multiply both sides by $y^{\deg f_0}$ to obtain
\[
	\prod_{z: j(z) \neq 0,\infty} (y-j(z))^{n_z} \cdot (y-1728)^{\frac{1}{2}(d_N+\epsilon_2(N))}  = c f_0(1/y)y^{\deg f_0}.
\]
The contribution of $\{z \in \Div(\omega): j(z) = 0\}$ to $F_{E,j}$ can be computed from (b), so
\begin{align*}
	F_{E,j}(y)
	 &= c \cdot y^{\deg f_n - \deg f_0 -  T \cdot d_N - \frac{1}{3}(d_N + 2\epsilon_3(N))}y^{\deg f_0} \cdot (y-1728)^{-\frac{1}{2}(d_N+\epsilon_2(N))} f_0(1/y) \\
	& = c \cdot y^{\deg f_n - T \cdot d_N - \frac{1}{3}(d_N + 2\epsilon_3(N))}(y-1728)^{-\frac{1}{2}(d_N+\epsilon_2(N))} f_0(1/y).
\end{align*}
\end{proof}

Now we describe the algorithm {\em Poly Relation}.
\begin{algorithm}[H]
\caption{{\em Poly relation}}
\label{IPR}
\begin{algorithmic}[1]
    \Require $E$ = Elliptic Curve over $\bQ$; $N$ =  conductor of $E$;  $f$ = the newform attached to $E$. 
    Values of $g =  g(X_0(N))$, $d_N, \epsilon_2(N)$, $\epsilon_3(N)$, and $c_N$ = number of cusps of $X_0(N)$.
    \Ensure The critical $j$-polynomial $F_{E,j}(x)$.
    \State  Fix a large integer $M$. $T := 2g-2$.
    \State  $r_1:= j^{2g-1}(j-1728)\frac{f}{j'}$, $u: = \frac{1}{j}$.
    \State $\deg r_1 :=  (2g-1)d_N - c_N, \deg u := d_N$.
    \State Compute the $q$-expansions of $r_1$ and $u$ to $q^{M}$.
    \State Let $\{c_{a,b}\}_{0 \leq a \leq \deg u, 0 \leq b \leq \deg r_1}$ be unknowns, compute a vector that spans the one-dimensional vector space \\
    $K$ = $\{(c_{a,b}) : \sum c_{a,b} r(q)^a u(q)^b \equiv 0 \pmod {q^M} \}$.

	\State $P(x,y) :=  \sum c_{a,b}x^ay^b$. Write $P(x,y) = f_n(y)x^n + \cdots + f_1(y)x + f_0(y)$.
	\State $A := \deg f_n - T \cdot d_N - \frac{1}{3}(d_N + 2\epsilon_3(N))$, $B :=  -\frac{1}{2}(d_N+\epsilon_2(N))$.
	\State Output $F_{E,j}(x) = c f_0(1/x) \cdot x^{A} (x - 1728)^B$.
\end{algorithmic}
\end{algorithm}

Note that an upper bound on the number of terms $M$ in the above algorithm can be taken to be $2\deg r \deg u +1$, by the following lemma.
\begin{Lemma}
Let $r, u \in \bQ(X_0(N))$ be non-constant functions. If there is a polynomial $P \in \bQ[x,y]$
such that $\deg_x P \leq \deg u$, $\deg_y P \leq \deg r$, and
\[
	P(r,u) \equiv 0 \pmod{q^M}
\]
for some $M > 2\deg u \deg r$,  then $P(r,u) = 0$.
\end{Lemma}

\begin{proof}
Suppose $P(r,u)$ is non-constant as a rational function on $X_0(N)$, then $\deg P(r,u) \leq \deg r^{\deg u} u^{\deg r} = 2\deg u \deg r$. It follows from $P(r,u) \equiv 0 \pmod{q^M}$ that $\ord_{[\infty]}P(r,u) \geq M$. Since $M> 2\deg u \deg r$, the number of zeros of $P(r,u)$ is greater than its number of poles, a contradiction. Thus $P(r,u)$ is a constant function. But then $P(r,u)$ must be 0 since it has a zero at $[\infty]$. This completes the proof.
\end{proof}

\begin{Remark}
When $N$ is square free, there is a faster method that computes $F_{E,j}$ by computing
the {\em Norm} of the modular form $f$, defined as $\Norm(f) = \prod f|[A_i]$, where $\{ A_i \}$ is
a set of right coset representatives of $\Gamma_0(N)$ in $\SL_2(\bZ)$. This approach is
inspired by Ahrlgen and Ono \cite{ahlgren2003weierstrass}, where $j$-polynomials of Weierstrass points on $X_0(p)$ are computed for $p$ a prime.
\end{Remark}

\begin{Remark}
\label{rem: in practice}
In practice, in order to make the algorithm faster, we make different choices of $r$ to make $\deg r$ small. 
Let $\eta$ denote the Dedekind $\eta$-function and let $\Delta =  \eta^{24}$ denote the discriminant modular 
form of level 1 and weight 12. When $4 \mid N$ we may take $r_4 = \frac{\omega jh_2}{\dif j (32+h_4)}$, where $h_2 = \frac{\Delta(z)-512\Delta(2z)}{\Delta(z)+256\Delta(2z)}$ and $h_4 = (\eta(z)/\eta(4z))^8$. Then $\Div(r_4) = \Div(\omega) + D - D'$, where $D$ and $D'$
are supported on the cusps of $X_0(N)$, and $\deg D = c_N - \delta$, where $\delta$ is the number of
cusps on $X_0(N)$ that are equivalent to $[\infty]$ modulo $\Gamma_0(4)$. Hence $r_4$ has a relatively small degree
and is better suited for computation.
\end{Remark}

\begin{Remark}
In order to speed up the computation, instead of taking $T = 2g-2$ in the algorithm, we may take $T = 0$. First, if $\Div(\omega)$ does not contain cusps(for example, this happens if $N$ is square free), then the functions $r$ and $u$ already satisfies the assumptions of Proposition \ref{prop: multiplicity}. Second, if $\Div(\omega)$ does contain cusps, then $\deg (r)$ will be smaller than its set value
in the algorithm, due to cancellation between zeros and poles. As a result, the vector space $K$ will have dimension greater than 1. Nonetheless, using a basis of  $K$, we could construct a set of polynomials $P_i(x,y)$ with $P_i(r,u) = 0$. Now $P(x,y)$ is the greatest common divisor of the $P_i(x,y)$.
\end{Remark}

We show a table of critical $j$-polynomials. Recall that $H_d(x)$ denotes the Hilbert class polynomial associated to a negative discriminant $d$. We use Cremona's labels \cite{cremona2004elliptic} for elliptic curves in Table~\ref{table: small}.

\begin{table}[h]
   \centering
   \caption{Critical polynomials for some elliptic curves with conductor smaller than $100$}
   \vspace{5mm}
    \begin{tabular}{ | l | l | l |p{5cm} |}
    \hline
    $E$ & $g(X_0(N))$ & Factorization of $F_{E,j}(x)$     \\[5pt] \hline \hline
    37a & 2 & $H_{-148}(x)$ \\ \hline
    37b & 2 &  $H_{-16}(x)^2$  \\ \hline
    44a &   4  &   $H_{-44}(x)^2$ \\ \hline
    48a &  3   &  $1$ \footnotemark \\ \hline
    67a &  5  &  $x^8 + 1467499520383590415545083053760x^7 + \cdots$ \\ \hline
    89a &  7  & $H_{-356}(x)$ \\ \hline
    \end{tabular}
    \label{table: small}
   \end{table}

\footnotetext[1]{In this case $\Div(\omega) = [1/4] + [3/4] + [1/12] +[7/12]$ in supported
on cusps.}


\section{Yang pairs and the algorithm {\em Poly Relation-YP}}
\label{sec: yang pair}

The main issue with the algorithm {\em Poly Relation} is efficiency. The matrix we used to solve for $\{c_{a,b}\}$ has size roughly of the same magnitude as conductor $N$. As $N$ gets around $1000$, computing the matrix kernel quickly becomes impractical. So a new method is needed.

We introduce an algorithm {\em Poly Relation-YP} to compute critical polynomials attached to elliptic curves. The algorithm is inspired by an idea of Yifan Yang in \cite{yang2006defining}. The algorithm {\em Poly Relation-YP} does not compute the critical $j$-polynomial. Instead, it computes a critical $h$-polynomial, where $h$ is some non-constant modular function on $X_0(N)$ chosen within the algorithm. First we restate a lemma of Yang.
\begin{Lemma}[Yang \cite{yang2006defining}]
\label{lem: yanggood}
Suppose $g$, $h$ are modular functions on $X_0(N)$ with a unique pole of order $m$, $n$ at the cusp $[\infty]$, respectively, such that $\gcd(m,n) =1$. Then \\
(1) $\bQ(g,h) = \bQ(X_0(N))$. \\
(2) If the leading Fourier coefficients of $g$ and $h$ are both 1, then there is a minimal
polynomial relation between $g$ and $h$ of form
\begin{equation}
\label{eq: yang}
	y^m - x^n + \sum_{a,b \geq 0, am+bn < mn} c_{a,b}x^ay^b.
\end{equation}
\end{Lemma}

\begin{Definition}
A pair of two non-constant modular functions on $X_0(N)$ is said to be a {\em Yang pair} if they satisfy the assumptions of Lemma \ref{lem: yanggood}. 
\end{Definition}
Following \cite{yang2006defining}, we remark that in order to find a minimal polynomial relation of a Yang pair, we can compute the Fourier expansion of $y^m - x^n$ and use products of form $x^ay^b$ to cancel the pole at $[\infty]$ until we reach zero. This approach is significantly faster than the method we used in {\em Poly Relation}, which finds a minimal polynomial relation of two arbitrary modular functions. This gain in speed is the main motivation of introducing {\em Poly Relation-YP}.


Let 
\[ 
\eta = q^{\frac{1}{24}}\prod_{n \geq 1}(1-q^n)
\] be the Dedekind $\eta$ function.
For any positive integer $d$, define the function $\eta_d$ as $\eta_d(z) = \eta(dz)$. 

Let $N$ be a positive integer. An {\em $\eta$-product of level $N$} is a function of the form 
\[ 
h(z) = \prod_{d \mid N} \eta_d(z)^{r_d}
\]
where $r_d \in \bZ$ for all $d \mid N$.

The next theorem of Ligozat gives sufficient conditions for a $\eta$-product to be a modular function on $X_0(N)$.
\begin{Lemma}[Ligozat's Criterion \cite{ligozat1975courbes}]
Let $h = \prod_{d \mid N} \eta_d(z)^{r_d}$ be an $\eta$-product of
level $N$. Assume the following: \\
(1)  $\sum_d r_d \frac{N}{d} \equiv 0 \pmod{24}$; (2)  $\sum_d r_d d \equiv 0 \pmod{24}$;
(3)  $\sum_d r_d = 0$;  \\
(4) $\prod_{d \mid N} (\frac{N}{d})^{r_d} \in \bQ^2$.  \\
Then $h$ is a modular function on  $X_0(N)$.
\end{Lemma}

If $h \in \bQ(X_0(N))$ is an $\eta$-product, then the divisor $\Div(h)$ is supported on the cusps of $X_0(N)$. The next theorem allows us to construct $\eta$-products with prescribed divisors.

\begin{Lemma}[Ligozat \cite{ligozat1975courbes}]
\label{lem:ligozat}
Let $N > 1$ be an integer. For every positive divisor $d$ of $N$, let $(P_d)$ denote the sum of all cusps on $X_0(N)$ of denominator $d$. Let $\phi$ denote the Euler's totient function. Then there exists an explicitly computable $\eta$-product $h \in \bQ(X_0(N))$ such that
\[
\Div(h) = m_d( \, (P_d) - \phi(\gcd(d,N/d))[\infty] \,)
\]
for some positive integer $m_d$.
\end{Lemma}

\begin{Remark}
\label{rem: expliciteta}
By `explicitly computable' in Lemma~\ref{lem:ligozat}, we mean that one can compute a set of integers $\{r_d :  d\mid N \}$ that defines the $\eta$-product $h$ with desired property. It is a fact that the order of
vanishing of an $\eta$ product at any cusp of $X_0(N)$  is an linear combination of
the integers $r_d$. So prescribing the divisor of an $\eta$-product is equivalent to giving a
linear system on the variables $r_d$. Thus we can solve for the $r_d$'s and obtain the $q$-expansion of $h$ from the $q$-expansion of $\eta$.
\end{Remark}

The next proposition is a direct consequence of Lemma~\ref{lem:ligozat}.
\begin{Prop}
\label{cor: majorize}
Let $D \geq 0 $ be a divisor on $X_0(N)$ such that $D$ is supported on the cusps. Then there exists an explicitly computable $\eta$-product $h \in \bQ(X_0(N))$ such that $\Div(h)$ is of the form $D' - m[\infty]$, where $m$ is a positive integer and $D' \geq D$.
\end{Prop}

Recall our notation from section \ref{sec: IPR} that $r = j(j-1728)\frac{\omega}{\dif j}$.
\begin{Prop}
\label{prop: niceproduct}
There exists an explicitly computable function $h \in \bQ(X_0(N))$ such that \\ (1) The functions $rh$ and $j(j-1728)h$
form a Yang pair; \; \\
 (2) $j(j-1728)h$ is zero at all cusps of $X_0(N)$ except the cusp $[\infty]$.
\end{Prop}

\begin{proof}
Let $T = \Div_\infty(j)$. Note that the support of $T$ is the set of all cusps. From (\ref{eq: divr}) we have $\Div_{\infty}(r) \leq T$, $\Div(j(j-1728)) = 2T$, $\ord_{[\infty]}(T) = 1$, and $\ord_{[\infty]}(r) = 0$. Applying Proposition~\ref{cor: majorize} to the divisor $D = 4(T-[\infty])$,  we obtain an $\eta$-product $h \in \bQ(X_0(N))$ such that $\Div(h) = D' - m[\infty]$, where $D' \geq D$ and $m \geq 0$. Then $\Div_{\infty}(rh) = m[\infty]$ and $\Div_{\infty}(j(j-1728)h) = (m+2)[\infty]$. If $m$ is odd, then $(m,m+2) = 1$ and (1) follows. Otherwise, we can replace $h$ by $jh$. Then a similar argument shows that $rh$ and $j(j-1728)h$ have a unique pole at $[\infty]$ and have degree $m+1$ and $m+3$, respectively. Since $m$ is even in this case, we have $(m+1, m+3) = 1$ and (1) holds.

What we just showed is the existence of an $\eta$-product $h \in \bQ(X_0(N))$ s.t. either $h$ or $jh$ satisfies (1).  Now (2) follows from the fact that $\Div_{0}(j(j-1728)h) > 2(T-[\infty])$ and $\Div_{0}(j^2(j-1728)h) > (T-[\infty])$.
\end{proof}

Let $h$ be a modular function that satisfies the conditions of Proposition~\ref{prop: niceproduct}. The next theorem allows us to compute $F_{E,j(j-1728)h}(x)$. For ease of notation, let $\tilde{r} = rh$ and $\tilde{h} = j(j-1728)h$.



\begin{theorem}
\label{thm: yangpoly}
Suppose $h$ is a modular function on $X_0(N)$ that satisfies the conditions in
Proposition~\ref{prop: niceproduct}.  Let $P(x,y)$ be a minimal polynomial relation of $\tilde{r}$ and $\tilde{h}$ of form (\ref{eq: yang}). Write  $P(x,y) = f_n(y)x^n + \cdots + f_1(y)x + f_0(y)$, and let $g$ be the genus of $X_0(N)$, then
\[
		F_{E,\tilde{h}}(x) = x^{2g-2-\deg h}f_0(x).
\]
\end{theorem}

\begin{proof}
The idea  is to apply Proposition \ref{prop: multiplicity} to the Yang pair $(\tilde{r}, \tilde{h})$. By Lemma \ref{lem: yanggood}, every Yang pair satisfies its first assumption. To see the second assumption holds, observe that $f_n(y) = -1$ in (\ref{eq: yang}), so $\gcd(f_n(y),f_0(y))$ = 1. Hence we can apply Proposition \ref{prop: multiplicity} and obtain
\[
 	f_0(y) = \prod_{z \in\Div_{0}(\tilde{r}) \setminus \Div_{\infty}(\tilde{h})} (y - \tilde{h}(z))^{n_z}.
\]
By construction of $h$, there is a divisor $D \geq 0$ on $X_0(N)$ supported on the finite set $j^{-1}(\{0,1728\}) \cup h^{-1}(0)$, such that $\Div(rh) = \Div(\omega) + D - (\deg h)[\infty]$. Taking degrees on both sides shows $\deg D = \deg h - (2g-2)$.  Since $\tilde{h}(z) = 0$ for all $z \in \supp D$, we obtain
\[
	f_0(x) = F_{E,\tilde{h}}(x) \cdot x^{\deg h -2g+2}.
\]
This completes the proof.
\end{proof}


Next we describe the algorithm {\em Poly Relation-YP}.

\begin{algorithm}[H]
\caption{{\em Poly Relation-YP}}
\begin{algorithmic} [1]
    \Require $E$ = Elliptic Curve over $\bQ$, $f$ = the newform attached to $E$.
    \Ensure a non-constant modular function $h$ on $X_0(N)$ and the critical $\tilde{h}$-polynomial $F_{E,\tilde{h}}$, where $\tilde{h} = j(j-1728)h$.
    \State Find an $\eta$ product $h$ that satisfies Proposition~\ref{prop: niceproduct}.

    \State $\tilde{r} :=  j(j-1728)h\frac{f}{j'}$, \; $\tilde{h} := j(j-1728)h$.
    \State $M := (\deg \tilde{r} +1)(\deg \tilde{h} + 1)$.
    \State Compute $q$-expansions of $\tilde{r}$, $\tilde{h}$ to $q^{M}$.
    \State Compute a minimal polynomial relation $P(x,y)$ of form (\ref{eq: yang}) \\using the method mentioned after Lemma \ref{lem: yanggood}.
    \State Output $F_{E,\tilde{h}}(x) = x^{2g-2-\deg h}P(0,x)$.
\end{algorithmic}
\end{algorithm}

\begin{Remark}
The functions $\tilde{r}$ and $\tilde{h}$ in the above algorithm are constructed in order that Theorem
 \ref{thm: yangpoly} has a nice and short statement. However, their degrees are large, which is not optimal for computational purposes. In practice, one can make different choices of two modular functions with smaller degrees to speed up the computation. This idea is illustrated in the following example.
\end{Remark}

\begin{Example}
\label{ex: 664a}
Let $E$ be the elliptic curve 
\[
	E: y^2 = x^{3} - 7 x + 10
\]
labeled as {\bf 664a1} in Cremona's table. Then $r_{\an}(E) = 2$, and $X_0(664)$ has genus 81.  Let $r = r_4$ be as defined in Remark~\ref{rem: in practice}. Using the method described in Remark~\ref{rem: expliciteta}, we found two $\eta$-products \\ 
\[h_1 =  (\eta_2)^{-4}(\eta_4)^6 (\eta_8)^4 (\eta_{332})^6 (\eta_{664})^{-12}, \, h_2 = (\eta_2)^{-1} (\eta_4) (\eta_{166})^{-1} (\eta_8)^2 (\eta_{332})^5 (\eta_{664})^{-6}
\]with the following properties: $h_1, h_2 \in \bQ(X_0(N))$, $\Div(rh_1) = \Div(\omega)  +  D -247[\infty]$, where $D \geq 0$ is supported on cusps, and $\Div(h_2)  = 21[1/332] + 61[1/8] + 21[1/4] - 103[\infty]$. Since (247,103) =1, the functions $rh_1$ and $h_2$ form a Yang pair. We then computed
\[
	F_{E,h_2}(x) =  x^{160} - 14434914977155584439759730967653459200865032120265600267555196444 x^{158}  + \cdots.
\]
The polynomial $F_{E,h_2}$ is irreducible in $\bQ[x]$. \end{Example}

\section{The critical subgroup $E_{crit}(\bQ)$}
\label{sec: crit}
Recall the definition of the critical subgroup for an elliptic curve $E/\bQ$: 
\[
E_{\crit}(\bQ) = \langle \tr(\varphi(e)): e \in \supp \Div(\omega)\rangle.
\]
Observe that to generate $E_{\crit}(\bQ)$, it suffices to take one representative from each Galois orbit of $\supp \Div(\omega)$. Therefore, if we let $n_{\omega}$ denote the number of Galois orbits in $\Div(\omega)$, then 
\[
\rank(E_{\crit}(\bQ)) \leq n_{\omega}.
\]
For any rational divisor $D = \sum_{[z] \in X_0(N)} n_z [z]$ on $X_0(N)$, let $p_{D} = \sum_{z \in \supp D} n_z \varphi([z])$, then $p_D \in E(\bQ)$. Note that $p_D = 0$ if $D$ is a principal divisor. The point $p_{\Div(\omega)}$ is a linear combination of the defining generators of $E_{\crit}(\bQ)$.
\begin{Lemma}
\label{lem: ell}
$6 \, p_{\Div(\omega)} \equiv  - 3 \sum_{c \in \cE_2(N)} \varphi(c) - 4 \sum_{d \in \cE_3(N)}  \varphi(d) \pmod{E(\bQ)_{\tors}}$.
\end{Lemma}

\begin{proof}
Let $r_0 = \omega/\dif j$, then $r_0 \in \bQ(X_0(N))$, hence $p_{\Div(r_0)}  =0$.
From $\Div(r_0) = \Div(\omega) - \Div(\dif j)$, we deduce that $p_{\Div(\omega)} = p_{\Div(\dif j)}$. The lemma then follows from the formula of $\Div(\dif j)$ given in (\ref{eq: divdj}) and the fact that the image of any cusp under $\varphi$ is torsion.
\end{proof}

\begin{Prop}
\label{prop: irr}
Assume at least one of the following holds: \\ (1) $r_{\an}(E) \geq 2$; \\(2) $X_0(N)$ has no elliptic point. \\Then $\rank(E_{\crit}(\bQ)) \leq n_\omega - 1$.
\end{Prop}

\begin{proof}
By Lemma~\ref{lem: ell} and Theorem~\ref{thm0}, either assumption implies that $p_{\Div(\omega)}$ is torsion. But $p_{\Div(\omega)}$ is a linear combination of the $n_\omega$ generators of $E_{\crit}(\bQ)$, so these generators are linearly dependent in $E_{\crit}(\bQ) \otimes \bQ$. Hence the rank of $E_{\crit}(\bQ)$ is smaller than $n_\omega$.
\end{proof}

Now we are ready to prove Theorem~\ref{thm1}. \\
{\bf Proof of Theorem~\ref{thm1}}.
First, note that the definition of $F_{E,j}$ only involves critical points that are not cusps.
However, since images of cusps under $\varphi$ are torsion, we can replace $\Div(\omega)$ by $\Div(\omega) \setminus  \{\mbox{ cusps of } X_0(N)\}$ if necessary and assume that $\Div(\omega)$ does not contain cusps.  \\
(1) Let $d = \deg F_0$, then there exists a Galois orbit in $\Div(\omega)$
of size $d$, and the other $(2g-2-d)$ points in $\Div(\omega)$ are CM points. Let $z$ be any one of the $(2g-2-d)$  points, then  $j(z)$ is a root of $H_{D_k}(x)$ and $z \in \bQ(\sqrt{D_k})$. Since $\Div(\omega)$ is invariant under the Fricke involution $w_N$, one sees that $j(Nz)$ is also a root of $F_{E,j}$. Therefore, $j(Nz)$ is the root of $H_{D_{k'}}(x)$ for some $1 \leq k' \leq m$.  Since $z$ and $Nz$ define the same quadratic field, we must have $\bQ(\sqrt{D_k}) = \bQ(\sqrt{D_{k'}})$, which implies $k = k'$ by our assumption. It follows
that $[z]$ is a ``generalized Heegner point''(as defined in Definition~\ref{def:heegner}) and  $\tr(\varphi([z]))$ is torsion.  By the form of $F_{E,j}$, there exists a point $[z_0] \in \supp \Div(\omega)$ such that $j(z_0)$  is a root of $F_0$. Then we have $\rank(E_{\crit}(\bQ))= \rank(\langle \tr(\varphi([z_0]) \rangle) = \rank(\langle p_{\Div(\omega)} \rangle)$. Finally, Lemma~\ref{lem: ell} implies $\langle p_{\Div(\omega)} \rangle = 0$, and it follows that $\rank(E_{\crit}(\bQ)) = 0$.

(2) If $F_{E,h}$ is irreducible,  then we necessarily have $n_\omega = 1$, and the claim follows from Proposition \ref{prop: irr}.

\begin{Remark}
Christophe Delaunay has an algorithm to compute $\Div(\omega)$ numerically as equivalence classes of points in the upper half plane(see \cite{delaunay2002thesis} and \cite{delaunay2005critical}). A table of critical points for the elliptic curve 
\[
	E: y^2 + y = x^{3} + x^{2} - 2 x
\]
with rank 2 and Cremona label {\bf 389a} is presented in \cite[Appendix B.1]{delaunay2002thesis}. The results suggested that $\Div(\omega)$ contains two Heegner points of discriminant 19, and the critical subgroup $E_{\crit}(\bQ)$ is torsion. Using the critical $j$-polynomial for {\bf 389a} in Table~\ref{table: rank two}, we can confirm the numerical results of Delaunay.
\end{Remark}


\section{Data: critical polynomials for rank two elliptic curves}
\label{sec: table}
The columns of Table~\ref{table: rank two} are as follows. The column labeled $E$ contains labels
of elliptic curves, and those labeled $g$ contains the genus of $X_0(N)$, where $N$ is the conductor of $E$. The column labeled $h$ contains a modular function on $X_0(N)$: either the $j$ invariant or some $\eta$-product. The last column contains the factorization of the critical $h$-polynomial of $E$ defined in Section ~\ref{sec:form of result}. The factors of $F_{E,j}$ that are Hilbert class polynomials are written out explicitly. Table \ref{table: rank two} contains {\em all} elliptic curves with conductor $N \leq 1000$ and rank 2. By observing that all the critical polynomials in the table satisfy one of the assumptions of
Theorem~\ref{thm1}, we obtain Corollary~\ref{cor2}.


From our computation, it seems hard to find an elliptic curve $E/\bQ$ with $r_{\an}(E) \geq 2$ and $\rank(E_{\crit}(\bQ)) > 0$. Nonetheless, some interesting questions can be raised.

\begin{question}
\label{q1}
For all elliptic curves $E/\bQ$, does $F_{E,j}$ always factor in $\bQ[x]$ as a product of Hilbert class polynomials and one irreducible polynomial?
\end{question}

If the answer to Question~\ref{q1} is positive, then we would know $E_{\crit}(\bQ)$ is torsion wheneve $r_{an}(E) \geq 2$. 



Another way to construct rational points on $E$ is to take any cusp form $g \in S_2(\Gamma_0(N), \bZ)$ and define $E_{g}(\bQ) = \langle \tr(\varphi([z]) : [z] \in \supp \Div(g(z)dz) \rangle$.
\begin{question}
Does there exist $g \in S_2(\Gamma_0(N), \bZ)$ such that $E_{g}(\bQ)$ is non-torsion?
\end{question}

\begin{Remark}
Consider the irreducible factors of $F_{E,j}$ that are {\em not} Hilbert class polynomials. It turns out that their constant terms has many small
primes factors, a property also enjoyed by Hilbert class polynomials. For example, consider the polynomial  $F_{67a, j}$. It is irreducible and not equal to any Hilbert class polynomial, while its constant term has factorization
\[
2^{68} \cdot 3^{2} \cdot 5^{3} \cdot 23^{6} \cdot 443^{3} \cdot 186145963^{3}.
\]
It is interesting to investigate the properties of these polynomials.
\end{Remark}


   \begin{table}[H]
   \caption{Critical polynomials for elliptic curves of rank 2 and conductor $<1000$}
   \vspace{5mm}
   \centering
    \begin{tabular}{ | l |  l  | l  |p{4.4cm}  |}
    \hline
    $E$ & $g(X_0(N))$   & $h$ & $\mbox{ Factorization of } F_{E,h}(x)$     \\ \hline \hline
    389a & 32  & $j$ & $H_{-19}(x)^2 (x^{60}+ \cdots)$ \\ \hline
    433a & 35  & $j$ &  $x^{68}+\cdots$  \\ \hline
     446d & 55  & $j$ &  $x^{108}+\cdots$ \\ \hline
    563a & 47  & $j$ &  $H_{-43}(x)^2 (x^{90} - \cdots)$   \\ \hline
    571b& 47  & $j$ &  $H_{-67}(x)^2 (x^{90} - \cdots)$ \\ \hline
    643a& 53  & $j$ &  $H_{-19}(x)^2 (x^{102} - \cdots)$ \\ \hline
    664a & 81    &   $\frac{\eta_4\eta_8^2 \eta_{332}^5}{\eta_{166}\eta_{664}^{6}{\eta_2}}$ & $x^{160} - \cdots$ \\ \hline
    655a& 65  & $j$ &  $x^{128} - \cdots$ \\  \hline
    681c& 75  & $j$ &  $x^{148} - \cdots$ \\  \hline
    707a & 67  & $j$ & $x^{132} - \cdots$  \\ \hline
    709a& 58  & $j$ &  $x^{114} - \cdots$\\ \hline
    718b& 89  & $j$ &  $ H_{-52}(x)^2 (x^{172} - \cdots)$\\ \hline
    794a& 98  & $j$ &  $H_{-4}(x)^2 (x^{192} - \cdots)$\\ \hline
    817a& 71  & $j$ &  $x^{140} - \cdots$\\ \hline
    916c & 113   & $j$ &$H_{-12}(x)^8(x^{216}+\cdots)$  \\ \hline


    944e & 115    & $\frac{\eta_{16}^4 \eta_{4}^2}{\eta_8^6}$ & $x^{224} - \cdots$ \\ \hline
    997b& 82  & $j$ &  $H_{-27}(x)^2 (x^{160} - \cdots)$\\ \hline
    997c& 82  & $j$ &  $x^{162} - \cdots$\\ \hline
    \end{tabular}
    \label{table: rank two}
   \end{table}




 
% ========== Chapter 1

\chapter{Fourier expansions of modular forms forms at all cusps}

Let $k$ be a positive even integer and let $f \in S_k(\Gamma_0(N))$ be a nonzero cusp form.
Then $f$ has a Fourier expansion at the cusp infinity: $$f = \sum_{n \geq 1} a_n(f) q^n$$
where $a_n$ are complex numbers and $q = e^{2 \pi i \tau}$. 
We are concerned with the problem of computing the Fourier expansion of $f$ at other cusps. When $N$ is 
square-free, this problem is solved by Asai \cite{asai1976fourier}.  The problem is studied in the Ph.D. thesis of Christophe Delaunay and in \cite{couveignes2011computational}, where a numerical algorithm is proposed. We will give a numerical algorithm to compute such expansions.  Our approach is different from the one proposed in \cite{couveignes2011computational}, for they require working at a higher level: to compute expansions at cusps of denominator $Q$, one needs to compute period matrices for forms of level $NR^2$, where $R = \gcd(Q, \frac{N}{Q})$. As a contrast, our algorithm works at levels dividing $N$. 

The main results of this chapter are Theorem~\ref{thm: ExpansionFormula} and Algorithm~\ref{alg: qexp}. The former gives a formula for the Fourier expansion of a newform $f \in S_k(\Gamma_0(N))$ at any cusp $z$ of width one, and the latter describes how to use the formula to explicitly compute such expansion. Along the way, we will develop algorithms to compute the twists $f \otimes \chi$ and the pseudo-eigenvalue of newforms under the Fricke involution. 

Section contains some examples. % In section, we describe an alternative approach and 

\section{Preliminaries}

Let $N \geq 1$ be an integer and let $X_0(N)$ be the modular curve of level $N$. 
\begin{Definition}
Let $z$ be a cusp on $X_0(N)$.  If $z \neq \infty$, write $z = [a/c]$ with $\gcd(a,c) =1$. The {\it denominator} of $z$ is $$d_z  = \gcd(c,N).$$ If $z = \infty$, we set $d_\infty = N$. Choose $\alpha \in SL_2(\bZ)$ such that $\alpha(\infty) = z$. 
The {\it width} of $z$ is 
\[
	h_z = \left| \frac{SL_2(\bZ)_\infty}{(\alpha^{-1} \{\pm I\} \Gamma_0(N) \alpha)_\infty}\right|
\]
where the subscript $\infty$ means taking  the isotropy subgroup of $\infty$ in the corresponding group.
\end{Definition}

The width of a cusp can be computed in terms of its denominator. In fact, we have
\begin{Lemma}
If $z$ is a cusp on $X_0(N)$, then
$$h_z = \frac{N}{\gcd(d_z^2, N)}.$$
\end{Lemma}

\begin{proof}
When $z = [\infty]$, we have $d_\infty = N$ and $h_\infty =  1$, so the formula holds trivially. Otherwise, write $z = [\frac{a}{c}]$ and find $\alpha = \abcd{a}{b}{c}{d} \in SL_2(\bZ)$. For $N' \in \bZ$ we compute 
\[
	\alpha   \abcd{1}{N'}{0}{1} \alpha^{-1} = \abcd{*}{*}{-c^2N'}{*}.
\]
Hence $\abcd{1}{N'}{0}{1}  \in  (\alpha^{-1} \{\pm I\} \Gamma_0(N) \alpha)_\infty \iff N \mid c^2 N' \iff \frac{N}{\gcd(d_z^2, N)} \mid N'$. This completes the proof.
\end{proof}

In particular, the width of a cusp $z$ is one if and only if $N \mid d_z^2$. 

Suppose $f$ is a modular form on $\Gamma_0(N)$ of positive even weight $k$ and $\alpha \in GL_2(\bQ)$. Recall the weight-$k$ action is defined as 
\[
	f | \alpha (\tau)  = (\det(\alpha))^{k/2} (cz+d)^{-k} f(\alpha \tau), \, \alpha = \abcd{a}{b}{c}{d}.
\]
In particular, if $\alpha \in SL_2(\bZ)$, then $f | \alpha $ is a modular form on $\Gamma(N)$. So $f | \alpha $ has a $q$-expansion, which is a power series in $q^{\frac{1}{N}}$. A natural thing to do is to define the expansion of $f$ at the cusp $z$ as the expansion of $f | \alpha$. However,  note that this may not be well-defined: in general the expansion depends on the choice of $\alpha$. Nonetheless, when the cusp $z$ has width one, the expansion is indeed well-defined as a power series in $q$. 

\begin{Lemma}
Let $z$ be a cusp on $X_0(N)$ with $h_z = 1$. Choose $\alpha \in SL_2(\bZ)$ such that 
$\alpha(\infty) =z$. Then $f | \alpha$ is a cusp form on $\Gamma_1(N)$. Moreover, the function $f|\alpha $ is independent of the choice of $\alpha$. 
\end{Lemma}

\begin{proof}
It is easy to verify that $\Gamma_1(N) \subseteq \alpha^{-1} \Gamma_0(N) \alpha$, hence the first claim holds. Now suppose $\beta \in SL_2(\bZ)$ is such that $\beta(\infty) = z$. Then $\alpha^{-1} \beta \in SL_2(\bZ)_\infty$. Since $z$ has width one, we have $\alpha^{-1} \beta \in \alpha^{-1}\Gamma_0(N) \alpha$. Hence $\beta \in \Gamma_0(N) \alpha$, and it follows that $f | [\beta] = f | [\alpha]$.
\end{proof}

In light of the lemma above, we define the {\it $q$-expansion of $f$ at a width one cusp $z$} to be the $q$-expansion of $f | [\alpha]$, and denote it by $f_z$. 


Assume further that $f$ is an eigenform under the Atkin-Lehner operators. We will show that in order to compute the expansion of $f|[\alpha]$ for any $\alpha \in SL_2(\bZ)$, it suffices to do so for $\alpha = \abcd{1}{0}{m}{1}$, where $0 \leq m < N$ and $N \mid \gcd(m,N)^2$. In particular, it suffices to compute the expansions of $f$ at a some cusps of width one.

\begin{Lemma}
For any $\alpha \in SL_2(\bZ)$, there exists a matrix $w_Q \in W_N$ and an upper triangular matrix $u \in GL_2(\bQ)$
such that $w\alpha = \alpha' u$, where $\alpha' = \abcd{a'}{b}{c'}{d'} \in SL_2(\bZ)$ satisfies $N \mid \gcd(N,c')^2$.
\end{Lemma}

Indeed, one may find $Q$ using Lemma. Now $f|[\alpha] = f|[w_Q][w_Q\alpha] = f|[w_Q][\alpha'][u] = \lambda_Q(f) f [\alpha'][u] =\lambda_Q(f) f[\alpha''][u]$, where $\alpha''$ is of form $\abcd{1}{0}{m}{1}$. Note that for an upper triangular matrix $u = \abcd{u_0}{u_1}{0}{u_2}$, we have $f[u](q) = f(q^{u_0/u_2} e^{2\pi i u_1/u_2})$.

\section{Reducing to the case of newforms}

The space $S_k(\Gamma_0(N))$ is spanned by elements of form $g(q^d)$, where $g$ is newform of level $M \mid N$ and $d$ is a divisor of $\frac{N}{M}$.  Note that $g(q^d) = d^{-k/2} g | \abcd{d}{0}{0}{1}$. For any $\alpha \in SL_2(\bZ)$, 
we can find $\alpha' \in SL_2(\bZ)$ and $u \in GL_2(\bQ)$ such that $\abcd{d}{0}{0}{1} \alpha  = \alpha' u$. Hence to compute all expansions $f | [\alpha]$, it suffices to give an algorithm for newforms.

In the rest of this chapter, we will restrict ourselves to solving the following problem: 

\begin{problem}
Let $f$ be a normalized newform in $S_k(\Gamma_0(N))$ and $z$ be a cusp on $X_0(N)$ of width one. Compute the $q$-expansion of $f_z$.
\end{problem}


\section{Twists of newforms}

For $f \in S_k(\Gamma_1(N), \epsilon)$ a newform with expansion $f = \sum_n a_n(f) q^n$ and $\chi$ a Dirichlet character, the {\it twist} $f_\chi$ is a modular form with expansion $f_\chi (q) = \sum a_n(f) \chi(n)  q^n$. 

\begin{Lemma}\cite[Proposition 3.1]{atkin1978twists}
Let $F \in S_k(\Gamma_1(N), \epsilon)$, where $\epsilon$ is a character of conductor $N'$.  Let $\chi$ be a character modulo $M$. Put $\tilde{N} = lcm(N, N'M, M^2)$. Then $f_\chi \in S_k(\Gamma_1(\tilde{N}), \epsilon \chi^2)$.
\end{Lemma}

In particular, when $\epsilon$ is the trivial character and the conductor $M$ of $\chi$ satisfies $M^2 \mid N$, we have
$F_\chi \in S_k(\Gamma_1(N), \chi^2))$.

We write $f \otimes \chi$ for the unique newform such that $a_p(f \otimes \chi) = a_p(f_\chi)$ for all but finitely many primes $p$. From now, we refer to $f \otimes \chi$ as {\it the twist of $f$ by $\chi$}. 

We quote two more results from \cite{atkin1978twists}, which we will use extensively. First, we recall the definitions of  $U_d$ and $B_d$ operators. For a modular form $f = \sum a_n q^n$ and a positive integer $d$, we put 
\[
	f |U_d = \sum a_{nd} q^n, \,  f |B_d = \sum a_n q^{nd}.
\]
It is easy to see that for any positive integers $d,d'$, we have $U_d$ commutes with $B_{d'}$.





\begin{Lemma}\cite[Theorem 3.1]{atkin1978twists} \label{lemma: winnie-level}
Let $q \mid N$ and $Q$ be the $q$-primary part of $N$. Write $N = QM$. Let $F$ be a newform in $S_k(\Gamma_1(N), \epsilon)$ with $\cond(\epsilon_Q) = q^{\alpha}, \alpha \geq 0$. Let $\chi$ be a character with conductor $q^{\beta}$, $\beta \geq 1$. Put $Q' = \max\{Q,q^{\alpha + \beta}, q^{2\beta}\}$. Then \\
(1)  For each prime $q' \mid M, F_\chi$ is not of level $Q'M/q$. \\
(2)  The exact level of $F_\chi$ is $Q'M$ provided (a) $\max\{q^{\alpha + \beta}, q^{2\beta}\} < Q$ if $Q' = Q$, or 
(b) $\cond(\epsilon_Q \chi) = \max \{q^{\alpha}, q^{\beta} \}$ if $Q' > Q$.
\end{Lemma}


\begin{Lemma}\cite[Theorem 3.2]{atkin1978twists} \label{lemma: fchi}
Let $q \mid N$ and $Q$ be the $q$-primary part of $N$. Write $N = QM$. Let $\chi$ be a character whose conductor equals a power of $q$.  Let $f$ be a newform in $S_k(\Gamma_1(N), \epsilon)$. Then $f \otimes \chi$  is a newform in $S_k(\Gamma_1(Q'M,\epsilon \chi^2)$, where $Q'$ is a power of $q$. Moreover, we have $$f_\chi  = f \otimes \chi - (f \otimes \chi) |U_q |B_q.$$
\end{Lemma}

Since our goal is to compute expansions of newforms on $\Gamma_0(N)$, we will make the following assumptions: 
from now, unless otherwise noted, we assume $f$ has trivial character, and that $\cond(\chi)^2 \mid N$.


Next, we consider the problem of identifying the newform $f \otimes \chi$. This includes finding its level and its $q$-expansion to arbitrarily many terms. We will assume that we have an oracle which, given weight $k$ and level $N$, computes the expansions of all newforms in $S_k(\Gamma_1(N))$ to arbitrarily many terms (for example, use the algorithm in \cite{stein2007modular}).

Now we proceed on how to recognise the level of $f \otimes \chi$ from the coefficients of $f$. One potential obstacle is that we do not know all Fourier coefficients of $f \otimes \chi$: we only know that $a_n(f \otimes \chi)  = a_n(f)\chi(n)$   when $\gcd(n, N) = 1$. This can be overcome using a variant of Sturm's argument. First we prove a lemma.
\begin{Lemma}
Let $f \in S_k(N, \epsilon)$ be a normalized newform and $q$ be any positive integer. Then $f|U_q|B_q \in S_k(Nq^2, \epsilon)$.
\end{Lemma}

\begin{proof}
It is a standard fact that for any integer $d \geq 1$, the map $f \mapsto f|B_d$ takes $S_k(N, \epsilon)$ to 
$S_k(Nd, \epsilon)$. To prove the lemma, we consider two separate cases. First, assume $q \nmid N$, then we have 
$T_q = U_q + q^{k-1} \epsilon(q) B_q$. By our assumption, we have $f|T_q = a_q(f) f$. Therefore, we have 
$f|U_q|B_q = f|(T_q - q^{k-1} \epsilon(q) B_q)|B_q = a_q(f)f|B_q - q^{k-1} \epsilon(q) f|B_q^2$. Hence $f|U_q|B_q \in S_k(Nq^2, \epsilon)$. \\
Now assume $q \mid N$, so $U_q = T_q$. Hence $f|U_q|B_q = a_q(f) f|B_q \in S_k(Nq, \epsilon) \subseteq  S_k(Nq^2, \epsilon)$. 
\end{proof}

The next proposition generalised the usual Sturm bound argument for modular forms. 

\begin{Prop}
Let $g_1$, $g_2$ be two normalised newforms of levels $N_1 \mid N_2$ and the same nybentypus character $\epsilon$. Assume $\epsilon$ has prime power conductor $Q = q^\beta$ such that  $Q^2 \mid N_1$. Let 
$B$ be the Sturm bound for the congruence subgroup $\Gamma_1(Nq^2)$. Suppose 
\[
	a_n(g_1) = a_n(g_2), \, \mbox{for all }1 \leq n \leq B \mbox{ such that } \gcd(n,q) = 1.
\]
Then $g_1 = g_2$. 
\end{Prop}

\begin{proof}
Following \cite{atkin1978twists}, we define the operator $K_q$ on the space of modular forms by
\[
	g|K_q = g - g|U_q|B_q. 
\]
Then the assumption is equivalent to the statement that $\delta  = (g_1 -g_2) |K_q$ has $a_n(\delta) = 0$ for all $1 \leq n \leq B$. Since $\delta \in S_k(Nq^2, \epsilon)$,  Sturm's theorem implies $\delta = 0$. We then know from \cite[Theorem 5.7.1]{diamond2006first}  that $g_1 - g_2 \in S_k(N_2,\epsilon)^{old}$. Suppose $N_1 < N_2$, then $g_1$ is in the old subspace, hence so is $g_2$, a contradiction. Therefore we must have $N_1 = N_2$. It follows that $g_1 - g_2 \in S_k(N_2, \epsilon)^{new}$, since $g_1, g_2$ are newforms. Since the new subspace and the old subspace intersect trivially, we must have $g_1 - g_2  = 0$. 
\end{proof}

Now we are ready to describe the algorithm. 

\begin{algorithm}[H]
\caption{Identifying  $f \otimes \chi$}
\label{alg: twist}
\begin{algorithmic}[1]
    \Require $k$ -- a positive even integer; $f \in S_k(\Gamma_0(N))$ a normalized newform; $\chi$ a Dirichlet character of prime power conductor $Q = q^\beta$; $Q^2 \mid N$;  $B$ -- a positive integer
    \Ensure The level $M_\chi$ of $f \otimes \chi$ and the Fourier expansion of $f \otimes \chi$ up to $q^B$.
    \If{$Q = 1$}
    \State return $N$.
    \EndIf
    \State $Q' := \cond(\chi^2)$; $N_0 := \frac{N}{q^{v_q(N)}}$; $M_0 := Q'N_0$; $t := \frac{N}{M_0} \in \bZ$. 
    \For{each positive divisor $d$ of $t$}
    	\State Set $V_d := S_k(M_0d, \chi^2)$. 
	\State Compute a basis of newforms $\{g_1^{(d)}, \cdots g_{s_d}^{(d)}\}$ of $V_d$.
    	\State Set $B_d$ := the Sturm bound for $\Gamma_1(M_0dq^2)$. 
    	\For{$1 \leq j \leq s_d$} 
		\If{$a_n(g_i^{(d)})= a_n(f)\chi(n)$ for all $1 \leq n \leq B_d, \gcd(n,q) = 1$}
			\State \Return $M_0d$.
		\EndIf
	\EndFor
    \EndFor	
\end{algorithmic}
\end{algorithm}

%It is natural to define {\it $p$-minimality} of newforms. The definition mimics that of [Brunault]. 

%\begin{Definition}
%Let $f \in S_k(\Gamma_1(N))$ be a newform. Let $p$ be a prime such that $p^2 \mid N$. 
%We say $f$ is {\it $p$-minimal} if $f \otimes \chi$ is new of level $N$ for all Dirichlet character 
%$\chi: (\bZ/p^{\frac{v_p(N)}{2}}\bZ)^{\times} \to \bC^{\times}$.
%\end{Definition}

We give some sample computations applying the above algorithm. 

\begin{Example}
Let $f$ be the normalised newform attached to the elliptic curve 
\[
	E: y^2 + x y + y = x^{3} -  x - 2
\]
of Cremona label {\bf 50a}. Then $f \otimes \chi$ is new of level 50 for all Dirichlet characters $\chi$ with modulus 5. 
In other words, $f$ is 5-minimal. 
\end{Example}

As another example, we demonstrate a newform which is not $p$-minimal. 
\begin{Example}
Let $f$ be the normalised newform attached to the elliptic curve 
\[
E: y^2 + x y = x^{3} + x^{2} - 25 x - 111
\]
of label {\bf 98a}. Let $\chi$ be the Dirichlet character modulo 7 defined by $\chi(3 \pmod{7}) = -1$. 
We found that $f \otimes \chi$ is a newform of level 14, with $q$-expansion
\[
 (f \otimes \chi) (q) = q - q^{2} - 2q^{3} + q^{4} + 2q^{6} + q^{7} - q^{8} + q^{9} - 2q^{12} - 4q^{13} - q^{14} + O(q^{15}).
\]
\end{Example}


\section{Pseudo-eigenvalues} 

Let $\epsilon$ be a Dirichlet character modulo $N$ and let $f$ be a newform in $S_k(N,\epsilon)$. For any divisor $Q$ of $N$ with $\gcd(Q, \frac{N}{Q}) =1$, there is an algebraic number  $w_Q(f)$ of absolute value one  and  a newform $g$ in $S_k(N, \overline{\epsilon_Q} \epsilon_{N/Q})$ such that 
\[
	W_Q(f) = w_Q(f) g, 
\]


\begin{Definition}
The number $w_Q(f)$ is called the {\it pseudo-eigenvalue} of $W_Q$ on $f$. 
\end{Definition}
For ease of notations, we write $w(f) = w_N(f)$. 

For a power series  $f = \sum_{n \geq 0} a_nq^n$, its complex conjugate, denoted by $f^*$, is $$f^*(q) = \sum \overline{a_n}q^n.$$ 

From \cite{atkin1978twists} we have $W_N(f) = w(f) f^*$. In the rest of this section, we describe an algorithm to efficiently compute $w(f)$ numerically.  For a positive even integer $k$, let $\bM(k)$ denote the space of weight-$k$ modular symbols defined in \cite{stein2007modular}. The space $\bM(k)$ is a quotient of $\bZ[X,Y]_{k-2} \otimes \bP^1(\bQ)^2$, and $GL_2(\bQ)$ acts on $\bM(k)$ via the following rule
\[
	g (P(X,Y) \otimes \{\alpha, \beta\}) = P( g^{-1}(X,Y)^T) \{g(\alpha), g(\beta)\}.
\]
Most importantly, there is a pairing between $\bM(k)$ and the space of modular forms of weight $k$, defined as
\[
		\langle f, P(X,Y) \otimes \{\alpha, \beta\}) \rangle_k = \int_{\alpha}^{\beta} f(z) P(z,1) dz.
\]
We will suppress the subscript $k$ if its value is clear from context.


\begin{Lemma}
Let $M \in \bM(k)$ and $f \in S_k(\Gamma_1(N))$. Then 
\[
	N^{\frac{k}{2}-1} \langle f|W_N, M \rangle = \langle f, W_N M \rangle.
\]
\end{Lemma}

\begin{proof}
See proof of  \cite[Proposition 8.17]{stein2007modular}. Note that the extra factor $N^{\frac{k}{2}-1}$ is due to the different constants involved in the definition of the weight-$k$ action of $GL_2(\bQ)$ on modular forms. 
\end{proof}

The map  
\[
	*: P(x,y)\{\alpha, \beta\} \mapsto P(-x,y) \{-\bar{\alpha}, -\bar{\beta}\} 
\]
defines the {\it star involution} on the space $\bM(k)$. We have $\langle f^*, M \rangle = \overline{\langle f, M^*\rangle}.$


\begin{Lemma}
Let $f$ be a normalised newform on $\Gamma_1(N)$ with positive even weight $k$ and let $M \in \bM(k)$ be such that $W_N(M) = N^{k/2 -1} M^*$. Assume $\langle f, M \rangle \neq 0$.  Then 
\[
	w(f) = \frac{\langle f,M \rangle }{\overline{\langle f,M \rangle}}.
\]
\end{Lemma} 

\begin{proof}
Since $W_N^2(M) = N^{k-2}M$ for all $M \in \bM(k)$, the assumption implies $W_N(M^*) = N^{k/2 -1} M$. 
Now 
\begin{align*}
& N^{k/2-1} \langle f|W_N, M^* \rangle  = \langle f,W_N(M^*) \rangle \\
\implies & N^{k/2-1} w(f) \langle   f^*,M^* \rangle = N^{k/2-1} \langle f, M \rangle \\
\implies & w(f) = \frac{\langle f, M \rangle}{\langle f^*,M^* \rangle} \\ 
\implies &	w(f) = \frac{\langle f, M \rangle}{\overline{\langle f, M \rangle}}.
\end{align*}
\end{proof}

Suppose $\alpha, \beta$ are distinct points on the arc $\{z \in \bC | Im(z) > 0,  |z| = 1/\sqrt{N}\}$. Then it is easy to verify that $M = (xy)^{k/2-1} \otimes \{\alpha, \beta\}$ satisfies $W_N(M) = M^*$. Finally, we arrive at the algorithm to compute $w(f)$.




\begin{algorithm}[H] 
\caption{Computing the pseudo-eigenvalue of newforms.}
\label{alg: pseudo-eigenvalue}
\begin{algorithmic}[1]
    \Require $k$ -- a positive even integer. $f \in S_k(\Gamma_1(N))$ a normalized newform.    
    \Ensure a numerical approximation of $w(f)$.
    \State $n_0 := 10$, $z_0 := \frac{i}{\sqrt{N}}$. $\delta = 10^{-3}$. 
    \State Randomly generate $n_0$ points $\{z_1, \cdots, z_{N_0}\} \subseteq \{z | 0 < Im(z) < \frac{1}{2\sqrt{N}}, |z| = \frac{1}{\sqrt{N}} \}$.
    \For{$1 \leq i \leq n_0$}
    	\State compute the period integral $c_i =  \int_{z_0}^{z_i} 2\pi i f(z) z^{\frac{k-2}{2}} dz$. 
	\State $w_i \gets c_i/\bar{c_i}$. 
    \EndFor
    \If{the standard deviation of $w_1, \cdots, w_{n_0}$ is less than $\delta$} 
     \State $w \gets \frac{1}{n_0}(\sum_i w_i)$. 
     \State \Return $w$.
    \Else
    	\State \Return {\bf FAIL}. 
    \EndIf
\end{algorithmic}
\end{algorithm}

%\begin{Remark}
%The period integral in step 4 of Algorithm~\ref{alg: pseudo-eigenvalue} is computed as follows: 
%This approach is taken from [Cre97].
%\end{Remark}


\section{Formula for the Fourier expansion of $f$ at width one cusps: Part 1}

First we recall some notations from \cite{atkin1978twists}.
\begin{Definition}
For a positive integer $c'$, let $S_c' = \abcd{1}{\frac{1}{c'}}{0}{1}$. If $\chi$ is a character modulo $c'$, we define the 
operator on modular forms 
\begin{equation*}
\label{formula: RS}
	f | R_\chi(c') = \sum_{u =0}^{c'-1} \bar{\chi}(u) f | S_{c'}^u.
\end{equation*}
\end{Definition}
Write $R_\chi$ in short for $R_\chi(\cond(\chi))$. Note that $f|R_\chi = g(\bar{\chi})f_\chi$.
Conversely, if $(a,M) = 1$, we have 
\begin{equation}
\label{formula: SR}
	\phi(c')S_{c'}^u = \sum_{\chi: cond(\chi) \mid c'} \chi(u) R_\chi(c').
\end{equation}

For our convenience, we define some operators, which are essentially the conjugates of $S_c'$ and $R_\chi(c')$ by $W_N$. Let $A_c' = \abcd{1}{0}{c'}{1}$.  Then it is easy to verify the following matrix identity.
\begin{Fact}
$-N \cdot A_{N/c'}^{-1} = W_N S_{c'} W_N$.
\end{Fact}

From now on, we assume $c$ is a divisor of $N$ and $c' = \frac{N}{c}$. Then as operators on modular forms, 
\[
	A_c^{-1} = W_N S_{c'} W_N.
\]

Since $W_N^2 = id$ as operators, we have
\[
	A_c^{-u} = W_N S_{c'}^u W_N, \, \forall u \in \bZ. 
\]

Parallel to the notion of $R_\chi(c')$, let $\Phi_\chi(c) = \sum_{u =0}^{c'-1} \bar{\chi}(u) A_c^{-u}$.Then $\Phi_\chi(c) = W_N R_\chi(c') W_N$. 
Similar to Formula~\ref{formula: SR}, we have
\begin{equation}
\label{formula}
	\varphi(c') A_c^{-a} = \sum_{\cond(\chi) \mid c'} \chi(a) \Phi_\chi(c) 
	 = \sum_{\cond(\chi) \mid c'} \chi(a) W_N R_\chi(c') W_N. 
\end{equation}

Applying Formula~\ref{formula} to $f$, we arrive at
\begin{eqnarray} \label{expansion0}
	f_{[\frac{a}{c}]} \left( q\right) &= \frac{1}{\varphi(c')}\sum_{\cond(\chi) \mid c'} \chi(-a) f| W_N R_\chi(c') W_N. \\ 
	&= \frac{w(f) }{\varphi(c')}\sum_{\cond(\chi) \mid c'} \chi(-a) f| R_\chi(c')  W_N. 
\end{eqnarray}

Now it left to compute the expansions of  each $f| R_\chi(c') W_N$ in the sum.

\section{Formula for the Fourier expansion of $f$ at width one cusps: Part 2}

In this section, we describe how to compute the expansion of $f| R_\chi(c') W_N$. First note that $T_p = U_p  + \epsilon(p) p^{\frac{k}{2}}B_p$ as operators on $S_k(\Gamma_1(N), \epsilon)$. It follows that $T_p$ commutes with $B_d$ for any positive integer $d$. 
 

We recall some notations and a result from \cite{delaunay2002thesis}.
\begin{Definition}\cite[Definition III.2.4]{delaunay2002thesis}
For a Dirichlet character $\chi$ modulo $b = \prod_{j \in J} p_j^{\alpha_j}$. Let $r = |J|$. Decompose $\chi$ uniquely as 
$\chi = \chi_1 \cdots \chi_r$, where $\chi_i$ is a character modulo $p_j^{\alpha_j}$.  We define $\cond'(\chi)$ multiplicatively, by putting 
\begin{equation} \label{modified conductor}
	\cond'(\chi_j) = \begin{cases} \cond(\chi_j) & if \cond(\chi_j) > 1 \\ p_j & else \end{cases}
\end{equation}
Also, if $I = \{j \in J : \chi_j \mbox{ is trivial character modulo } p_j^{\alpha_j}\}$, we put $tr = \prod_{j \in I} p_j^{\alpha_j} $
$nt = b/tr$,  $\chi_{tr} = \prod_{j \in I} \chi_j$, and $\chi_{nt} = \chi/\chi_{tr}$. Then we set
\begin{equation} \label{modified gauss sum}
	g'(\chi) = (-1)^{|I|} \chi_{nt}(tr) g(\chi_{nt}). 
\end{equation}
Here $g(\chi)$ is the usual Gauss sum of $\chi$: if $\chi$ is a character modulo $d$, then $g(\chi) = \sum_{a=1}^{d} e^{\frac{2\pi i a}{d}} \chi(a)$. If $\chi = \chi_0$ is the trivial character, we set $g(\chi_0) = 0$. 
\end{Definition}


\begin{Lemma}\cite[Prop 2.6]{delaunay2002thesis} \label{lemma of delaunay}
Let $c'$ be an integer such that $c'^2 \mid N$. For a Dirichlet character $\chi$ mod $c'$, we have
$$f|R_{\chi}(c') = \begin{cases} g'(\bar{\chi}) f_{\chi_{nt}} & if \cond'(\chi) = c' \\ 0 & else.  \end{cases}$$
\end{Lemma}

Using this lemma, we can simplify formula \ref{expansion0} to 
\begin{equation} \label{expansion01}
	f_{[\frac{a}{c}]}= \frac{w(f) }{\varphi(c')}\sum_{\cond'(\chi) = c'} \chi(-a)g'(\bar{\chi}) f_{\chi_{nt}} | W_N. 
\end{equation}


Next, we compute $f_{\chi_{nt}}$ by the following: suppose $g  = f \otimes \chi_{nt}$. Then 
\begin{equation} \label{twistformula1}
	f_{\chi_{nt}} =  g |\prod_{i=1}^r K_{p_i}.
\end{equation}
Moreover, we have 
\begin{equation} \label{twistformula2}
K_{p} = 1  - U_{p} B_p =  \begin{cases} 1- (T_p - \chi_{nt}^2(p) p^{\frac{k}{2}} B_p) |B_p & p \nmid M \\  1 - T_p |B_p & p \mid M \end{cases}.
\end{equation} Using the commutativity of $T_*$ and $B_*$, we can write  $f_{\chi_{nt}}$ in the form $\sum c_i (f \otimes \chi)(q^{d_i})$, where $c_i$ and $d_i$ are constants. To give a precise formula, we use the following notation. For a finite set $S$ of integers, let $\pi(S) = \prod_{s \in S} s$ denote the product of all elements in $S$. For a Dirichlet character $\chi$ of conductor $d$, let $S_\chi$ be the set of prime divisors of $d$. For any positive integer $M$ and any finite set of integers $S$, define 

\begin{equation} \label{index set}
\cB_{S,M} = \{(S_1, S_2) \in (2^\bZ)^2 | S_1, S_2 \subseteq S, S_1 \cap S_2 = \emptyset, \gcd(M, \pi(S_2)) = 1\}
\end{equation} 

\begin{Prop} \label{thm: formula1}
Let $k \geq 2$ be an even integer and let $f$ be a newform in $S_k(\Gamma_0(N))$. Then 
	$$f_{\chi_{nt}} =  \sum_{(S_1, S_2) \in \cB_{S_\chi,M}} (-1)^{|S_1|}a_{\pi(S_1)}(g_\chi)  \pi(S_2)^{k/2} \chi_{nt}^2(\pi(S_2)) g_\chi | B_{\pi(S_1) \pi(S_2)^2}.$$
Here $g_\chi = f \otimes \chi$, $M$ is the level of $g_\chi$ and $\cB_{S_\chi,M}$ is as in  \ref{index set}.
\end{Prop}

\begin{proof}
This is a direct consequence of multiplying out \ref{twistformula1} using \ref{twistformula2}, using the fact that $T_p$ commutes with $B_d$, and noting that $T_p$ acts as multiplication by $a_p(g_\chi)$ on $g_\chi$.
\end{proof}

Theorem~\ref{thm: formula1} will be our starting point of computing the expansion of $f$ at width one cusps. 
We will use it to compute $f_{\chi_{nt}} | W_N$.  First we prove two lemmas. 
\begin{Lemma} \label{lemma: bdwn}
Let $f$ be a newform of even weight $k$ on $\Gamma_1(M)$ and suppose $d, N$ are positive integers such that $Md \mid N$. Then
   $$f| B_d|W_N = \left(\frac{N}{Md^2} \right)^{k/2}  w(f)  (f|B_{\frac{N}{Md}})^{*}.$$
\end{Lemma}
\begin{proof} Straightforward computation.
\begin{align*}
f |B_d | W_N & = d^{-k/2} f | \abcd{d}{0}{0}{1} \abcd{0}{-1}{N}{0} \\ 
& = d^{-k/2} f| \abcd{0}{-1}{M}{0} \abcd{N/md}{0}{0}{1} \abcd{d}{0}{0}{d} \\
& = \left( \frac{N}{Md^2} \right)^{k/2} f | W_M | B_{N/Md} \\
& = \left( \frac{N}{Md^2} \right)^{k/2} w(f) f^* | B_{N/Md} \\ 
& = \left( \frac{N}{Md^2} \right)^{k/2} w(f) (f | B_{N/Md})^*.
\end{align*}
\end{proof}

Before stating the second lemma, we quote another result in \cite{li1975newforms} on the coefficients of a newform at primes dividing the level.
\begin{Lemma}\cite[Theorem 3 (iii)]{li1975newforms}  \label{lemma: winnie-vanishing}
Let $f = \sum_{n \geq 1} a_n(f) q^n$ be a normalized newform in $S_k(\Gamma_1(N), \epsilon)$ and let $p$ be a prime dividing $N$.  Then \\
(1) If $\epsilon$ is a character modulo $N/p$ and  $p^2 \mid N$, then $a_p(f) = 0$. \\ 
(2) If $\epsilon$ is a character modulo $N/p$ and  $p^2 \nmid N$, then $a_p(f)^2 = \epsilon(p) p^{k-2}$. \\ 
(3) If $\epsilon$ is not a character modulo $N/p$, then $|a_p(f)| = p^{\frac{k-1}{2}}$.
\end{Lemma}


\begin{Lemma} \label{lemma: well-definedness}
Keep the notations in Proposition~\ref{thm: formula1}. If $(S_1, S_2) \in \cB_{S_\chi,M_\chi}$ is such that $a_{\pi(S_1)}(g_\chi) \neq 0$. Then 
$M \pi (S_1) \pi(S_2)^2 \mid N$.
\end{Lemma}

\begin{proof}
Let $p$ be a prime divisor of $N' := M \pi(S_1) \pi(S_2)^2$. If $p \nmid M$, then $\ord_p(N') \leq \ord_p(\cond(\chi)^2) \leq \ord_p(N)$. So we assume $p \mid M$, hence $p \nmid p(S_2)$. If $p \nmid p(S_1)$, then there's nothing to prove; 
if $p \mid \pi(S_1)$, we want to show that $\ord_p(M) < \ord_p(N)$. Suppose not, then $\ord_p(M) = \ord_p(N) \geq 2 \ord_p(\cond(\chi))$. Since $\cond(\chi^2) \leq \cond(\chi)$, we know $\chi^2$ is a character modulo $M/p$. Applying case (1) of Lemma~\ref{lemma: winnie-vanishing} to the newform $g_\chi$, we see that $a_{p}(g_\chi) = 0$, hence $a_{\pi(S_1)}(g_\chi) = 0$ by multiplicativity.
\end{proof}

Now we can state our main theorem from this chapter.
% Main Theorem of expansion 

\begin{theorem} \label{thm: ExpansionFormula}
Let $k \geq 2$ be an even integer and let $f$ be a normalized newform in $S_k(\Gamma_0(N))$. Let $z$ be a cusp on $X_0(N)$ of width one. Write $z = \left[ \frac{a}{d} \right]$ such that $\gcd(a,d) = 1$, $d \mid N$ and $N \mid d^2$. Let $d' = \frac{N}{d}$. Then the Fourier expansion of $f$  at the cusp $z$ is 
\[
	f_z(q) = \frac{w(f) }{\varphi(d')}\sum_{\chi: \cond'(\chi) = d'} \chi(-a) g'(\bar{\chi}) w(f \otimes \chi) f_\chi^{!}(q).
\]
Here

\begin{itemize}

\item $w(f)$ and $w(f \otimes \chi)$ are the pseudo-eigenvalues. 
\item $g'(\chi)$ is the modified Gauss sum defined in \ref{modified gauss sum} . 
\item $\cond'$ is the modified conductor of a Dirichlet character in \ref{modified conductor}. 
\item $f_\chi^{!}$ is as follows: let $M_\chi$ denote the level of  $f \otimes \chi$. Then
\[
f_\chi^{!} = \sum_{(S_1, S_2) \in \cB_{S_{\chi_{nt}},M_\chi}} (-1)^{|S_1|}a_{\pi(S_1)}(f \otimes \chi)  \left(\frac{N}{M_\chi \pi(S_1)^2 \pi(S_2)^3} \right)^{k/2}\chi^2(\pi(S_2)) (f \otimes \chi | B_{\frac{N}{M_\chi \pi (S_1) \pi(S_2)^2}})^*
\]
where the notations follow \ref{thm: formula1}.
\end{itemize}
\end{theorem}

\begin{proof}
We start from formula \ref{expansion01}:
$$f_{[\frac{a}{c}]}= \frac{w(f) }{\varphi(c')}\sum_{\cond'(\chi) = c'} \chi(-a)g'(\bar{\chi}) f_{\chi_{nt}} | W_N.$$
From \ref{thm: formula1}, we have 
	$$f_{\chi_{nt}} =  \sum_{(S_1, S_2) \in \cB_{S_\chi,M_\chi}} (-1)^{|S_1|} a_{\pi(S_1)}(f \otimes \chi)  \pi(S_2)^{k/2} \chi_{nt}^2(\pi(S_2)) f \otimes \chi | B_{\pi(S_1) \pi(S_2)^2}.$$
	To simplify notations, let $c(f, \chi, S_1, S_2) = (-1)^{|S_1|} a_{\pi(S_1)}(f \otimes \chi)  \pi(S_2)^{k/2} \chi_{nt}^2(\pi(S_2))$. Then 
\begin{align*}
	f_{\chi_{nt}} | W_N &= \sum_{(S_1, S_2) \in \cB_{S_\chi,M_\chi}} c(f, \chi, S_1, S_2) f \otimes \chi | B_{\pi(S_1) \pi(S_2)^2} W_N  \\
& =  \sum_{(S_1, S_2) \in \cB_{S_\chi,M_\chi}} c(f, \chi, S_1, S_2) \left(\frac{N}{M_\chi (\pi(S_1) \pi(S_2)^2)^2} \right)^{k/2} w(f\otimes \chi) (f \otimes \chi | B_{\frac{N}{M_\chi \pi (S_1) \pi(S_2)^2}})^* \\
& = w(f \otimes \chi) f_\chi^{!}. 
\end{align*}
Note that we applied Lemma~\ref{lemma: bdwn} to obtain the penultimate equality, and we could do that because of Lemma~\ref{lemma: well-definedness}. Now the result follows. 
\end{proof}



Theorem~\ref{thm: ExpansionFormula} gives us an algorithm to compute the expansion of $f_z$, which we will describe below. But first, we take a closer look at what ingredients goes into the expansion. Given a newform $f \in S_k(\Gamma_0(N))$ and a width one cusp $z$ of denominator $c$. We need to consider the twist of $f$ by all Dirichlet characters of conductor dividing $c$. 
For each such character $\chi$, we then need to determine the level $M_\chi$ and $q$-expansion of the newform $f \otimes \chi$, the latter boils down to knowing $a_p(f \otimes \chi)$ for all primes $p \mid \cond(\chi)$. Then we need to compute the pseudo-eigenvalues of $f \otimes \chi$. Finally, we combine these information together and apply Throem~\ref{thm: ExpansionFormula} to compute $f_z$. 

\begin{algorithm}[H]
\caption{Computing Fourier coefficients of $f$ at width one cusps}
\label{alg: qexp}
\begin{algorithmic}[1]
    \Require $f \in S_k(\Gamma_0(N))$ a newform; $a, c$ -- coprime integers such that $N \mid c^2$; $B$ -- a positive integer. 
    \Ensure The first $B$ Fourier coefficients of $f_{\left[\frac{a}{c} \right]}(q)$. 
    
    \State  $c' \gets N/c$. $X \gets$ The set of all Dirichlet characters $\chi$ such that $\cond'(\chi) = c'$. 
    \State compute $w(f)$  using Algorithm~\ref{alg: pseudo-eigenvalue}. 
    \For{$\chi$ in $X$}   
    	\State Using  Algorithm~\ref{alg: twist}, compute the level $M_\chi$ and the $q$-expansion of $g_\chi := f \otimes \chi$ to $B$ terms.
	\State Compute $w(g_\chi)$ using Algorithm~\ref{alg: pseudo-eigenvalue}.
    \EndFor
    \State Apply Theorem~\ref{thm: ExpansionFormula} to compute $f_z$ to $B$ terms. 
    \end{algorithmic}
\end{algorithm}

\section{A Converse Theorem}

Given the work in previous sections, it is a natural question then to ask whether the information on twists of $f$ is uniquely determined by the expansion of $f$ at width one cusps. The answer is yes, and the precise statement is in the following theorem. 

\begin{theorem}
Let $f$ be a normalized newform in $S_k(\Gamma_0(N))$. Assume the eigenvalue $w_N(f)$ is known. Suppose $c$ is a positive divisor of $N$ such that $N \mid c^2$.  Then the expansions of $f_z$, where $z$ runs through all cusps of denominator $c$, uniquely determines the following: for each Dirichlet character 
$\chi$ of such that $\cond'(\chi) = c'$, the level $M_\chi$, the pseudo-eigenvalue $w_{M_\chi}$ and the $q$-expansion of the newform $f \otimes \chi$. 
\end{theorem}

\begin{proof}
By plug in different $a$'s. We can solve for $t_\chi$.  Consider the first nonzero term of $t_\chi$. Suppose 
\[
	t_\chi = u_\chi q^{v_\chi} + O(q^{v_\chi + 1}), \, u_\chi \neq 0.
\]
Assuming that $\chi$ has prime power conductor $p^\beta > 1$, we claim that 
$$\left| \frac{v^{k/2}}{u} \right| = \begin{cases} p^{k/2} & if p \nmid M_\chi \\ p^{1/2} & if p \mid M_\chi \mbox{ and }  a_p(g) \neq 0 \\ 1 & else \end{cases}.$$

Proof of claim: the first and third case are easy to verify using Theorem~\ref{thm: ExpansionFormula}. Now assume $p \mid M$ and $a_p(g_\chi) \neq 0$. By Lemma~\ref{lemma: winnie-vanishing}, we have $|a_p(g_\chi)| = p^{k/2 - 1/2}$ or $p^{k/2 - 1}$. However, $|a_p(g_\chi)| = p^{k/2-1}$ only if $p \mid \mid M_\chi$ and $\chi^2$ is a character modulo $M_\chi/p$. This means $\chi^2$ is the trivial character. By Lemma~\ref{lemma: winnie-level}, we compute the $p$-level of $f = g_\chi \otimes \bar{\chi}$: note that $\max{p, p^{\alpha+\beta}, p^{2\beta}} > p$, so (ii) applies and the $p$-level of $f$ is equal to $\max(p^{\alpha}, p^{\beta})  = p^{\beta}$, i.e., $\ord_p(N) = \beta$. This is impossible since we have $p^{2 \beta} = \cond(\chi)^2 \mid N$. 

Therefore, we have $|a_p(g_\chi)| = p^{k/2-1/2}$ and the claim follows.

Since $k \geq 2$, we could determine which case we are in. Then we can read off $M_\chi$  and $w_M(g_\chi)$. For example, if we are in the second case, then the level can be computed via $M_\chi = \frac{N}{v_\chi p}$.  Now  the $N/M_\chi$'s coefficient of $t_\chi$ is 
\begin{eqnarray*}
	a_{\frac{N}{M}}(t_\chi) &= w(g_\chi) (\frac{N}{M})^{k/2} ( 1 -  |a_p(g_\chi)|^2 \chi^2(p) p^{-k/2} )  \\
	& = w(g_\chi) (\frac{N}{M})^{k/2} ( 1 - p^{k/2 -1} \chi^2(p)). 
\end{eqnarray*}

This allows us to solve $w(g_\chi)$. Finally, we compute $a_p(g_\chi)$ by $a_p(g) = \frac{-u_\chi}{w(g_\chi) \chi^2(p) (\frac{N}{Mp})^{k/2}}$. The value $a_p(g)$ determines the expansion of $g_\chi$. Recursively, we could solve for all $pn$-coefficients of $g_\chi$, from which we deduce it complete $q$-expansion. 


In the general case,  we consider the following subsets of $S_\chi$.  Let $S_1^* = \{ p \in S_\chi: p \mid M \}$, $S_2^* = 
S_\chi \setminus S_1^*$, and $\widetilde{S_1^*}= \{p \in S_1^*: a_p(g_\chi) \neq 0\}$.

It follows that the leading term of $t_\chi$ belongs to the summand corresponding to $(\widetilde{S_1^*}, S_2^*)$ in  Theorem~\ref{thm: ExpansionFormula}. Still writing the leading term as $u_\chi q^{v_\chi}$, we have 
\[
	u_\chi = w(g_\chi) \chi^2(p(S_2)) a_{p(\widetilde{S_1^*})}(g_\chi) p(\widetilde{S_1^*})^{-k} (p(S_2^*)^{-3k/2} \left(\frac{N}{M_\chi}\right)^{k/2}, v_\chi = \frac{N}{M_\chi p(\widetilde{S_1^*}) p(S_2^*)^2}. 
\]
Similar to the prime power conductor case above, we have $|a_{p(\widetilde{S_1^*})}(g_\chi)|  = p(\widetilde{S_1^*})^{k/2 -1/2}$.  So 
\begin{equation} \label{formula: converse}
	|v_\chi^{k} u_\chi^{-2}| = p(\widetilde{S_1^*}) p(S_2^*)^2.
\end{equation}
Hence we can factor $|v_\chi^{k} u_\chi^{-2}|$ and obtain $p(\widetilde{S_1^*})$
and $p(S_2^*)$. Then $M_\chi$ can be solved using $v_\chi$. Plug it back into $u_\chi$, we obtain $a_{p(\widetilde{S_1^*})} w(g_\chi)$. Finally, for each $p \in \widetilde{S_1^*}$,  the $v_\chi p$'s coefficient of 
$t_\chi$ allows us to compute $a_{p(\widetilde{S_1^*})/p}(g_\chi) w(g_\chi)$. These together determines $w(g_\chi)$ and 
$a_{p(\widetilde{S_1^*})}$. The other Fourier coefficients of $g_\chi$ can then be computed recursively. 
\end{proof}




\section{Field of definition}

In the previous sections, we have described an algorithm to compute the Fourier coefficients of $f_z$ as complex numbers. In fact, the Fourier coefficients are algebraic numbers. More precisely, if $d$ is the denominator of  $z$ and $d' = N/d$, then $f_z (q) \in K_f(\zeta_{d'})[[q]]$. Here $K_f$ is the number field generated by the Fourier coefficients of $f$ (at the cusp $\infty$). In this section, we provide a proof of this fact. 

\begin{Lemma}[\cite{stevens2012arithmetic}] \label{stevens}
(1) The cusps of $X_0(N)$ are rational over the field $\bQ(\zeta_N)$. \\
(2) For $s \in (\bZ/N\bZ)^*$, let $\tau_s \in Gal(\bQ(\zeta_N)/\bQ)$ be defined by $\tau_s(\zeta_N) = \zeta_N^s$. 
Then 
\[
	\tau_s \left(\left[\frac{a}{y}\right]\right) = \left[\frac{a}{ys'} \right], 
\]
where $s' \in \bZ$ is chosen so that $ss' \equiv 1 \mod{N}$. 
\end{Lemma}

Using the lemma above, we can obtain a precise description of cusps of the same denominator $d$. We summarize 
the facts in the following proposition. 
\begin{Prop}
Let $d$ be a positive divisor of $N$ and let $d' = N/d$. Then \\
(1) The cusps of denominator $d$ on $X_0(N)$ are defined over the field $\bQ(\zeta_{d'})$. 

(2) Let $\tau_s \in Gal(\bQ(\zeta_{d'}/\bQ))$ be the map $\tau_s: \zeta_{d'} \to \zeta_{d'}^s$. Then 
\[
	\tau_s \left([\frac{a}{d}]\right) = \left[\frac{a}{ds'} \right], 
\]
where $s' \in \bZ$ is chosen so that $ss' \equiv 1 \mod{d'}$. 
\end{Prop}

\begin{proof}
From part (2) of Lemma~\ref{stevens}, we see that if $c$ is a cusp of denominator $d$ and $s \equiv1 \mod{d'}$, then $\tau_s(c) = c$.  The claims now follow directly from this observation.
\end{proof}

\begin{Prop} \label{field of definition}
We have \\
(1)	$\bQ(\{a_n(f_c)\}) \subseteq \bQ( \{a_n(f)\}, \zeta_{d'}).$ \\
(2) Let $\sigma \in Gal(\bar{\bQ}/\bQ)$  be such that $\sigma|_{\bQ(\zeta_{N})} = \tau_s$. Then 
\[
	(f_c)^{\sigma} = (f^{\sigma})_{\tau_s(c)}.
\]
\end{Prop}

\begin{proof}
First, (1) follows from (2), for if $\sigma$ fixes $\bQ( \{a_n(f)\}, \zeta_{d'})$, then $f^{\sigma} = \sigma$ and $\tau_s(z) = z$. To prove (2), let $g$ be a meromorphic modular form of weight $k$, level 1, and rational Fourier coefficients (for example, one can choose $g = (\dif j)^{k/2}$). Then it suffices to prove the claim for $f/g$, since $g|_k\gamma = g$ for all $\gamma \in SL_2(\bZ)$. Now $f/g$ is a rational function on $X_0(N)$. Since the function field of $X_0(N)$ is generated by $j$ and $j_N$, So we may write write $f/g =  P(j,j_N)/Q(j,j_N)$, where $P, Q \in K_f[x,y]$. Now fix$ \gamma_c \in SL_2(\bZ)$ such that $\gamma_c(\infty) = c$. Since $j|\gamma_c = j$, it suffices to prove the claim for $j_N$. 
WLOG, we can assume $\gamma_c = \abcd{1}{ld}{0}{1}$, where $d  = d_z$ and $\gcd(l,N) = 1$. Then 
$j_N | \gamma_c = j(N \gamma_c(z)) = j( \frac{dz+l'}{d'})= \sum a_n(j) e^{2\pi i \frac{l'}{d'}}q^{nd/d'}$, where $l'$ is an integer such that $l'l \equiv 1 \mod{d'}$. Hence for $\sigma$ in (2), we 
have $(j_N | \gamma_c)^{\sigma} =  \sum a_n(j) e^{2\pi i \frac{l's}{d'}}q^{nd/d'}$. On the other hand, we compute 
$(j_N^{\sigma})_{\tau_s(c)} = (j_N)_{\tau_s(c)} = j_N | \abcd{1}{lds'}{0}{1} = \sum a_n(j) e^{2\pi i \frac{l'}{d'}}q^{nd/d'}$. So
$(j_N | \gamma_c)^{\sigma} = (j_N^{\sigma})_{\tau_s(c)}$. Hence the same claim for $f$ holds. 
\end{proof}


%\begin{proof}
%Let $K_0 = \bQ( \{a_n(f)\}$
%Choose a form $0 \neq g \in S_k(\Gamma_1(N))$ with rational Fourier coefficients such that 
%$h = \frac{f}{g}$ is non-constant. From [Cox, ] it is easy to see that $h \in $. Then we have 
%\end{proof}
\section{Examples}

Let $E = {\bf 50a}$ and consider the 4 cusps of denominator 10 on $X_0(50)$. The corresponding first terms 
of $q$-expansions at these cusps are  
\begin{align*}
	a_1(f, \frac{1}{10}) &= \frac{1}{5} \zeta_{5}^{3} - \frac{3}{5} \zeta_{5}^{2} + \frac{3}{5} \zeta_{5} - \frac{1}{5} \\ 
	a_1(f, \frac{3}{10}) &= \frac{3}{5} \zeta_{5}^{3} + \frac{6}{5} \zeta_{5}^{2} + \frac{4}{5} \zeta_{5} + \frac{2}{5} \\
	a_1(f, \frac{7}{10}) &= \frac{2}{5} \zeta_{5}^{3} - \frac{1}{5} \zeta_{5}^{2} - \frac{4}{5} \zeta_{5} - \frac{2}{5}\\
	a_1(f, \frac{9}{10}) &=-\frac{6}{5} \zeta_{5}^{3} - \frac{2}{5} \zeta_{5}^{2} - \frac{3}{5} \zeta_{5} - \frac{4}{5}.
\end{align*}

As another examples, let $E = {\bf 98a}$ and $z = [\frac{1}{14}]$. We computed numerically that


{\small
\begin{align*}
f_c(q) = \left(-0.755001687308946 - 0.172324208281817i\right)q + \left(0.441471704846525 - 0.916725441095080i\right)q^{2} \\ 
+ \left(1.39294678431094 + 1.11083799261729i\right)q^{3} + \left(0.696473392155471 - 0.555418996308649i\right)q^{4} \\ 
+ \left(1.51000337461789 - 0.344648416563641i\right)q^{6} + \left(-3.80647894157196 \times 10^{-16} - 3.02371578407382i\right)q^{7} \\
+ \left(0.755001687308946 + 0.172324208281817i\right)q^{8} + \left(-0.441471704846525 + 0.916725441095080i\right)q^{9} +  \\ 
\left(-0.882943409693050 - 1.83345088219016i\right)q^{12} + \left(-3.02000674923578 + 0.689296833127282i\right)q^{13} \\
+ \left(3.80647894157196 \times 10^{-16} + 3.02371578407382i\right)q^{14} + O(q^{15})
\end{align*}
}



\section{Automorphic representations; norm of first terms}

References: \cite{bushnell2006local}, \cite{loeffler2010computation}. \cite{brunault2012ramification}. \cite{kraus1990defaut}.  \cite{jacquet1972automorphic}. 

In this section, we will restrict ourselves to the case when the Fourier coefficients of $f$ are rational numbers. Then 
$f$ induces an admissible reprensetation $\pi_f$ of $GL_2(\bA_\bQ)$. We will see that the expansion of $f$ at all cusps can also be computed from the local component $\pi_{f,p}$. Loeffler and Weinstein gave an algorithm to compute such 
local components. 

We will restrict ourselves to the simplest case when $f$ is twist-minimal, which means that the conductor of $\pi_f$ is the smallest among all  twists $\pi_{f \otimes \chi}$. 

We will follow the notations of David Loeffler and use the formula of  \cite{brunault2012ramification}. 

Let $z$ be a width one cusp of denominator $c$. Then the first coefficient $a_1(f_z)$ is an element in $K_f(\zeta_{c'})$. For simplicity, we assume that $c' = p^{\alpha}$ is a prime power. It can be proved using automorphic representations + local langlands correspondence that there exists $\beta$ such that $p^\beta a_1(f_z) \in \bar{\bZ}$.  One question is: what prime ideals appears in the prime factorisation of $(a_1(f,z))$?  It seems from our numerical data, that
\[
	\ord_\fq(a_1(f_z)) > 0 \implies \fq \cap \bZ \equiv \pm 1 \pmod {p}.
\]
The following is a table of data. 

\subsection{Cuspidal local constants}

We keep the assumptions that $f$ is a newform attached to an elliptic curve $E/\bQ$ and $f$ is twist-minimal. Assume 
$p$ is a prime dividing the conductor $N$ of $E$ such that $v_p(N) = 2$. Then there exists a character 
$\varphi: \bF_{p^2}^{\times} \to \bC^{\times}$ which determines $\pi_{f,p}$. We will prove 

\begin{Lemma} \label{cuspidal constant}
Let $\psi: \bQ_p \to \bC^{\times}$ be a character of level one (e.g. $\psi(x) = e(\{\frac{x}{p}\}_p)$). Then
\[
	\epsilon(\pi_{f,p}, 1/2, \psi)  = \frac{-1}{p} \sum_{x \in \bF_{p^2}^{\times}} \psi( x + x^p) \varphi(x). 
\]
If $\chi$ is a Dirichlet character such that the $f\otimes \chi$ has the same level as $f$. Then 
\[
	\epsilon(\pi_{f \otimes \chi,p}, 1/2, \psi)  = \frac{-1}{p} \sum_{x \in \bF_{p^2}^{\times}} \psi( x + x^p) \varphi(x) \bar{\chi}(x^{p+1}). 
\]
\end{Lemma}


\begin{proof}
By \cite{bushnell2006local}, taking $n =  r = 1$, we have 
\begin{equation} 
	p^2 \epsilon(\pi_{f,p}, 1/2, \psi) \cdot \text{id} = \sum_{x \in GL_2(\bF_p)} \psi(tr(x)) \pi_{f,p}^{\vee}(x).
\end{equation}
where $\pi_{f,p}^{\vee}$ denotes the contragredient representation. The representation $\pi_{f,p}$ has dimension $(p-1)$.
Taking traces, we obtain 
\begin{equation} \label{cuspidal const}
	p^2 (p-1) \epsilon(\pi_{f,p}, 1/2, \psi) \cdot \text{id} = \sum_{x \in GL_2(\bF_p)} \psi(tr(x)) Tr(\pi_{f,p}^{\vee}(x)).
\end{equation}
By assumption, $\pi_{f,p}$ arises from a cupsidal representation of the finite group $GL_2(\bF_p)$, which is in turn induced from $\varphi$. (See Fulton-Harris), we have formulae for $Tr(\pi_{f,p}^{\vee}(x))$. Splitting the sum corresponding to four types of conjugacy classes, we computed $S_1 =  (p-1)  \sum_{x \in \bF_p^{\times}} \psi(2x)$, $S_2 = (p^2-1) \sum_{x \in \bF_p^{\times}} \psi(2x) (-1)$, $S_3 = 0$, and $S_4=  (p^2-p)/2 \sum_{x \in \bF_{p^2} \setminus \bF_p} \psi(tr(x))(\overline{\varphi(x) + \varphi(x^p)})$. So the sum on the right hand side of \ref{cuspidal const}  equals $(p-p^2) \sum_{x \in \bF_{p^2}^{\times}} \psi(tr(x)) \overline{\varphi(x)}$. Dividing by $p^2(p-1)$ gives the formula.

\end{proof}

Moreover, since $E$ is defined over $\bQ$, the character of $\pi_{f,p}$ takes rational values. Hence the order of $\varphi$ is 3,4 or 6. The local Langlands correspondence claims that the order of $\varphi$ is equal to the order of the inertia subgroup of $Gal(L/\bQ)$, where $L$ is the smallest number field over which $E$ acquires good reduction \red{(to-do: check this)}. The case $p \geq 5$ is easy, as we have the following lemma: 

\begin{Lemma}\cite[Proposition 1]{kraus1990defaut}
Let $\Delta$ denote the minimal discriminant of $E$. Then for $p \geq 5$, the order of $\varphi$ is equal to 
$\frac{12}{\gcd(12, v_p(\Delta))}$.
\end{Lemma}

We remark that for $p = 2$ or $3$, the order of $\varphi$ can be determined using results of \cite{kraus1990defaut}. 



%, or Dokchitser's paper Euler factors determine local Weil representations. \cite{loeffler2010computation}

We remark that for elliptic curves, $v_2(N)$ is at most 8 and $v_3(N)$ is at most 5. For the sake of simplicity, we do not treat the case when $v_p(N) > 2$ here, but we point out the local constants can be  also computed from formula in \cite{bushnell2006local}, 
once the local component is determined using \cite{loeffler2010computation}. 

\begin{Example}
An example with trivial central character.  Let $f$ be the newform attached to $E = \bf{121a}$. Using Sage, we computed $w(f) = -1$. Since the weight of $f$ is 2, we know $\epsilon_\infty = -1$ (since the central character of $\pi_f$ is trivial, the level of the additive character $\psi_\infty$ does not matter). The discriminant of $E$ is $\Delta = -121$, so $\varphi$ has order 6. Using Lemma~\ref{cuspidal constant}, we computed that
$\epsilon_{11}(\pi_{f,11}, 1/2) = -1$. This verifies $w(f) = - \prod_{p \leq \infty} \epsilon_p$. 
\end{Example}

\begin{Example}
We give an example with nontrivial central character. Let $f$ be as in the previous example, and let $\chi$ be the Dirichlet character of $\bF_{11}^{\times}$ defined by $\chi(2) = e^{2\pi i /10}$. Lemma~\ref{cuspidal constant} gives 
\[
	\epsilon_{11}(\pi_{f \otimes \chi,11}, 1/2) = 0.64..+0.76..i
\]
an algebraic number with miminal polynomial $x^{20} + 109/121 x^15 + 2861/1331 x^10 + 109/121x^5 + 1$.
So $w = - \epsilon_{11} \epsilon_\infty = \epsilon_{11}$. Using the numerical Algorithm~\ref{alg: pseudo-eigenvalue},
we compute $w(f \otimes \chi) = 0.642573377564283 + 0.766224154177894i$. This confirms the computation. 
\end{Example}


\section{Norm of first terms computations}

We keep the assumptions from the previous section, that $f$ is a newform in $S_2(\Gamma_0(N))$, attached to an elliptic curve $E/\bQ$. We assume $f$ is twist-minimal and $p \geq 5$ is a prime dividing the conductor $N$ such that 
$v_p(N) = 2$.  In this case, the cusp $z_p = \left[\frac{-p}{N} \right]$ is of width one, and the $q$-expansion of $f$ at $z_p$
takes an especially simple form.  We summarize this in the lemma below. 
\begin{Lemma}
With the assumptions above, there exists a Galois-invariant set of numbers $\{b_1, \ldots, b_{p-1}\} \subseteq \bQ(\zeta_p)$, such that 
\[
	f_{z_p}(q) = \sum_{n \geq 1} a_n(f) b_{n\mod{p}} q^n. 
\]
More precisely, the $b_j$ are given by 
\[
 b_j = w(f) \sum_{\chi: \cond(\chi) = p} g(\bar{\chi}) w(f\otimes \chi) \chi(n)
\]
\end{Lemma}

\begin{proof}
First, the assumptions imply that $a_{n}(f) =  0$ if $p \mid n$. So the right hand side of the formula is well-defined. 
The formulae then follow directly from Theorem \ref{thm: ExpansionFormula}.   We have $b_j \in \bQ(\zeta_p)$ since the cusp $z_p$ is defined over $\bQ(\zeta_p)$, by Proposition~\ref{field of definition}.  Moreover, the cusps $\{z_p^{(j)} = \frac{-jp}{N}: 1 \leq j \leq p-1\}$ form a Galois orbit on $X_0(N)$, and one has 
\[
	a_n(f_{z_p^{(j)}}) = a_{jn}(f_{z_p}), \, \forall n \geq 1, 1 \leq j \leq p-1. 
\]
In particular, we have $\{b_j\} = \{a_1(f_{z_p^{(j)}})\}$. Since the latter set is Galois-invariant, so is the former. 
\end{proof}

We remark that it is clear from the formula of $b_j$ that they are algebraic number. However, the formula does not imply directly that they lie in $\bQ(\zeta_p)$. 

We give another formula of $a_1(f_{z_p})$ in light of the previous section. 

\begin{Lemma} \label{formula: first term}
Keeping the assumptions in the previous two sections, we have 
$$a_1(f_{z_p}) = \frac{\sum_{x \in \bF_{p^2}^{\times}} \psi(x+x^p+x^{p+1}) \varphi(x)}{\sum_{x \in \bF_{p^2}^{\times} }\psi(x+x^p) \varphi(x)}. $$
\end{Lemma}

\begin{proof} 
In this special case, formula [] simplifies to 
\begin{align*}
	a_1(f_{z_p}) &= w(f) \sum_{\chi: \cond(\chi) = p}  g(\bar{\chi}) w(f \otimes \chi) \\
	& = \sum_{\chi: \cond(\chi) = p}  g(\bar{\chi}) w(f \otimes \chi) w(f)^{-1} \, (\text{since } w(f)^2 =1) \\
	& =  \sum_{\chi: \cond(\chi) = p}  g(\bar{\chi}) \frac{\epsilon_p(\pi_{f \otimes \chi}, 1/2, \psi)}{\epsilon_p(\pi_f,1/2,\psi)}.
\end{align*}

We explain the last equality: first we have $w(f) =  \prod_{l \leq \infty} \epsilon_l(\pi_{f,l}, 1/2, \psi)$ and $w(f \otimes \chi) =  \prod_{l \leq \infty} \epsilon_l(\pi_{f \otimes \chi, l}, 1/2, \psi)$. Since $\chi$ has conductor $p$, we know the epsilon factors are the same except for $l = p$. Hence $\frac{w(f \otimes \chi)}{w(f)} = \frac{\epsilon_p(\pi_{f \otimes \chi}, 1/2, \psi)}{\epsilon_p(\pi_f,1/2,\psi)}$. 

Now by a formula in \cite{brunaultnotet}, we have  $$\sum_{\chi: \cond(\chi) = p}  g(\bar{\chi})\epsilon_p(\pi_{f \otimes \chi}, 1/2, \psi) = \frac{-1}{p}\sum_{x \in \bF_{p^2}^{\times}} \psi(x+x^p+x^{p+1}) \varphi(x).$$
This combined with our Lemma~\ref{cuspidal constant} gives the result. 
\end{proof}

\begin{Example}
Let $f$ be the newform attached to $E  =\bf{49a}$. One checks that $f$ is twist-minimal and $\varphi$ has order 4. Using Lemma~\ref{formula: first term}, we computed $$a_1(f_{-1/7}) = -\frac{5}{7} \zeta_{7}^{5} - \frac{3}{7} \zeta_{7}^{4} - \frac{1}{7} \zeta_{7}^{3} + \frac{1}{7} \zeta_{7}^{2} + \frac{3}{7} \zeta_{7} - \frac{2}{7} = 0.623489... + 1.29468...i. $$
The numerical algorithm gives $a_1(f_{-1/7}) = 0.623489801858733...+ 1.29468991410431...i$. Hence our formulae are 
consistent for this example.
\end{Example}

\subsection{A result linking first term to critical polynomials}

One motivation to study the factorization of $a_1(f_{z_p})$ as a principal fractional ideal in $\bQ(\zeta_p)$ is that they relate to critical points of modular parametrizaiton of $E$ in the following way:
\begin{theorem} \label{thm: denominator}
Suppose there exists a prime ideal $\fq$ in $\bQ(\zeta_p)$ lying above a prime $q \neq p$, such that $\fq \mid a_1(f_{z_p)}$ and  $\ord_q (a_1(f_{z_p})) < \frac{p-1}{2}$. Then $F_{E,j}(x)$ is not integral at $q$.
\end{theorem}

To prove the theorem, first we make some preparations in $p$-adic analysis. Let $p$ be a prime, $\bC_p$ be a completion of a choice of $\bar{\bQ}_p$. Let $q$ be a formal variable.  Let $\ord_p$ denote the p-adic valuation on $\bC_p$ and let $D(1)$ denote the open unit disk $$D(1) = \{x \in \bC_p: \ord_p(x) > 0\}.$$
 
\begin{Lemma} \label{newton polygon}
Let $f = 1 + \sum a_n q^n \in \bC_p[[q]]$ be such that $f$ converges on $D(1)$. Then the following are equivalent: \\
(1)  there exists some $i$ such that $\ord_p(a_i) < 0$. \\
(2)  there exists $\alpha \in D(1)$ with $f(\alpha) = 0$.
\end{Lemma}

\begin{proof}
Consider the first segment of the Newton polygon of $f$. Assume (1) then the segment is necesarily finite, since otherwise $f$ does not . Hence by a theorem on Newton polygon, we have (2); now assume that (2) holds. Let $\lambda = -\ord_p(\alpha) < 0$. Assume towards contradiction that (1) is false. Let $N$ be the total horizontal length of all segments of $N(f)$ with slope $\leq \lambda$. The assumption then implies $N = 0$. Hence by Weierstrass preparation theorem, we know $f$ is nonzero on the closed disc $D(|\alpha|_p^{+})$, a contradiction to (2). 
\end{proof}

\begin{Prop}
Suppose $K/\bQ_p$ is a finite extension with uniformizer $\pi$. Let $f_j: 1 \leq j \leq n$ be power series with constant terms one and let $F =  \prod_j f_j$.  Suppose there exists $i$ such that $\ord_\pi(a_i(f_1)) < 0$. Then there exists an index $i' \geq 1$ such that $\ord_\pi(a_{i'}(F)) < 0$.
\end{Prop}

\begin{proof}
By Lemma~\ref{newton polygon}, the condition implies that $f_j$ converges on $D(1)$ for all $j$ and $f_1$ has a root in $D(1)$. Since $F$ is the product of the $f_j$'s, we know that $F$ has a root in $D(1)$. Hence the claim follows, again using Lemma~\ref{newton polygon}. 
\end{proof}

Fix a prime $\fp$ above $p$ in $\bar{\bQ}$. We say a laurent series $f \in \bar{\bQ}((q))$ is called integral at $p$ if $f = q^s(1 + \sum_{n \geq 1} a_n q^n)$ with $\ord_p(a_n) \geq 0$ for all $n$. One sees that if $f$ is integral, then $\frac{1}{f}$ is also integral. Product of two integral power series is integral. We similarly define the notion for a monic polynomial $F[x] \in \bQ[x]$ to be integral at $p$. 



Let $K/\bQ$ be a cyclic extension with Galois group $G$. Let $n = [K:\bQ]$. Let $l$ be a prime, unramified in $K$. 
Let $I \subseteq \cO_K$ be an ideal whose norm is a power of $l$, and let $H = \{\sigma \in G: \sigma(I) = I\}$ be the stabilizer of $I$ under the action of $G$. 
\begin{Lemma}
Assume that $H$ contains $G^2$, i.e., the subgroup of squares in $G$. Then $\ord_l(Norm(I)) \geq n/2$. 
\end{Lemma}


\begin{proof}
Really we have two cases: $H = G$ and $H = G^2$. In the first case, $I$ is necessarily a power of $l$. Hence 
$\ord_l(Norm(I)) \geq n$ indeed; in the second case, Let $\fl_1 ,\ldots, \fl_g$ denote the primes above $l$. Since $G$ is abelian, we know $Stab(\fl_i) = Stab(\fl_1)$ for any $i$, so let $H_0$ denote that stabilizer.  The action of $G/H_0$ on the set $\{ \fl_1 ,\ldots, \fl_g \}$ provides an embedding it as a cyclic subgroup of $S_g$, generated by a $g$-cycle $\tau$. If $g$ is odd; then $\tau^2$ is another $g$-cycle, so it does not fix any proper subset of $\{1,2, \ldots, g\}$, so again I is divisible by $l$; suppose $g$ is even, then $\tau^2$ factors as a product of two $(g/2)$-cycles $s_1$ and $s_2$. 
WLOG $s_1 = (135 \cdots g-1)$. Then $I = (\fl_1 \fl_3 \ldots \fl_{g-1})^t$ for some positive integer $t$. Hence 
$Norm(I) = l^{nt/2} \geq l^{n/2}$.  This completes the proof. 
\end{proof}

\begin{Lemma} \label{non-integral}
For any prime $p$, $F_{E,j}$ is integral at $p$ if and only if $\Norm(f)$ is integral at $p$. 
\end{Lemma}

\begin{proof}
Note that $\Norm(f)$ is integral at $p$ if and only if $F_f(q)$ is . Now we use the fact that $F_{E,j}(j(q)) = F_f(q)$. Suppose 
$F_{E,j}$ is integral at $p$. Since $j(q)$ is integral at $p$, the coefficients of $F_f(q)$ have nonnegative valuations. Moreover, since $F_{E,j}(x$) is monic, the leading coefficient of $F_f(q)$ is 1. Hence $F_f(q)$ is integral at $p$. Now assume that $F_{f}(q)$ is integral at $p$. By examining Algorithm~\ref{NORM}, we see that the coefficients of $F_{E,j}$ lies in the ring generated over $\bZ$ by the coefficients of $F_f$ and coefficients of $j$. In particular, $F_{E,j}$ is integral at $p$.
\end{proof}


\begin{Prop} \label{prop: denominator}
Let $b_1, \ldots, b_{p-1}$ be the ``first terms"  in our case. Assume for some prime $l$ we have $$0 < \ord_l(Norm(b_1)) < \frac{p-1}{2}.$$ Then $Norm(f)$ is not integral at $l$. 
\end{Prop}

\begin{proof}
Since $\prod_{cusp z} \tilde{f}_z$ divides $\Norm(f) = \prod \widetilde{f|A_i}$,  by Lemma~\ref{newton polygon}, the claim will follow if we can show there exists $z$ and a prime ideal $\fl$ such that the normalized series $\tilde{f_z}(q) = \sum a_n b_n/b_1 q^n$ has a non-$\fl$-integral coefficient. So let us assume that this is not the case. Let $I$ be the $l$-part of the principal ideal $(b_1)$, and let $H \leq G = (\bZ/p\bZ)^{\times}$ denote the stablizer of $I$.  The assumption implies that $H$ does not contain $G^2$. Hence there exists an integer $m$ such that (1) $m$ is a square modulo $p$; (2) $\sigma_m(I) \neq I$; (3) $j \mod{p} \notin H$. Pick such an integer $m$ and set $c_m= b_m/b_1$. Then from (2) we see that there exists some prime ideal $\fl$ above $l$ such that $\ord_\fl(c_m) < 0$. By Dirichlet's theorem on primes in arithmetic progressions, we can find a prime $r \neq l$ such that $l'^2 \equiv m \mod{p}$. Then $r \mod{p} \notin H$. 
Hence there exists some prime $\fl'$ above $l$ such that $\ord_{\fl'}(c_r) < 0$. Recall that for any $n$ we have $a_n(f_z) = a_n(f) c_{n \mod {p}}$. Suppose towards contradiction that $\tilde{f_z}(q)$ is $l$-integral. Then $a_r(f)$ and $a_{r^2}(f)$ must be both divisible by $l$. But $a_{r^2} = a_r^2 - r$, so $r = l$, a contradiction. 
\end{proof}

Now we can see that Theorem~\ref{thm: denominator} is a direct consequence of Proposition~\ref{prop: denominator} and Lemma~\ref{non-integral}. 
   
\subsection{Data}
From the 
above discussion, we see that there are at most three possibilities for each $p$, corresponding to the order of $\varphi$ being 3,4 or 6. 

Consider $N_{f,p} = Norm(a_1(f_{z_p})) \in \bZ$. It is easy to show that we always have $p \mid N_{f,p}$. The following is a table of the prime divisors of the norm, when such primes exist.

\begin{table}
\centering
\caption{table of prime divisors $l \neq p$ of $N_{f,p}$}
\begin{tabular}{c|c|c}
$p$ & order of $\varphi$ & primes  $l \neq p, l \mid N_{f,p}$ \\ \hline
17 & 3 & 509 \\
19 & 4 & 37 \\
23 & 3 & 1103 \\
23 & 4 & 47 \\
29 & 3 & 173 \\
31 & 4 & 557 \\
41 & 3 & 1209, 9103 \\
41 & 6 & 163 \\
43 & 4 & 4129 \\
47 & 3 & 13034039 \\
47 & 4 & 2819 \\
53 & 3 & 107, 317, 8161 \\
53 & 6 & 107  \\
59 & 3 & 1061, 537173407 \\
59 & 4 & 827, 42953 \\
67 & 4 & 2143, 10853 \\
71 & 3 & 634532719903 \\
71 & 4 & 6613947917 \\
71 & 6 & 3407 
\end{tabular}
\end{table}

As an observation, we found that the primes $l$ in the third column of the above table all satisfy a congurence relation
\[
	l \equiv \pm 1 \mod{p}. 
\]
It would be interesting to prove or disprove this in general. 

\chapter{Index of Chow-Heegner points}

We consider a special case of the Chow-Heegner points that has a simple description due to Shouwu Zhang. Let $E,F$ be nonisogenous elliptic curves defined over $\bQ$ of the same conductor $N$. The Chow-Heegner point $P_{E,F} \in E(\bQ)$ is constructed by the following procedure: take any point on $F(\bC)$, take its inverse image on $X_0(N)$, then map that image down to $E$ and take the sum the resulting points. In \cite{darmon2015algorithms}, Darmon, Daub, Lichtenstein and Rotger developed an algorithm to compute Chow-Heegner points via iterated integrals. In \cite{stein2012numerical}, Stein developed a fast and conceptually easy algorithm to numerically compute Chow-Heegner points. The following theorem is proved by Yuan-Zhang-Zhang in \cite{yuan2011triple}: 
\begin{theorem}[Yuan-Zhang-Zhang] \label{thm: yzz}
Let $L(E, F, F, s) = L(E, s) � L(E, Sym^2(F), s)$.
Assume that the local root numbers of $L(E,F,F,s)$ at every prime of bad reduction is +1 and that the root number at 
infinity is $-1$. Then 
\[
	\hat{h}(P_{E,F}) = (\star) \cdot L'(E,F,F,\frac{1}{2}),
\]
where $(\star)$ is nonzero.
\end{theorem}
In particular, when the analytic rank of $E$ is at least two, the Chow-Heegner point $P_{E,F}$ is torsion. When the rank of $E(\bQ)$ is one, we consider the index $i_{E,F} = [E(\bQ)/tors: \bZ P_{E,F}]$. Theorem~\ref{thm: yzz} combined with the Bloch-Kato conjecture on critical values of motivic $L$-functions suggests that this index might be linked to interesting arithmetic invariants related to $E$ and $F$.  \\

Numerical evidence in  \cite{stein2012numerical} suggests that the index $i_{E,F}$ is always divisible by 2, when it is finite. I proved the following theorem.
\begin{theorem} \label{thm: evenindex}
Let $\sigma_0(N)$ denote the number of distinct prime factors of $N$. If 
\[
	\sigma_0(N) > \log_2(\# E[2](\bQ)) + \log_2(\# F[2](\bQ)) + 2,
\]
then $P_{E,F} \in 2E(\bQ)$. Hence the index $i_{E,F}$ is divisible by 2, if it is finite.
\end{theorem}

I prove the theorem in Section 5.2. In Section 5.3,  I develop an exact algorithm to compute the Chow-Heegner point, using the methods in Chapter 2. 
\section{Definitions}

We recall the definition from \cite{stein2012numerical}. Consider a pair $E,F$ of nonisogenous optimal elliptic curves over $\bQ$ of the same conductor $N$ and fix modular parametrizations from $X_0(N)$ to both curves. 

Let $(\varphi_E)_*$ and $(\varphi_F)^*$ denote the pushforward and pullback map on divisors. Let $Q \in F(\bC)$ be any point, we define $$P_{E,F, Q} = \sum (\varphi_E)_* (\varphi_F)^*(Q),$$ where $\sum$ means the sum of the points in the divisor, using the group law on $E$. By \cite[Proposition~1.1]{stein2012numerical}, $P_{E,F,Q}$ is independent of the choice of $Q$. Let $P_{E,F} = P_{E,F,Q}$ for any choice of $Q$. Since we may choose $Q  = \cO \in \bF(\bQ)$, it follows that 
$P_{E,F} \in E(\bQ)$. 

\section{The index}

In this section, we make the additional assumption $r_{an}(E) = 1$. Consider the index $$i_{E,F} = [E(\bQ)/tors: \bZ P_{E,F}].$$  We quote a lemma of Calegari and Emerton \cite{calegari2009elliptic}.

\begin{Lemma}[\cite{calegari2009elliptic}]
Let $E/k$ be an elliptic curve and let $A$ be the group of automorphisms of $E$ as a curve over $k$.
Suppose $W$ is a finite elementary abelian 2-subgroup  of $A$. Then the order of $W$ divides
twice the order of $E[2](k)$.
\end{Lemma}




\begin{theorem}
Let $E,F$ be elliptic curves defined over $\bQ$, with the same conductor $N$. Let $\sigma_0(N)$ denote the number of distinct prime factors of $N$. If 
\[
	\sigma_0(N) > \log_2(|E(\bQ)[2]|) + \log_2(|F(\bQ)[2]|) + 2,
\]
then $P_{E,F} \in 2E(\bQ)$. In particular, if $\sigma_0(N) \geq 7$, then the condition holds automatically, and $P_{E,F} \in 2E(\bQ)$.
\end{theorem}

\begin{proof}
Consider the group $\cW$ of Atkin-Lehner involutions on $X_0(N)$. This group is elementary 2-abelian,
and it descends to automorphisms on $F$ and automorphisms on $E$, as curves. So we have a map 
\[
	\pi: \cW \to Aut(E) \times Aut(F)
\]
By the Lemma above, we have $im(p_1 \circ \pi) \leq 2|E[2](\bQ)|$ and $im(p_2 \circ \pi) \leq 2|F[2](\bQ)|$. Hence the size of the image of $\pi$ is bounded above by $4 |E(\bQ)[2]| \cdot |F(\bQ)[2]|$. But we also know that
\[
	|\cW|  = 2^{\sigma_0(N)}.
\]
Hence our assumption implies that $\ker(\pi)$ is nontrivial. Equivalently, there exists $w \in \cW$ that acts as identity on both $E$ and $F$. Now we consider the following diagram:
\begin{center}
\begin{tikzcd}
& X_0(N) \arrow[d]  \arrow[ddr]  \arrow[ddl]& \\
& X_0(N)/w \arrow[dl] \arrow[dr] &\\
E &  & F
\end{tikzcd}
\end{center}
Let $\cO \in F(\bQ)$ be the identity element. 
We have $P_{E,F} = P_{E,F,\cO} = \sum (\varphi_{E})_* (\varphi_F)^*(O) = \sum (\tilde{\varphi_E})_* \pi_* \pi^* \tilde{\varphi_F}^*(O) =   2\sum (\tilde{\varphi_E})_* \tilde{\varphi_F}^*(O) \in 2E(\bQ)$. 
\end{proof}


%Question: Can we sharpen this? Note that we can have $W(z) = z + P$, because this would mean
%that the Atkin-Lehner has no fixed point, which is possible only when $Q  \neq N$.

\section{Applying the idea of \textbf{IPR} to the computation of Chow-Heegner points}

We develop an algorithm to compute the Chow-Heegner point $P_{E,F}$. Let $x_E, y_E$, $x_F,y_F$ be the compositions of $\varphi, \psi$ with the $x$ and $y$ coordinate functions on $E$ and $F$, respectively. Note that there exists an algorithm to compute the q-expansions of $x_E,x_F,y_E$ and $y_F$.  

\red{fixme: talk about how to use PARI to get $x_E(q)$}

\begin{algorithm}
\caption{Using polynomial relation to compute the Chow-Heegner point $P_{E,F}$}          % give the algorithm a caption
\label{IPR}                           % and a label for \ref{} commands later in the document
\begin{algorithmic} [1]               % enter the algorithmic environment
    \Require  $E, F$  = non-isogeneous elliptic curves of conuductor $N$; $q$-expansions of $x_E, y_E$, $x_F,y_F$. 
    \Ensure the Chow-Heegner point $P_{E,F}$. 
    \State{$u_E \Leftarrow (x_F)^{-1}$ and $u_F \Leftarrow (x_F)^{-1}$}. 
    \State{Mimicing steps 4-7 of Algorithm~\ref{IPR},  compute an irreducible polynomial $F(x,y)$ such that $F_{E,F}(u_E,u_F) = 0$}. 
    \State $f_{ch,x}(x) \Leftarrow F_{E,F}(x,0)$.
    \State Repeat steps 2-5 for $v_E = (y_E)^{-1}$ and $u_F$, get $f_{ch,y}(x)$.
    \State $K \Leftarrow$ the splitting field of $f_{ch,x}$. Write $f_{ch,x}(x) = \prod (x-a_i), a_i \in K$.
    \For{each $a_i$}
	\State{Find a point $p_i = (a_i, b_i)$  on $E(\bar{\bQ})$}.
	\If{$f_{ch,y}(b_i) = 0$}
		\State $P_i = p_i$.
	\Else
         	\State $P_i = -p_i$. 
	\EndIf
    \EndFor
\State Output $P_{E,F} = \sum_i P_i$.

\end{algorithmic}
\end{algorithm}


\begin{Example}
Consider $E = \textbf{89a}$ and $F = \textbf{89b}$. Here $\deg(\varphi_E) = 2$ and $\deg(\varphi_F) = 5$. Let 
$D =  \varphi(\psi^*(\infty)) \in \Div(E)$. Define $G_1(x) = \prod_{P \in D} (x - x(P))$ 
and $G_2(y) = \prod_{P \in D} (y - y(P))$. Using \textbf{IPR-CH}, we computed 
\[
G_1(x) = x^{4} + \frac{13}{4} x^{3} + \frac{17}{4} x^{2} + \frac{21}{4} x + \frac{9}{2}, \; G_2(y) = y^{4} + \frac{1}{8} y^{3} + \frac{21}{4} y^{2} + \frac{7}{2} y + 3.
\]
\end{Example}

It turns out that $G_1(x)$ is irreducible over $\bQ$. Let $K$ be its splitting field, and write $G_1(x) = \prod (x-a_i)$ with $a_i \in K$. For each $a_i$, we found that $b_i = -\frac{8}{9} a_i^{3} - \frac{20}{9} a_i^{2} - \frac{28}{9} a_i - \frac{10}{3}$ is the corresponding root of $G_2$ such that $(a_i,b_i) \in E$. Hence 
\[
	P_{E,F} = \sum_{i=1}^4 P_i, \; \mbox{ where } P_i = (a_i, b_i).
\]
Carrying out the summation in Sage, we obtain $P_{E,F} = (\frac{3}{4},-\frac{15}{8})$. This agrees with Stein's result for 
the pair (\textbf{89a},\textbf{89b}) in \cite{stein2011numerical}.




\chapter{Things I tried to do but did not end up giving a nice result}

generalizing the ``congruence number'' definition using other cusps.  

Prove the ``$\pm 1$ mod p'' guess. 

Generalize another paper by William on computing order of component groups. (The original paper uses a trick). 


Computing the critical subgroup for 5077a (multimodular is not practical). 

Critical points of reduction of modular parametrization. 


\nocite{*}
\bibliographystyle{alpha}
\bibliography{uwthesis}


\end{document}
